%
% %CopyrightBegin%
%
% Copyright Ericsson AB 2017. All Rights Reserved.
%
% Licensed under the Apache License, Version 2.0 (the "License");
% you may not use this file except in compliance with the License.
% You may obtain a copy of the License at
%
%     http://www.apache.org/licenses/LICENSE-2.0
%
% Unless required by applicable law or agreed to in writing, software
% distributed under the License is distributed on an "AS IS" BASIS,
% WITHOUT WARRANTIES OR CONDITIONS OF ANY KIND, either express or implied.
% See the License for the specific language governing permissions and
% limitations under the License.
%
% %CopyrightEnd%
%

\chapter{Summary of differences between \OldErlang\ and \StdErlang}

\begin{Lentry}
\item[\ifStd\S\ref{section:unicode}\fi\ifOld\S\ref{chapter:lexical}\fi]
\OldErlang\ uses the ASCII character set, in which character codes are 7 bits wide.
In \StdErlang\ the character set is Unicode, in which character codes are 16 bits wide.
\NewErlang\
would be expected to support at least the 256 first Unicode characters, which coincide with
Latin-1.

This leads to minor differences between \OldErlang\ and \StdErlang\
in many parts of the specification where characters are mentioned.

Also there are Unicode escape sequences in \StdErlang\ but not in \OldErlang.

\item[\S\ref{section:keywords}]
\StdErlang\ has three keywords that \OldErlang\ does not: \T{all_true}, \T{some_true}
and \T{try}.

\item[\S\ref{section:operators}]
\StdErlang\ has two operators that \OldErlang\ does not: \T{//} and \T{mod}.

\item[\S\ref{section:escapes}]
\StdErlang\ has an escape sequence that \OldErlang\ does not: \T{\char`\\s} for space.

In \OldErlang, an octal escape may be from \T{\char`\\000} to \T{\char`\\777}
while in \StdErlang, only bytes can be expressed, i.e.,
\T{\char`\\000} to \T{\char`\\377}.

\item[\S\ref{section:integer-literals}]
In \StdErlang\ integer literals may contain underscore characters to separate groups of digits.

\item[\S\ref{section:float-literals}]
In \StdErlang\ float literals may contain underscore characters to separate groups of digits.

\item[\S\ref{section:char-literals}]
In \OldErlang\ a \T{\char`\$} character may be followed by any character and its value would be
the code of that character.  In \StdErlang\ a \T{\char`\$} character must not be followed by
whitespace.

\item[\S\ref{section:string-literals}]
In \OldErlang\ a control character may be included in a string literal.
In \StdErlang\ it must not.

\item[\S\ref{section:atoms}]
In \StdErlang\ the maximum length of atoms' printnames is an implementation-dependent
constant while in \OldErlang\ it is always 255.

\item[\S\ref{section:chars}]
In \StdErlang\ there is a separate character type while in \OldErlang, integers in the
range $[0,127]$ serve this purpose.

\item[\S\ref{section:functions}]
In \OldErlang, functions are transparently represented by tuples, while in
\StdErlang\ they must be a proper type.

\item[\S\ref{chapter:arithmetics}]
In \StdErlang, the arithmetics have been generalized, as compared with
\OldErlang, in the sense that there are
various requirements on the arithmetics but few exact prescriptions.
For example, an implementation may
or may not have bignums and the floating-point representation can vary
between implementations (and on platforms). The international standard LIA-1 has been followed in
the description.

In \OldErlang, the exit cause for all arithmetic errors is \T{badarith} (except that
a few BIFs will give \T{badarg} instead) while in
\StdErlang\ there are four exit causes, depending on the problem.

\item[\S\ref{section:scope}]
The scope of variables is different in \OldErlang\ and \StdErlang. In \StdErlang\
the binding occurrence can be determined at compile-time while in \OldErlang\ it
cannot, in general.  This forces \OldErlang\ to verify that for \emph{any} order of
evaluation that can be chosen, no applied variable occurrence will be unbound.

\item[\S\ref{section:evorder}]
In \OldErlang\ it is not defined in which order the subexpressions of an expression
will be evaluated, except that a body is evaluated from left to right.  In \StdErlang\
there is a left-to-right order also for arguments of function applications, list
literals, etc.

\item[\S\ref{section:patterns}]
In \StdErlang\ there is an additional pattern \T{$\Z{P}_1$ = $\Z{P}_2$} that will
match a term if, and only if, both $\TZ{P}_1$ and $\TZ{P}_2$ match it.

\item[\S\ref{section:function-application}]
The order in which shadowing will be resolved is different in \OldErlang\ and
\StdErlang.

\item[\S\ref{section:appl-unnamed-function}]
If an application of an explicit \T{fun} expression completes
abruptly because there is no clause with a matching pattern and a successful
guard, it exits with reason
\T{\{lambda_clause,\{\Z{Mod},\Z{T},[$\Z{v}_1$,\tdots,$\Z{v}_{\TZm{n}}$]\}\}}
in \StdErlang\ but
\T{\{lambda_clause,\{\Z{Mod},[$\Z{v}_1$,\tdots,$\Z{v}_{\TZm{n}}$]\}\}}
in \OldErlang.  \TZ{T} is an implementation-defined term.

\item[\S\ref{section:catch}]
In \OldErlang\ \T{catch} expressions are the only expressions for restoring
normal mode of evaluation.

\item[\ifStd\S\ref{section:intdiv-f}\fi\ifOld\S\ref{section:multiplicative}\fi]
In \StdErlang\ there is an operator \T{//} for integer division with rounding
towards negative infinity.

\item[\ifStd\S\ref{section:intmod}\fi\ifOld\S\ref{section:multiplicative}\fi]
In \StdErlang\ there is an operator \T{mod} for integer remainder with rounding
towards negative infinity.

\item[\S\ref{section:unaryplus}]
In \OldErlang\ the unary \T{+} operator can be applied to a term of any type.
In \StdErlang\ it can only be applied to numbers.

\item[\S\ref{section:tuple-skeletons}]
In \StdErlang\ there is no restriction on the size of tuple skeletons while in
\OldErlang\ they can have at most 255 elements.

\item[\ifStd\S\ref{section:alltrue-exprs}\fi\ifOld\S\ref{section:primary-exprs}\fi]
In \StdErlang\ there are expressions \T{all_true $\Z{E}_1$ ; \tdots ; $\Z{E}_k$ end}
for expressing nonstrict conjunction.

\item[\ifStd\S\ref{section:sometrue-exprs}\fi\ifOld\S\ref{section:primary-exprs}\fi]
In \StdErlang\ there are expressions \T{some_true $\Z{E}_1$ ; \tdots ; $\Z{E}_k$ end}
for expressing nonstrict disjunction.

\item[\ifStd\S\ref{section:cond-expr}\fi\ifOld\S\ref{section:primary-exprs}\fi]
In \StdErlang\ there are \T{cond} expressions
for expressing \T{if}-like choices but where the conditions are arbitrary Boolean
expressions rather than guards.

\item[\ifStd\S\ref{section:try-expr}\fi\ifOld\S\ref{section:primary-exprs}\fi]
In \StdErlang\ there are \T{try} expressions that are more powerful than
\T{catch} expressions for capturing abrupt completion.

\T{try} expressions have the following advantages over \T{catch} expressions:
\begin{itemize}
\item It is possible to distinguish safely between the three possible
outcomes of evaluating an expression:
\begin{itemize}
\item Normal completion.
\item Abrupt completion due to a \T{throw}.
\item Abrupt completion due to an \T{exit} (possibly caused by an error).
\end{itemize}
As an example, here is how to define a function \T{foo_with_log/1} which
is exactly like a function \T{foo/1} except that on abrupt completion,
the reason is logged:
\begin{verbatim}
foo_with_log(X) ->
    try
        foo(X)
    catch
        {'THROW',T} ->
            io:format(logger,
                      "foo(~w) threw ~w.\n",
                      [X,T]),
            throw(T) ;
        {'EXIT',E} ->
            io:format(logger,
                      "foo(~w) exited due to ~w.\n",
                      [X,E]),
            exit(E)
    end.
\end{verbatim}
\item It is possible to catch only certain thrown terms or exits.  For example,
in the following function, throws of a particular reference \T{BadSyntax} are
caught but any other thrown terms or exits still cause abrupt completion of the
function and can be caught outside it:
\begin{verbatim}
read_entry_safely(Device) ->
    BadSyntax = make_ref(),
    try
        read_entry(Device,BadSyntax)
    catch
        {'THROW',BadSyntax} ->
            recover(Device),
            read_entry_safely(Device)
    end.
\end{verbatim}
The function above shows a useful programming idiom.
\item \T{try} expressions do not have the syntactic inconvenience of \T{catch}
expressions (that many expressions have to be enclosed in parentheses when
being the argument of \T{catch} and that one has to use \T{begin} \ldots\ \T{end}
in order to give a body to \T{catch}).
\end{itemize}

\item[\S\ref{section:signals}]
In the signal model of \OldErlang, there are three kinds of signals that are not included
in \StdErlang: \I{Suspend/resume request}, \I{Garbage collection request} and
\I{Trace/notrace request}.

\item[\S\ref{section:process-state-dynamic}]
In \StdErlang\ it is possible for a process to \emph{monitor} another. It is like
monitoring of nodes but for processes.  It is like linking but one-way.

\item[\S\ref{section:magic-cookie}]
In \OldErlang\ the magic cookie mechanism is used only for messages, not for other communication.
In \StdErlang\ all communication between nodes is subject to magic cookies.

\item[\S\ref{section:opening-ports}]
In \OldErlang\ the first argument to the BIF \T{open_port/2} can be an atom or
a string.  This is not included in \StdErlang\ (but \NewErlang\ may well inlude it).

\item[\S\ref{chapter:bifs}]
There are (proposed) changes to the exit causes.  For example:

In \StdErlang, only if abnormal completion of a BIF was because there was something
wrong with the
value of an argument, e.g., is was of the wrong type or it was an index
outside the meaningful range, the exit cause will be \T{badarg}.  In \OldErlang,
\T{badarg} is given for various conditions.

In \StdErlang\ some BIFs will give a new exit \T{badindex} if they are given an
index argument that is not in the permitted range.

\item[\S\ref{section:recognizer-bifs}]
In \StdErlang\ the recognizer BIFs all have names that begin with \T{is_} and there
have been some additions and changes in the repertoire.

\item[\ifStd\S\ref{section:sign1}\fi\ifOld\S\ref{section:number-bifs}\fi]
In \StdErlang\ there is a new BIF \T{sign/1}.

\item[\S\ref{section:termtobinary1}]
The presence of functions in \StdErlang\ and the fact that they cannot be portably
communicated means that \T{term_to_binary/1} may exit if given a term containing a
function.

\item[\ifOld\S\ref{section:listtopid1}\fi\ifStd\S\ref{section:process-bifs}\fi]
The BIFs \T{list_to_pid/1} and \T{pid_to_list/1} are not part of \StdErlang\
(but may well be included in \NewErlang).

\item[\S\ref{section:atom-tables}]
In \StdErlang\ the size of atom tables is an implementation-dependent constant
whereas in \OldErlang\ it is always 256.
\end{Lentry}
