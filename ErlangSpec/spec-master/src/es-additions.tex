%
% %CopyrightBegin%
%
% Copyright Ericsson AB 2017. All Rights Reserved.
%
% Licensed under the Apache License, Version 2.0 (the "License");
% you may not use this file except in compliance with the License.
% You may obtain a copy of the License at
%
%     http://www.apache.org/licenses/LICENSE-2.0
%
% Unless required by applicable law or agreed to in writing, software
% distributed under the License is distributed on an "AS IS" BASIS,
% WITHOUT WARRANTIES OR CONDITIONS OF ANY KIND, either express or implied.
% See the License for the specific language governing permissions and
% limitations under the License.
%
% %CopyrightEnd%
%

\chapter{Differences from \OldVersion}

This appendix summarizes the differences between \OldVersion\ and \NewVersion.

\section{Characters}

In \OldVersion\ there was no character type --- a character was represented
by its character code --- an integer.  It was not documented which
character codes were valid but the ASCII codes (0--127) have always been supported.
In \NewVersion, a character type as been added and it uses the Unicode
character encoding.

\section{The evaluation order is defined}

\label{section:new-evaluation-order}

It used to be that the evaluation order between subexpressions in
\Erlang\ was defined in as few cases as possible.  (For example, the
subexpressions of a body were evaluated in order but the operands of
binary operators and the arguments of function applications could be
evaluated in any order.)
This had some advantages:
\begin{itemize}
\item Each implementation was free to exploit the order of evaluation
that was most efficient on its target architecture.
\item No evaluation order was easy to describe.
\end{itemize}
However, there were also disadvantages:
\begin{itemize}
\item \Erlang\ expressions provide bindings for variables and this either
puts implicit constraints on the evaluation order or complicated
scope rules.
\item It is undesirable that for some expressions, their exact effects were not
well defined.  (This also means that for some expressions, their values were
not well defined.)
\end{itemize}
For \NewVersion\ it was thus decided to define an evaluation order.

\section{Control escapes more restricted}

\label{section:new-control-escapes}

Some minor differences for control escapes (in string literals and
quoted atoms):
\begin{itemize}
\item \OldVersion\ allows any character after \TXT{\char`\\} \TXT{\char`\^}.
Use of other characters than \T{@}--\T{_} should render warnings in
\NewVersion\ and be prohibited in some later version.
\item \OldVersion\ allows any octal numeral with up to three digits in
a \NT{OctalEscape}. Use of octal numerals \T{400}--\T{777} should be
an error in \NewVersion.
\end{itemize}

\section{Function terms form their own type}

\label{section:new-functions}

Due to the implementation of function terms as tuples in \OldVersion,
function terms were not properly distinguishable from tuples.
\NewVersion\ requires that
function terms form a distinguishable elementary type.

BIFs?

\iffalse
\section{Records form their own type}

\label{section:new-records}

Due to the implementation of records as tuples in \OldVersion,
they were not properly distinguishable.  \NewVersion\ requires that
records form a distinguishable elementary type.

BIFs?
\fi

\section{Operator precedence of match and send changed}

The relative precedence of the operators \T{=} and \T{!}\ was not defined previously but
as a matter of fact they were the same, so the expressions
\T{$\Z{E}_1$ = $\Z{E}_2$ ! $\Z{E}_3$} and \T{$\Z{E}_1$ ! $\Z{E}_2$ = $\Z{E}_3$}
were both allowed, meaning the same as the expressions
\T{$\Z{E}_1$ = ($\Z{E}_2$ ! $\Z{E}_3$)} and \T{$\Z{E}_1$ ! ($\Z{E}_2$ = $\Z{E}_3$)},
respectively.

This grouping indicated a closer relationship between the operators
than is actually the case and in \ErlVsn{4.7} and onwards, only the
expression \T{$\Z{E}_1$ = $\Z{E}_2$ ! $\Z{E}_3$} can now be written
without parentheses.

\iffalse
\section{Integer division rounds towards negative infinity}

\label{section:integer-division}

In \OldVersion, the operators \T{div} and \T{rem} compute $\mathit{div}_I^t$
and $\mathit{rem}_I^t$, respectively.  It is widely held to be more useful
if $\mathit{div}_I^f$ and $\mathit{rem}_I^f$ are easily accessible as
$\mathit{div}_I^f$ satisfies the relation
\[\mathit{div}_I^f(x+i*y,y) = \mathit{div}_I^f(x,y)+i\]
if $y\neq0$ and no overflow occurs.
\fi

\section{\T{try} expressions replace \T{catch} expressions}

\label{section:new-try}

The new \T{try} expressions generalize the old \T{catch} expressions,
which should not be used in any new code.  \T{try} expressions have
the following advantages over \T{catch} expressions:
\begin{itemize}
\item It is possible to distinguish safely between the three possible
outcomes of evaluating an expression:
\begin{itemize}
\item Normal completion.
\item Abrupt completion due to a \T{throw}.
\item Abrupt completion due to an \T{exit} (possibly caused by an error).
\end{itemize}
As an example, here is how to define a function \T{foo_with_log/1} which
is exactly like a function \T{foo/1} except that on abrupt completion,
the reason is logged:
\begin{verbatim}
foo_with_log(X) ->
    try
        foo(X)
    catch
        {'THROW',T} ->
            io:format(logger,
                      "foo(~w) threw ~w.\n",
                      [X,T]),
            throw(T) ;
        {'EXIT',E} ->
            io:format(logger,
                      "foo(~w) exited due to ~w.\n",
                      [X,E]),
            exit(E)
    end.
\end{verbatim}
\item It is possible to catch only certain thrown terms or exits.  For example,
in the following function, throws of a particular reference \T{BadSyntax} are
caught but any other thrown terms or exits still cause abrupt completion of the
function and can be caught outside it:
\begin{verbatim}
read_entry_safely(Device) ->
    BadSyntax = make_ref(),
    try
        read_entry(Device,BadSyntax)
    catch
        {'THROW',BadSyntax} ->
            recover(Device),
            read_entry_safely(Device)
    end.
\end{verbatim}
The function above shows a useful programming idiom.
\item \T{try} expressions do not have the syntactic inconvenience of \T{catch}
expressions (that many expressions have to be enclosed in parentheses when
being the argument of \T{catch} and that one had to use \T{begin} \ldots\ \T{end}
in order to give a body to \T{catch}).
\end{itemize}

\section{Sending to an unregistered process name}

\label{section:unregistered-name}

In \OldVersion, the documentation claims that if process $\TZ{P}_1$ evaluates
the send operator (\T{!})\ with
(evaluated) operands $\TZ{v}_1$ and $\TZ{v}_2$ where $\TZ{v}_1$ is an
atom that is not the registered name of any process,
the result of the send expression is obtained by evaluating
\begin{alltt}
\Z{M}:unregistered_name(\Z{v}\(\sb{1}\),\Z{P}\(\sb{1}\),\Z{v}\(\sb{2}\))
\end{alltt}
where \TZ{M} is the value of the \T{error_handler} process flag for
$\TZ{P}_1$.  The function
\T{error_handler:unregistered_name/3} which is the default error handler
for undefined process names simply exits with \T{badarg}.

In \NewVersion, the evaluation of the send operator simply has no effect
in this case.

\section{New exit causes}

\label{section:new-exits}

In \OldVersion\ many BIFs would exit with \T{badarg} regardless of the actual
reason for abnormal completion.  In \NewVersion, a BIF or operator will exit
with \T{badarg} if, and only if, the exit was because an argument or operand was
of a type not permitted for that position.  This means that some exits that
were previously \T{badarg} (e.g., when \T{register/2} was given a name that
was already registered)) are now different.  

Arithmetic exits are now on the form \T{\{arith,\Z{A}\}}, where \TZ{A} is one of
the atoms \T{integer_overflow} (should not happen in an implementation with
bignums), \T{float_overflow}, \T{float_underflow} and \T{undefined_arith}.
