%
% %CopyrightBegin%
%
% Copyright Ericsson AB 2017. All Rights Reserved.
%
% Licensed under the Apache License, Version 2.0 (the "License");
% you may not use this file except in compliance with the License.
% You may obtain a copy of the License at
%
%     http://www.apache.org/licenses/LICENSE-2.0
%
% Unless required by applicable law or agreed to in writing, software
% distributed under the License is distributed on an "AS IS" BASIS,
% WITHOUT WARRANTIES OR CONDITIONS OF ANY KIND, either express or implied.
% See the License for the specific language governing permissions and
% limitations under the License.
%
% %CopyrightEnd%
%

\chapter{Parse trees}

\label{chapter:parse-trees}
\index{parse tree!standard representation|(}

In this appendix we define the standard representation of parse trees
for \Erlang\ programs as \Erlang\ terms.  A
\T{parse_transform/1}\index{parse_transform/1
function@\T{parse_transform/1} function}
(\S\ref{section:parse-transform}) takes a list of such terms as input
and is expected to return a list of such terms.  We define the
representation in a top-down way by first examining the forms that
make up a module declaration and then going through their lexical
constituents, etc.

The representation admits representation of many parse trees that
would be rejected by the grammar of \Erlang.  For example, it allows
representation of a module declaration in which forms are not in a
permitted order.

\index{Rep@\Rep|(}
We define the representation semi-formally through a function \Rep\
that maps \Erlang\ formulas or sequences of formulas to \Erlang\
terms.

\index{L@\LINE|(}
When we write \LINE, we mean an \Erlang\ integer that may be interpreted by
software tools as a line number when reporting errors, etc.
\index{L@\LINE|)}

\section{Module declarations and forms}

A module declaration consists of a sequence of forms that are either
function declarations or attributes
(\S\ref{section:module-declarations}).
\begin{itemize}
\item If \TZ{D} is a module declaration consisting of the forms
$\TZ{F}_1$, \ldots, $\TZ{F}_k$, then
\[\Rep(\TZ{D}) = \text{\T{[$\Rep(\Z{F}_1)$,\tdots,$\Rep(\Z{F}_k)$]}}.\]
\item If \TZ{F} is an attribute \T{-module(\Z{Mod})}, then
\[\Rep(\TZ{F}) = \text{\T{\{attribute,\LINE,module,\Z{Mod}\}}}.\]
\item If \TZ{F} is an attribute \T{-export([$\Z{Fun}_1$/$\Z{A}_1$,\tdots,$\Z{Fun}_k$/$\Z{A}_k$])}, then
\[\Rep(\TZ{F}) = \text{\T{\{attribute,\LINE,export,[\{$\Z{Fun}_1$,$\Z{A}_1$\},\tdots,\{$\Z{Fun}_k$,$\Z{A}_k$\}]\}}}.\]
\item If \TZ{F} is an attribute \T{-import(\Z{Mod},[$\Z{Fun}_1$/$\Z{A}_1$,\tdots,$\Z{Fun}_k$/$\Z{A}_k$])},
then
\begin{align*}
\Rep(\TZ{F}) =
\text{\T{\{attribute,\LINE,import,\{\Z{Mod},[}}&\text{\T{\{$\Z{Fun}_1$,$\Z{A}_1$\},\tdots,}} \\
                                               &\text{\T{\{$\Z{Fun}_k$,$\Z{A}_k$\}]\}\}}.}
\end{align*}
\item If \TZ{F} is an attribute \T{-compile([$\Z{T}_1$,\tdots,$\Z{T}_k$])}, then
\[\Rep(\TZ{F}) = \text{\T{\{attribute,\LINE,compile,[$\Z{T}_1$,\tdots,$\Z{T}_k$]\}}}.\]
\item If \TZ{F} is an attribute \T{-file(\Z{File},\Z{Line})}, then
\[\Rep(\TZ{F}) = \text{\T{\{attribute,\LINE,file,\{$\Rep(\Z{File})$,$\Rep(\Z{Line})$\}\}}}.\]
\item If \TZ{F} is a record declaration \T{-record(\Z{Name},\{$\Z{V}_1$,\tdots,$\Z{V}_k$\})}, then
\begin{align*}
\Rep(\TZ{F}) =
\text{\T{\{attribute,\LINE,record,\{\Z{Name},[}}&\text{\T{$\Rep(\Z{V}_1)$,\tdots,}} \\
                                                &\text{\T{$\Rep(\Z{V}_k)$]\}\}}.}
\end{align*}
\item If \TZ{F} is a wild attribute \T{-\Z{A}(\Z{T})}, then
\[\Rep(\TZ{F}) = \text{\T{\{attribute,\LINE,\Z{A},\Z{T}\}}}.\]
\item If \TZ{F} is a function declaration \T{\Z{Name}($\Z{Ps}_1$) $[\text{\T{when $\Z{Gs}_1$}}]$ -> $\Z{B}_1$ ; \tdots ; \Z{Name}($\Z{Ps}_k$) $[\text{\T{when $\Z{Gs}_k$}}]$ -> $\Z{B}_k$ end},
where $k\geq1$ and each $\Z{Ps}_i$, $\Z{Gs}_i$ and $\Z{B}_i$, $1\leq i\leq k$,
is a pattern sequence, a guard and a body, respectively (and $\Z{Gs}_i$ is \T{true} if
omitted), and each $\Z{Ps}_i$, $1\leq i\leq k$, has the same length \Z{Arity}, then
\begin{align*}
\Rep(\TZ{F}) =
\text{\T{\{}}&\text{\T{function,\LINE,\Z{Name},\Z{Arity},}} \\
             &\text{\T{[\{clause,\LINE,$\Rep(\Z{Ps}_1)$,$\Rep(\Z{Gs}_1)$,$\Rep(\Z{B}_1)$\},\tdots,}} \\
             &\text{\T{~\{clause,\LINE,$\Rep(\Z{Ps}_k)$,$\Rep(\Z{Gs}_k)$,$\Rep(\Z{B}_k)$\}]\}}.}
\end{align*}
\end{itemize}
In addition to the representations of forms, the list that represents
a module declaration may contain tuples \T{\{error,\Z{E}\}}, denoting
syntactically incorrect forms, and \T{\{eof,\LINE\}}, denoting an end
of stream encountered before a complete form had been parsed.

Each field declaration in a record declaration is with or without an
explicit default initializer expression
(\S\ref{section:record-declarations}).
\begin{itemize}
\item If \TZ{V} is \TZ{A}, then
\[\Rep(\TZ{V}) = \text{\T{\{record_field,\LINE,$\Rep(\Z{A})$\}}}.\]
\item If \TZ{V} is \T{\Z{A} = \Z{E}}, then
\[\Rep(\TZ{V}) = \text{\T{\{record_field,\LINE,$\Rep(\Z{A})$,$\Rep(\Z{E})$\}}}.\]
\end{itemize}

\section{Atomic literals}

\label{section:atomic-literal-rep}

There are five kinds of atomic literals, which are represented in the
same way in patterns, expressions and guard expressions:
\begin{itemize}
\item If \TZ{L} is an integer literal, then
\[\Rep(\TZ{L}) = \text{\T{\{integer,\LINE,\Z{L}\}}}.\]
\item If \TZ{L} is a float literal, then
\[\Rep(\TZ{L}) = \text{\T{\{float,\LINE,\Z{L}\}}}.\]
\item If \TZ{L} is a character literal, then
\ifStd
\[\Rep(\TZ{L}) = \text{\T{\{char,\LINE,\Z{L}\}}}.\]
\else
\[\Rep(\TZ{L}) = \text{\T{\{integer,\LINE,\Z{L}\}}}.\]
\fi
\item If \TZ{L} is a string literal consisting of the characters
$\TZ{C}_1$, \ldots, $\TZ{C}_k$, then
\[\Rep(\TZ{L}) = \text{\T{\{string,\LINE,[$\Z{C}_1$,\tdots,$\Z{C}_k$]\}}}.\]
\item If \TZ{L} is an atom literal, then
\[\Rep(\TZ{L}) = \text{\T{\{atom,\LINE,\Z{L}\}}}.\]
\end{itemize}

\section{Patterns}

For a sequence \TZ{Ps} of patterns $\TZ{P}_1$, \ldots, $\TZ{P}_k$,
\[\Rep(\TZ{Ps}) = \text{\T{[$\Rep(\Z{P}_1)$,\tdots,$\Rep(\Z{P}_k)$]}}.\]

\noindent Individual patterns are represented as follows:
\begin{itemize}
\item If \TZ{P} is a compound pattern \T{$\Z{P}_1$ = $\Z{P}_2$}, then
\[\Rep(\TZ{P}) = \text{\T{\{match,\LINE,$\Rep(\Z{P}_1)$,$\Rep(\Z{P}_2)$\}}}.\]
\item If \TZ{P} is an atomic literal \TZ{L}, then
\[\Rep(\TZ{P}) = \Rep(\TZ{L}),\]
cf.\ \S\ref{section:atomic-literal-rep}.
\item If \TZ{P} is a variable pattern \TZ{V}, then
\[\Rep(\TZ{P}) = \text{\T{\{var,\LINE,\TZ{A}\}}},\]
where \TZ{A} is an atom with a printname that constitutes the same characters as \TZ{V}.
\item If \TZ{P} is a universal pattern \T{_}, then
\[\Rep(\TZ{P}) = \text{\T{\{var,\LINE,'_'\}}}.\]
\item If \TZ{P} is a tuple pattern \T{\{$\Z{P}_1$,\tdots,$\Z{P}_k$\}}, then
\[\Rep(\TZ{P}) = \text{\T{\{tuple,\LINE,[$\Rep(\Z{P}_1)$,\tdots,$\Rep(\Z{P}_k)$]\}}}.\]
\item If \TZ{P} is a nil pattern \T{[]}, then
\[\Rep(\TZ{P}) = \text{\T{\{nil,\LINE\}}}.\]
\item If \TZ{P} is a cons pattern \T{[$\Z{P}_h$|$\Z{P}_t$]}, then
\[\Rep(\TZ{P}) = \text{\T{\{cons,\LINE,$\Rep(\Z{P}_h)$,$\Rep(\Z{P}_t)$\}}}.\]
\item If \TZ{P} is a record pattern \T{\char`\#\TZ{Name}\{$\Z{Field}_1$=$\Z{P}_1$,\tdots,$\Z{Field}_k$=$\Z{P}_k$\}}, then
\begin{align*}
\Rep(\TZ{P}) =
\text{\T{\{}}&\text{\T{record,\LINE,\Z{Name},}} \\
             &\text{\T{[\{record_field,\LINE,$\Rep(\Z{Field}_1)$,$\Rep(\Z{P}_1)$\},\tdots,}} \\
             &\text{\T{~\{record_field,\LINE,$\Rep(\Z{Field}_k)$,$\Rep(\Z{P}_k)$\}]\}}.}
\end{align*}
\end{itemize}

\section{Expressions}

\label{section:expressions-rep}

A body \TZ{B} is a sequence of expressions \T{$\Z{E}_1$, \ldots, $\Z{E}_k$},
where $k\geq 1$, and
\[\Rep(\TZ{B}) = \text{\T{[$\Rep(\Z{E}_1)$,\tdots,$\Rep(\Z{E}_k)$]}}.\]

%For a sequence \TZ{Es} of expressions $\TZ{E}_1$, \ldots, $\TZ{E}_k$,
%\[\Rep(\TZ{Es}) = \text{\T{[$\Rep(\Z{E}_1)$,\tdots,$\Rep(\Z{E}_k)$]}}.\]

\noindent An expression \TZ{E} is one of the following alternatives:
\begin{itemize}
\item If \TZ{E} is \T{catch $\Z{E}_0$}, then
\[\Rep(\TZ{E}) = \text{\T{\{'catch',\LINE,$\Rep(\Z{E}_0)$\}}}.\]
\item If \TZ{E} is \T{\Z{P} = $\Z{E}_0$}, then
\[\Rep(\TZ{E}) = \text{\T{\{match,\LINE,$\Rep(\Z{P})$,$\Rep(\Z{E}_0)$\}}}.\]
%\item If \TZ{E} is \T{$\Z{E}_1$ !\ $\Z{E}_2$}, then
%\[\Rep(\TZ{E}) = \text{\T{\{op,\LINE,'!',$\Rep(\Z{E}_1)$,$\Rep(\Z{E}_2)$\}}}.\]
\item If \TZ{E} is \T{$\Z{E}_1$ \Z{Op} $\Z{E}_2$}, where \TZ{Op} is \T{!}, \T{or}, \T{and},
a \NT{RelationalOp}, an \NT{EqualityOp}, a \NT{ListConcOp}, an \NT{AdditionOp},
a \NT{ShiftOp} or a \NT{MultiplicationOp}, then
\[\Rep(\TZ{E}) = \text{\T{\{op,\LINE,\Z{Op},$\Rep(\Z{E}_1)$,$\Rep(\Z{E}_2)$\}}}.\]
\item If \TZ{E} is \T{\Z{Op} $\Z{E}_0$}, where \TZ{Op} is a \NT{PrefixOp}, then
\[\Rep(\TZ{E}) = \text{\T{\{op,\LINE,\Z{Op},$\Rep(\Z{E}_0)$\}}}.\]
\item If \TZ{E} is \T{\char`\#\Z{Name}.\Z{Field}}, then
\[\Rep(\TZ{E}) = \text{\T{\{record_index,\LINE,\Z{Name},$\Rep(\Z{Field})$\}}}.\]
\item If \TZ{E} is \T{$\Z{E}_0$\char`\#\Z{Name}.\Z{Field}}, then
\[\Rep(\TZ{E}) = \text{\T{\{record_field,\LINE,$\Rep(\Z{E}_0)$,\Z{Name},$\Rep(\Z{Field})$\}}}.\]
\item If \TZ{E} is \T{\char`\#\Z{Name}\{$\Z{Field}_1$=$\Z{E}_1$,\tdots,$\Z{Field}_k$=$\Z{E}_k$\}}, then
\begin{align*}
\Rep(\TZ{E}) =
\text{\T{\{}}&\text{\T{record,\LINE,\Z{Name},}} \\
             &\text{\T{[\{record_field,\LINE,$\Rep(\Z{Field}_1)$,$\Rep(\Z{E}_1)$\},\tdots,}} \\
             &\text{\T{~\{record_field,\LINE,$\Rep(\Z{Field}_k)$,$\Rep(\Z{E}_k)$\}]\}}.}
\end{align*}
\item If \TZ{E} is \T{$\Z{E}_0$\char`\#\Z{Name}\{$\Z{Field}_1$=$\Z{E}_1$,\tdots,$\Z{Field}_k$=$\Z{E}_k$\}}, then
\begin{align*}
\Rep(\TZ{E}) =
\text{\T{\{}}&\text{\T{record,\LINE,$\Rep(\Z{E}_0)$,\Z{Name},}} \\
             &\text{\T{[\{record_field,\LINE,$\Rep(\Z{Field}_1)$,$\Rep(\Z{E}_1)$\},\tdots,}} \\
             &\text{\T{~\{record_field,\LINE,$\Rep(\Z{Field}_k)$,$\Rep(\Z{E}_k)$\}]\}}.}
\end{align*}
\item If \TZ{E} is \T{$\Z{E}_0$($\Z{E}_1$,\tdots,$\Z{E}_k$)} (the case where
$\Z{E}_0$ is an atom is not distinguished), then
\[\Rep(\TZ{E}) = \text{\T{\{call,\LINE,$\Rep(\Z{E}_0)$,[$\Rep(\Z{E}_1)$,\tdots,$\Rep(\Z{E}_k)$]\}}}.\]
\item If \TZ{E} is \T{$\TZ{E}_m$:$\Z{E}_0$($\Z{E}_1$,\tdots,$\Z{E}_k$)}, then
\begin{align*}
\Rep(\TZ{E}) =
\text{\T{\{call,\LINE,\{remote,\LINE,$\Rep(\Z{E}_m)$,$\Rep(\Z{E}_0)$\},[}}&\text{\T{$\Rep(\Z{E}_1)$,\tdots,}} \\
                                                                          &\text{\T{$\Rep(\Z{E}_k)$]\}}.}
\end{align*}
\item If \TZ{E} is a variable \TZ{V}, then
\[\Rep(\TZ{E}) = \text{\T{\{var,\LINE,\TZ{A}\}}},\]
where \TZ{A} is an atom with a printname that constitutes the same characters as \TZ{V}.
\item If \TZ{P} is an atomic literal \TZ{L}, then
\[\Rep(\TZ{P}) = \Rep(\TZ{L}),\]
cf.\ \S\ref{section:atomic-literal-rep}.
\item If \TZ{E} is a tuple skeleton \T{\{$\Z{E}_1$,\tdots,$\Z{E}_k$\}}, then
\[\Rep(\TZ{E}) = \text{\T{\{tuple,\LINE,[$\Rep(\Z{E}_1)$,\tdots,$\Rep(\Z{E}_k)$]\}}}.\]
\item If \TZ{E} is \T{[]}, then
\[\Rep(\TZ{E}) = \text{\T{\{nil,\LINE\}}}.\]
\item If \TZ{E} is a cons skeleton \T{[$\Z{E}_h$|$\Z{E}_t$]}, then
\[\Rep(\TZ{E}) = \text{\T{\{cons,\LINE,$\Rep(\Z{E}_h)$,$\Rep(\Z{E}_t)$\}}}.\]
\item If \TZ{E} is a list comprehension \T{[$\Z{E}_0$ || $\Z{W}_1$, \tdots, $\Z{W}_k$]},
where each $\TZ{W}_i$, $1\leq i\leq k$, is a generator or a filter, then
\[\Rep(\TZ{E}) = \text{\T{\{lc,\LINE,$\Rep(\Z{E}_0)$,[$\Rep(\Z{W}_1)$,\tdots,$\Rep(\Z{W}_k)$]\}}}.\]
\item If \TZ{E} is \T{begin \TZ{B} end}, where \TZ{B} is a body, then
\[\Rep(\TZ{E}) = \text{\T{\{block,\LINE,$\Rep(\Z{B})$\}}}.\]
\ifStd
\item If \TZ{E} is \T{cond $\Z{E}_1$ -> $\Z{B}_1$ ; \tdots ; $\Z{E}_k$ -> $\Z{B}_k$ end},
where $k\geq1$ and each $\Z{E}_i$ and $\Z{B}_i$, $1\leq i\leq k$, is an expression
and a body, respectively, then
\begin{align*}
\Rep(\TZ{E}) =
\text{\T{\{'cond',\LINE,[}}&\text{\T{\{c_clause,\LINE,$\Rep(\Z{E}_1)$,$\Rep(\Z{B}_1)$\},\tdots,}} \\
                           &\text{\T{\{c_clause,\LINE,$\Rep(\Z{E}_k)$,$\Rep(\Z{B}_k)$\}]\}}.}
\end{align*}
\fi
\item If \TZ{E} is \T{if $\Z{Gs}_1$ -> $\Z{B}_1$ ; \tdots ; $\Z{Gs}_k$ -> $\Z{B}_k$ end},
where $k\geq1$ and each $\Z{Gs}_i$ and $\Z{B}_i$, $1\leq i\leq k$, is a guard and a body,
respectively, then
\begin{align*}
\Rep(\TZ{E}) =
\text{\T{\{'if',\LINE,[}}&\text{\T{\{clause,\LINE,[],$\Rep(\Z{Gs}_1)$,$\Rep(\Z{B}_1)$\},\tdots,}} \\
                         &\text{\T{\{clause,\LINE,[],$\Rep(\Z{Gs}_k)$,$\Rep(\Z{B}_k)$\}]\}}.}
\end{align*}
\item If \TZ{E} is \T{case $\Z{E}_0$ of $\Z{P}_1$ $[\text{\T{when $\Z{Gs}_1$}}]$ -> $\Z{B}_1$ ; \tdots ; $\Z{P}_k$ $[\text{\T{when $\Z{Gs}_k$}}]$ -> \\ $\Z{B}_k$ end},
where $\TZ{E}'$ is an expression, $k\geq1$ and each $\Z{P}_i$, $\Z{Gs}_i$ and $\Z{B}_i$, $1\leq i\leq k$,
is a pattern, a guard and a body, respectively (and $\Z{Gs}_i$ is \T{true} if
omitted), then
\begin{align*}
\Rep(\TZ{E}) =
\text{\T{\{}}&\text{\T{'case',\LINE,$\Rep(\Z{E}_0)$,}} \\
             &\text{\T{[\{clause,\LINE,[$\Rep(\Z{P}_1)$],$\Rep(\Z{Gs}_1)$,$\Rep(\Z{B}_1)$\},\tdots,}} \\
             &\text{\T{~\{clause,\LINE,[$\Rep(\Z{P}_k)$],$\Rep(\Z{Gs}_k)$,$\Rep(\Z{B}_k)$\}]\}}.}
\end{align*}
\item If \TZ{E} is \T{receive $\Z{P}_1$ $[\text{\T{when $\Z{Gs}_1$}}]$ -> $\Z{B}_1$ ; \tdots ; $\Z{P}_k$ $[\text{\T{when $\Z{Gs}_k$}}]$ -> $\Z{B}_k$ end},
where $k\geq1$ and each $\Z{P}_i$, $\Z{Gs}_i$ and $\Z{B}_i$, $1\leq i\leq k$,
is a pattern, a guard and a body, respectively (and $\Z{Gs}_i$ is \T{true} if
omitted), then
\begin{align*}
\Rep(\TZ{E}) =
\text{\T{\{}}&\text{\T{'receive',\LINE,}} \\
             &\text{\T{[\{clause,\LINE,[$\Rep(\Z{P}_1)$],$\Rep(\Z{Gs}_1)$,$\Rep(\Z{B}_1)$\},\tdots,}} \\
             &\text{\T{~\{clause,\LINE,[$\Rep(\Z{P}_k)$],$\Rep(\Z{Gs}_k)$,$\Rep(\Z{B}_k)$\}]\}}.}
\end{align*}
\item If \TZ{E} is \T{receive $\Z{P}_1$ $[\text{\T{when $\Z{Gs}_1$}}]$ -> $\Z{B}_1$ ; \tdots ; $\Z{P}_k$ $[\text{\T{when $\Z{Gs}_k$}}]$ -> $\Z{B}_k$ after $\TZ{E}'$ -> $\Z{B}_{k+1}$ end},
where $k\geq1$, each $\Z{P}_i$, $\Z{Gs}_i$ and $\Z{B}_i$, $1\leq i\leq k$,
is a pattern, a guard and a body, respectively (and $\Z{Gs}_i$ is \T{true} if
omitted), $\TZ{E}'$ is an expression and $\TZ{B}_{k+1}$ is a body, then
\begin{align*}
\Rep(\TZ{E}) =
\text{\T{\{}}&\text{\T{'receive',\LINE,}} \\
             &\text{\T{[\{clause,\LINE,[$\Rep(\Z{P}_1)$],$\Rep(\Z{Gs}_1)$,$\Rep(\Z{B}_1)$\},\tdots,}} \\
             &\text{\T{~\{clause,\LINE,[$\Rep(\Z{P}_k)$],$\Rep(\Z{Gs}_k)$,$\Rep(\Z{B}_k)$\}],}} \\
             &\text{\T{$\Rep(\Z{E}_0)$,$\Rep(\Z{B}_{k+1})$\}}.}
\end{align*}
\ifStd
\item If \TZ{E} is \T{try $\Z{B}$ catch $\Z{P}_1$ $[\text{\T{when $\Z{Gs}_1$}}]$ -> $\Z{B}_1$ ; \tdots ; $\Z{P}_k$ $[\text{\T{when $\Z{Gs}_k$}}]$ -> $\Z{B}_k$ end},
where \TZ{B} is a body, $k\geq1$ and each $\Z{P}_i$, $\Z{Gs}_i$ and $\Z{B}_i$, $1\leq i\leq k$,
is a pattern, a guard and a body, respectively (and $\Z{Gs}_i$ is \T{true} if
omitted), then
\begin{align*}
\Rep(\TZ{E}) =
\text{\T{\{}}&\text{\T{'try',\LINE,$\Rep(\Z{B})$,}} \\
             &\text{\T{[\{clause,\LINE,[$\Rep(\Z{P}_1)$],$\Rep(\Z{Gs}_1)$,$\Rep(\Z{B}_1)$\},\tdots,}} \\
             &\text{\T{~\{clause,\LINE,[$\Rep(\Z{P}_k)$],$\Rep(\Z{Gs}_k)$,$\Rep(\Z{B}_k)$\}]\}}.}
\end{align*}
\item If \TZ{E} is \T{try $\Z{B}$ end},
where \TZ{B} is a body, then
\[\Rep(\TZ{E}) = \text{\T{\{'try',\LINE,$\Rep(\Z{B})$,[]\}}}.\]
\fi
\item If \TZ{E} is \T{fun \Z{Name}/\Z{Arity}}, then
\[\Rep(\TZ{E}) = \text{\T{\{'fun',\LINE,\{function,\Z{Name},\Z{Arity}\}\}}}.\]
\item If \TZ{E} is \T{fun $\Z{P}_1$ $[\text{\T{when $\Z{Gs}_1$}}]$ -> $\Z{B}_1$ ; \tdots ; $\Z{P}_k$ $[\text{\T{when $\Z{Gs}_k$}}]$ -> $\Z{B}_k$ end},
where $k\geq1$ and each $\Z{P}_i$, $\Z{Gs}_i$ and $\Z{B}_i$, $1\leq i\leq k$,
is a pattern, a guard and a body, respectively (and $\Z{Gs}_i$ is \T{true} if
omitted), then
\begin{align*}
\Rep(\TZ{E}) =
\text{\T{\{'fun',\LINE,\{}}&\text{\T{clauses,}} \\
                           &\text{\T{[\{clause,\LINE,[$\Rep(\Z{P}_1)$],$\Rep(\Z{Gs}_1)$,$\Rep(\Z{B}_1)$\}]\},}} \\
                           &\text{\T{\tdots,}} \\
                           &\text{\T{~\{clause,\LINE,[$\Rep(\Z{P}_k)$],$\Rep(\Z{Gs}_k)$,$\Rep(\Z{B}_k)$\}]\}\}}.}
\end{align*}
\ifOld
\item If \TZ{E} is \T{query [$\Z{E}_0$ || $\Z{W}_1$,\tdots,$\Z{W}_k$] end},
where each $\TZ{W}_i$, $1\leq i\leq k$, is a generator or a filter, then
\[\Rep(\TZ{E}) = \text{\T{\{'query',\LINE,\{lc,\LINE,$\Rep(\Z{E}_0)$,[$\Rep(\Z{W}_1)$,\tdots,$\Rep(\Z{W}_k)$]\}\}}}.\]
\item If \TZ{E} is \T{$\Z{E}_0$.\Z{Field}}, a Mnesia record access
inside a \T{query}, then
\[\Rep(\TZ{E}) = \text{\T{\{record_field,\LINE,$\Rep(\Z{E}_0)$,$\Rep(\Z{Field})$\}}}.\]
\fi
\item If \TZ{E} is \T{( $\TZ{E}'$ )}, then
\[\Rep(\TZ{E}) = \Rep(\TZ{E}'),\]
i.e., parenthesized expressions cannot be distinguished from their bodies.
\end{itemize}
When \TZ{W} is a generator or a filter (in the body of a list comprehension), then:
\begin{itemize}
\item If \TZ{W} is a generator \T{\Z{P} <- \Z{E}}, where \TZ{P} is a pattern and \TZ{E}
is an expression, then
\[\Rep(\TZ{W}) = \text{\T{\{generate,\LINE,$\Rep(\Z{P})$,$\Rep(\Z{E})$\}}}.\]
\item If \TZ{W} is a filter \TZ{E}, which is an expression, then
\[\Rep(\TZ{W}) = \Rep(\TZ{E}).\]
\end{itemize}

\section{Guards}

A guard \TZ{Gs} is a nonempty sequence of guard tests $\TZ{G}_1$, \ldots, $\TZ{G}_k$, and
\[\Rep(\TZ{Gs}) = \text{\T{[$\Rep(\Z{G}_1)$,\tdots,$\Rep(\Z{G}_k)$]}}.\]

A guard test \TZ{G} is either \T{true}, an application of a BIF to a sequence of guard
expressions (syntactically this includes guard record tests), or a binary operator
applied to two guard expressions.
\begin{itemize}
\item If \TZ{G} is \T{true}, then
\[\Rep(\TZ{G}) = \text{\T{\{atom,\LINE,true\}}}.\]
\item If \TZ{G} is an application \T{\Z{A}($\Z{E}_1$,\tdots,$\Z{E}_k$)}, where \TZ{A}
is an atom and $\TZ{E}_1$, \ldots, $\TZ{E}_k$ are guard expressions, then
\[\Rep(\TZ{G}) = \text{\T{\{call,\LINE,\{atom,\LINE,\Z{A}\},[$\Rep(\Z{E}_1)$,\tdots,$\Rep(\Z{E}_k)$]\}}}.\]
\item If \TZ{G} is an operator expression \T{$\Z{E}_1$ \Z{Op} $\Z{E}_2$}, where \TZ{Op}
is a \NT{RelationalOp} or an \NT{EqualityOp}, and $\TZ{E}_1$, $\TZ{E}_2$ are guard
expressions, then
\[\Rep(\TZ{G}) = \text{\T{\{op,\LINE,\Z{Op},$\Rep(\Z{E}_1)$,$\Rep(\Z{E}_2)$\}}}.\]
\end{itemize}
All guard expressions are expressions and are represented in the same way as
the corresponding expressions, cf.\ \S\ref{section:expressions-rep}.
\index{Rep@\Rep|)}
\index{parse tree!standard representation|)}
