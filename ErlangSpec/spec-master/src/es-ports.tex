%
% %CopyrightBegin%
%
% Copyright Ericsson AB 2017. All Rights Reserved.
%
% Licensed under the Apache License, Version 2.0 (the "License");
% you may not use this file except in compliance with the License.
% You may obtain a copy of the License at
%
%     http://www.apache.org/licenses/LICENSE-2.0
%
% Unless required by applicable law or agreed to in writing, software
% distributed under the License is distributed on an "AS IS" BASIS,
% WITHOUT WARRANTIES OR CONDITIONS OF ANY KIND, either express or implied.
% See the License for the specific language governing permissions and
% limitations under the License.
%
% %CopyrightEnd%
%

\chapter{Ports}

\label{chapter:more-about-ports}

\section{Overview of ports}

\index{port!communication|(}

An \Erlang\ node and thus an \Erlang\ process on it communicates with
resources in the outside world (including the rest of the computer on
which it resides) through \emph{ports}.  Examples of such external
resources are files, drivers (\S\ref{section:drivers}) and non-\Erlang\
processes running on the same
host.  Information sent to a port from outside the \Erlang\ node
can be read by an \Erlang\ process as messages and
messages sent by an \Erlang\ process to a port are made available to
the external program (or written to a file).

\index{port!direction|(}
A port can be unidirectional or bidirectional, as requested when the port
is opened.  If the port is for communication
with an external process, it would typically be bidirectional.
\ifOld
If the port is connected to a file that is being read, it would typically
be input-only.  If the port is connected to a file that is to be written,
it would typically be output-only.
\fi
\index{port!direction|)}

\index{port!ownership|(}
A port is always \emph{owned} by some process that resides on the
same node as the port.
That process will receive input in the form of messages from the
port when information is sent to the port from outside \Erlang.
The port is initially owned by the process that opened the port but ownership
of a port can be transferred to another process on the same node.
\index{port!ownership|)}

An external resource has much in common with a process and thus a
port has much in common with a PID:
\begin{itemize}
\item Communicating with an external resource is similar to communicating
with a process: messages to a port are sent using the \T{!}\ operator and
messages from a port are received using \T{receive} expressions.

\item \index{port!linking|(}
A process can link to a port in order to be notified when an
external resource completes.
\index{port!linking|)}
\end{itemize}
However, ports are restricted as compared with processes.
For example, information written to a port from outside \Erlang\ is always sent
as a message to the process that owns the port.
An external resource thus cannot communicate directly
with an arbitrary \Erlang\ process.\footnote{A typical way to establish
communication between an external resource and arbitrary processes is
to let the process that is connected to the port be a server acting
as a proxy.}  Also, it is not possible to
register a name for a port.

Each port is identified by an \Erlang\ term that is itself referred to
as a port (although calling it a \emph{port identifier} would be more
in harmony with process vs.\ process identifier).

When a port is created, it is connected with an external entity, which
is
\ifOld
either
\begin{itemize}
\item a recently spawned external process or recently opened driver
(\S\ref{section:drivers});
\item an external resource, such as a file.
\end{itemize}
\fi
\ifStd
a recently spawned external process or recently opened driver
(\S\ref{section:drivers}).
\fi

As in the case of PIDs, there can be ports that were created for communication
with an external process or resource that no longer exists.
What happens when a BIF is given such a port varies, cf.\
\ifOld\S\ref{section:port-bifs}\fi
\ifStd\S\ref{section:port-module}\fi.

The interface to a port from the outside of Erlang is dependent on the operating
system where the node resides.  A driver or external process that has been
started by the \Erlang\ system when a port was opened can read from a port and write
to it using a pair of procedures that we refer to below as \I{read} and \I{write}.

\Erlang\ BIFs relating to ports are described in
\ifOld\S\ref{section:port-bifs}\fi
\ifStd\S\ref{section:port-module}\fi.
\index{port!communication|)}

\section{Drivers}

\label{section:drivers}
\index{driver|(}

An \Erlang\ system may provide a way to extend a node with software written in other
programming languages than \Erlang.  Such software is called a \emph{driver} and would
normally communicate with \Erlang\ processes through a port.

The interface to external software is thus the same regardless of whether it runs as
part of the node or as external processes.
\index{driver|)}

\section{I/O terms}

\label{section:io-term}

\index{I/O term!definition of|(}
Define an I/O term to be either
\begin{itemize}
\item a binary,
\item nil, or
\item a cons where the head is a byte\ifStd, a character\fi\ or an I/O term
and the tail is an I/O term.
\end{itemize}
\index{I/O term!definition of|)}

\index{I/O term!contents of|(}
The \emph{contents} of an I/O term is a sequence of bytes defined
recursively as follows:
\begin{itemize}
\item The contents of a binary is the sequence of bytes of the binary,
in the order they appear in the binary.
\item The contents of an empty list is an empty sequence.
\item The contents of a cons of a byte and an I/O term is
the byte followed by the contents of the tail.
\ifStd
\item The contents of a cons of a character and an I/O term is
the code of the character modulo 256 followed by the contents of the tail.
\fi
\item The contents of a cons of two I/O terms is
the contents of the head followed by the contents of the tail.
\end{itemize}
\index{I/O term!contents of|)}

To clarify, the function \T{contents/1} defined below returns the contents
of an I/O term, represented as a list of bytes.\footnote{It is designed for
clarity, not efficiency.}

\ifStd
\begin{verbatim}
contents(B) when is_binary(B) -> binary:to_list(B) ;
contents([]) -> [] ;
contents([I|Y]) when is_integer(I) -> [I | contents(Y)] ;
contents([C|Y]) when is_char(C) ->
    [char:to_integer(C) mod 256 | contents(Y)] ;
contents([X|Y]) -> contents(X) ++ contents(Y).
\end{verbatim}
\fi
\ifOld
\begin{verbatim}
contents(B) when binary(B) -> binary_to_list(B) ;
contents([]) -> [] ;
contents([I|Y]) when integer(I) -> [I | contents(Y)] ;
contents([X|Y]) -> contents(X) ++ contents(Y).
\end{verbatim}
\fi

\section{Opening ports}

\label{section:opening-ports}
\index{port!opening|(}

A port is opened by calling the BIF
\T{open_port/2}\index{open_port/2 BIF@\T{open_port/2} BIF}
(\S\ref{section:openport2}).
In an application \T{open_port(\Z{Resource},\Z{Options})}, the \TZ{Resource}
argument identifies the resource to which a port is being opened and the
\TZ{Options} argument determines the details of the behaviour of the port.

\TZ{Resource} is one of
\begin{itemize}
\item a 2-tuple \T{\{spawn,\Z{Command}\}}, where \TZ{Command} is a string
or an atom (in which case its printname will be used).  A driver is opened
inside the \Erlang\ node or an external
process is spawned and a new port is connected with the driver or the
standard input and output of the external process.
\ifOld
\TZ{Command} must not contain a null character (\T{\char`\\\ifStd u0000\fi\ifOld 000 \fi}).
Suppose that the result of extracting whitespace-separated words from
\TZ{Command} is $\TZ{Wd}_1$, $\TZ{Wd}_2$, \ldots, $\TZ{Wd}_k$.
\begin{itemize}
\item If $\TZ{Wd}_1$ is the name of an installed driver, then such a driver
will be started with the string $\TZ{Wd}_2$, \ldots, $\TZ{Wd}_k$ as
input.
\item Otherwise $\TZ{Wd}_1$ will be interpreted as the name of an
external program which is started with $\TZ{Wd}_2$, \ldots, $\TZ{Wd}_k$ as its arguments.
\end{itemize}
\fi
\ifStd
The exact interpretation of \TZ{Command} is implementation-defined.
\fi
\ifOld
\item a string or atom (in which case its printname will be used) \TZ{Resource}.
If \TZ{Resource} can be interpreted
by \OldErlang\ as identifying a resource (for example, naming a file),
a new port is connected with the resource.  If the resource is a file, then
either the option \T{in} (for reading from the file) or the option \T{out}
(for writing to the file) must be given.
\fi
\item a 3-tuple \T{\{fd, \Z{In}, \Z{Out}\}}, where \TZ{In} and \TZ{Out} are
integers.  On a node running on a POSIX-compliant operating system, \TZ{In}
and \TZ{Out} are interpreted as file descriptors and a new port is connected
writing to \TZ{In} and reading from \TZ{Out}.  On other nodes, the behaviour
is implementation-defined.
\end{itemize}
\ifStd
A \StdErlang\ implementation may accept further alternatives for
\TZ{Resource}.  It should be documented for an implementation which
alternatives for \TZ{Resource} are accepted and how they are
interpreted.
\fi
If \TZ{Resource} is not one of the permitted alternatives, \T{open_port/2}
should exit with \T{badarg}.

\TZ{Options} is a list of items, each of which is a term from one of the
following groups.  At most one term from each group should be present.
If no term from a group is in \TZ{Options}, it is as if the default
term for the group was present.

\begin{Lentry}
\item[Stream/packets]
\index{port!stream mode|(}
\index{port!packet mode|(}
\begin{itemize}
\item \T{stream} (the default). Messages are sent without packet headers.
\item \T{\{packet,\Z{N}\}}, where \TZ{N} is one of the integers \T{1}, \T{2}
and \T{4}.  Messages are sent in packets with the packet headers in \TZ{N} bytes
(\S\ref{section:packets}).
\end{itemize}
\index{port!stream mode|)}
\index{port!packet mode|)}
\item[Process inputs and outputs]
(Only relevant on POSIX-compliant systems and for \T{\{spawn,\TZ{Command}\}}.)
\begin{itemize}
\item \T{use_stdio}.  (The default.)
Use the standard input and standard output (i.e., file descriptors 0 and 1 on UNIX
systems) of the
spawned process for communicating with \Erlang.
\item \T{nouse_stdio}.
Use alternative input and output channels of the spawned process (file descriptors
3 and 4 on UNIX systems) for communicating with \Erlang.
\end{itemize}
\item[Direction]
\index{port!direction|(}
\begin{itemize}
\item \T{in}. The port can only be used for input from the external resource.
\item \T{out}. The port can only be used for output to the external resource.
\item \ifOld \I{None}. \fi \ifStd \T{bi}. (The default.) \fi
The port can be used for input from and output to the external resource.
\end{itemize}
\index{port!direction|)}
\item[Received data]
\index{port!binary mode|(}
\index{port!list mode|(}
\begin{itemize}
\item \T{binary}.  Input from the external resource is received as binaries.
\item \ifOld \I{None}. \fi \ifStd \T{list}. (The default.) \fi
Input from the external resource is received as lists of bytes.
\end{itemize}
\index{port!binary mode|)}
\index{port!list mode|)}
\item[End of stream]
\index{port!closing|(}
\begin{itemize}
\item \T{eof}.  The port is not closed when the external resource is depleted,
instead a message \T{\{\Z{Port},eof\}} is sent to the owning process.
\item \T{noeof}.  (The default.) The port is closed when the external resource is depleted.
All processes linked to the port will then receive exit signals.
\end{itemize}
\index{port!closing|)}
\end{Lentry}
\ifStd
A \StdErlang\ implementation may accept further alternatives for
elements of \TZ{Options} and it should be documented which they are
and how they are interpreted.
\fi
If \TZ{Options} is not a list or if \TZ{Options} contains unrecognized terms,
\T{open_port/2} should exit with \T{badarg}.

\index{port!linking|(}
When a port is opened, it is automatically linked to the owning process
(\S\ref{section:port-linking}).
\index{port!linking|)}
\index{port!opening|)}

\section{Controlling a port from an \Erlang\ process}

\label{section:port-control}
\index{port!control|(}

By controlling a port \TZ{R} to an external resource we mean to
write data to the port or changing properties of the port itself.
Such control is exercised by
sending a message with \TZ{R} as destination
(\S\ref{section:send-expr}).  As before, evaluation
of such a send expression always succeds.  However, the \Erlang\ system
subsequently attempts to interpret the message as port control information and if
that interpretation fails, an exit signal\index{exit!signal} (\S\ref{section:exit-signals})
will be sent to the process owning \TZ{R}, i.e.,
\T{owner[\Z{R}]}\index{owner port property@\T{owner} port property}
(\S\ref{section:port-state-dynamic}).

Any process may control a port.  However, the process that sends
a message to \TZ{R}
must have the PID of the process that owns \TZ{R} (because \T{owner[\Z{R}]} must
be part of the message, as described below).

\index{badsig exit signal@\T{badsig} exit signal|(}
Consider a message \TZ{v} sent to a port \TZ{R}.
\begin{itemize}
\item If the port has been closed, then nothing happens.
\item If \TZ{v} is a tuple \TZ{\{\T{owner[\Z{R}]}, \{command, \Z{Data}\}\}}, then:
\begin{itemize}
\item If \T{in[\Z{R}]} is \T{false}, then
an exit signal \T{badsig} is sent to \T{owner[\Z{R}]}.
\item If \TZ{Data} is an I/O term (\S\ref{section:io-term}), then the contents
of \TZ{Data} will be transmitted through \TZ{R}, as described in
\S\ref{section:send-port} (where it is also described how an exit
signal \T{badsig} is sent to \T{owner[\Z{R}]} if the number of bytes
in the contents of \TZ{Data} is too large to fit in the packet header)
\item Otherwise, an exit signal \T{badsig} is sent to \T{owner[\Z{R}]}.
\end{itemize}
\item If \TZ{v} is a tuple \TZ{\{\T{owner[\Z{R}]}, close\}}, then
the port \TZ{R} is closed and a message \T{\{\Z{R}, closed\}} is sent
to \T{owner[\Z{R}]}.
\item If \TZ{v} is a tuple \TZ{\{\T{owner[\Z{R}]}, \{connect, $\Z{P}_2$\}\}}, then
what happens depends on $\TZ{P}_2$:
\begin{itemize}
\item If $\TZ{P}_2$ is the PID of a process residing on the same node as \TZ{R}
(and thus \T{owner[\Z{R}]}, then a message \T{\{\Z{R}, connected\}} is sent
to \T{owner[\Z{R}]} and the value of \T{owner[\Z{R}]} is changed to $\Z{P}_2$.
\item Otherwise, an exit signal \T{badsig} is sent to \T{owner[\Z{R}]}.
\end{itemize}
\item Otherwise, an exit signal \T{badsig} is sent \T{owner[\Z{R}]}.
\end{itemize}
It should be noted that
all processing when writing to a port happens asynchronously, also error processing.
\index{badsig exit signal@\T{badsig} exit signal|)}
\index{port!control|)}

\section{Transmitting data from an \Erlang\ process to the outside}

\label{section:send-port}
\label{section:packets}
\index{port!sending to a|(}

Suppose that an \Erlang\ process has sent an I/O term \TZ{T} to a port \TZ{R} and that
the contents of the I/O term is a sequence of bytes $\TZ{b}_1$, \ldots, $\TZ{b}_k$.

Depending on the value of
\T{packeting[\Z{R}]}\index{packeting port property@\T{packeting} port property},
the byte sequence transmitted to the
external resource may be prepended with a packet header by the \Erlang\ system:
\begin{itemize}
\item \index{port!stream mode|(}
If \T{packeting[\Z{R}]} is \T{stream}, then the byte sequence will be transmitted
as is.
\index{port!stream mode|)}
\item \index{port!packet mode|(}
If \T{packeting[\Z{R}]} is \T{\{packet,\Z{N}\}}, the transmitted byte sequence
will be $\TZ{h}_1$, \ldots, $\TZ{h}_{\TZm{N}}$, $\TZ{b}_1$, \ldots, $\TZ{b}_k$.
The first \TZ{N} bytes constitute a \emph{packet header}, where for each $i$,
$1\leq i\leq\TZ{N}$, $\TZ{h}_i=\lceil k/256^{\TZm{N}-i}\rceil \bmod 256$.
(That is, the packet header encodes the length of the byte sequence, \TZ{k},
as a big-endian numeral, cf.\ p.~\pageref{page:big-endian}.)
However, if $k\geq256^{\TZm{N}}$, no byte sequence at all will be transmitted to the
external resource and instead an exit signal
\T{badsig} will be sent to \T{owner[\Z{R}]}.
\index{port!packet mode|)}
\end{itemize}
If the external resource is a file opened for writing, then the transmitted byte sequence
written to the file (but the file is not closed until the port is closed).
If the external resource is an opened driver or an external process, it can read the transmitted
byte sequence using the \I{read} function.
However, no
assumption should
be made about how many bytes of data will be read by each invocation of the \I{read}
function.
Note that the transmission of a byte sequence is atomic in that the bytes of a single
transmission (including any
prepended bytes) will never be separated by bytes from other transmissions.
\index{port!sending to a|)}

\section{Transmitting data from the outside to an \Erlang\ process}

\index{port!receiving from a|(}

When the external resource to which a port \TZ{R} is opened is a file and \T{in[\Z{R}]} is
\T{true}, then the contents of the file is written to the port as if by a single invocation
of \I{write}.  We will now describe what happens when an opened driver or an external process writes
data to a port using the \I{write} function.

If \T{out[\Z{R}]} is \T{false}, then nothing is transmitted.
Otherwise messages will be sent to
\T{owner[\Z{R}]}\index{owner port property@\T{owner} port property}\index{port!ownership}.

Let the complete sequence of bytes written (through one or more invocations of \I{write})
by the driver or external process be $\TZ{b}_1$, \ldots, $\TZ{b}_n$.
Depending on the value of \T{packeting[\Z{R}]}, the byte sequence may be
interpreted as containing packet headers, in which case each packet will be
delivered to \T{in[\Z{R}]} as a separate message.
\begin{itemize}
\item \index{port!stream mode|(}
If \T{packeting[\Z{R}]} is \T{stream}, all $n$ bytes will be delivered but
it is not specified how the byte sequence is to be divided into messages.
Each message $i$ (where $i\geq1$) contains the bytes
$\TZ{b}_{k_i}$, \ldots, $\TZ{b}_{k_{i+1}-1}$ where $k_1=1$ and for each $i$, either $k_i\leq n$ and
$k_i<k_{i+1}$, or $k_i=n+1$.  (Thus there should be no empty messages.)  It is encouraged that
each invocation of \I{write} (with a nonempty sequence of bytes) should cause a message to be
sent, in order to facilitate interaction between the driver or external process and the \Erlang\
processes.
\index{port!stream mode|)}
\item \index{port!packet mode|(}
If \T{packeting[\Z{R}]} is \T{\{packet,\Z{N}\}}, the sequence $\TZ{b}_1$, \ldots, $\TZ{b}_n$ is
interpreted as containing packet headers of \TZ{N} bytes each.  Each message $i$ (where $i\geq1$)
will contain the bytes $\TZ{b}_{k_i+\TZm{N}}$, \ldots, $\TZ{b}_{k_{i+1}-1}$ where $k_1=1$ and for each $i$,
either $k_i\leq n$,
\[l_i = \I{BigEndianValue}(\TZ{b}_{k_i}, \ldots, \TZ{b}_{k_i+\TZm{N}})\]
(cf.\ p.~\pageref{page:big-endian})
and $k_i+\TZ{N}+l_i=k_{i+1}$, or $k_i= n+1$.  
(That is, the first \TZ{N} bytes are interpreted as a packet header and the subsequent
bytes [where the number of bytes is given by the packet header] will be contained in a
single message.  Then comes another packet header, etc.  Note that there can be empty messages.)
It is encouraged that a message should be sent as soon as all the bytes of a package have been
written, in order to facilitate interaction between the driver or external process and the \Erlang\
processes.
\index{port!packet mode|)}
\end{itemize}
Each message to \T{owner[\Z{R}]} is on the form \T{\{\Z{R},\Z{T}\}} where \TZ{T} contains $k$ transmitted
bytes, but the term \TZ{T} depends on whether \T{in_format[\Z{R}]} is \T{binary} or \T{list}:
\begin{itemize}
\item \index{port!binary mode|(}
If \T{in_format[\Z{R}]} is \T{binary}, then \TZ{T} is a binary consisting of the $k$ bytes in the
order they were written.
\index{port!binary mode|)}
\item \index{port!list mode|(}
If \T{in_format[\Z{R}]} is \T{list}, then \TZ{T} is a list of the $k$ bytes in the
order they were written.
\index{port!list mode|)}
\end{itemize}
\index{port!receiving from a|)}

\section{Closing ports}

\label{section:closing-ports}
\index{port!closing|(}

A port can be closed (through an implementation-defined method)
by the external resource.  It can also be closed by an \Erlang\ process
using the BIF
\ifOld\T{port_close/1}\index{port_close/1 BIF@\T{port_close/1} BIF} (\S\ref{section:portclose1}).\fi
\ifStd\T{port:close/1}\index{port:close/1 BIF@\T{port:close/1} BIF} (\S\ref{section:port:close1}).\fi

The options \T{eof} and \T{noeof} to the BIF \T{open_port/2}
(\S\ref{section:opening-ports})
determine how the \Erlang\ process that owns
the port is notified if the port is closed by the external resource.
\index{port!closing|)}

\section{Ports, links and exit signals}

\label{section:port-linking}
\index{port!linking|(}

An \Erlang\ process may be linked to a port in the same way that it may be linked to a process
(\S\ref{section:links}).
A process \TZ{P} may set up (or remove) a link with a port \TZ{R} by calling the BIF
\T{link/1}\index{link/1 BIF@\T{link/1} BIF}
(or \T{unlink/1}\index{unlink/1 BIF@\T{unlink/1} BIF}) with \TZ{R} as argument.
When a port is closed, an exit signal will be sent to
every process linked to the port.  The actual exit signal is implementation-defined, except that
implementations are encouraged to use the exit signal \T{normal} when the driver or external process
completes normally.

\index{exit!signal|(}
When a port \TZ{R} receives an exit signal \TZ{T} from a process \TZ{P}, the following happens.
\begin{itemize}
\item If \TZ{T} is \T{normal}\index{normal@\T{normal}!exit signal} and \TZ{P}
is not \T{owner[\Z{R}]}, then nothing happens.
\item Otherwise, if \TZ{T} is \T{kill}\index{kill@\T{kill}!exit signal},
the port is closed and an exit signal \TZ{killed} is sent
to every process that is linked to the port.
\item Otherwise, the port is closed and an exit signal \TZ{T} is sent to every process that is
linked to the port.
\end{itemize}
Note that this behaviour is somewhat different from how a process handles incoming exit signals
(\S\ref{section:receiving-exit-signal}).
\index{exit!signal|)}
\index{port!linking|)}

\ifStd
\section{Ports and Unicode}

\label{section:ports-unicode}
\index{Unicode!and ports|(}

It should be noticed that all communication with ports is through sequences of bytes,
i.e., numbers between 0 and 255.  The codes of Unicode characters require two bytes
and there is no general provision for sending such characters directly to ports.
(However, note that characters \T{\char`\u0000} to \T{\char`\u00FF} can be sent directly
to ports.)

An arbitrary character string that is to be sent through a port must thus be transformed to a
sequence of bytes before being transmitted and there are several such standard transformations,
such as:
\begin{Lentry}
\item[Big-endian.]
Each character is represented by two bytes where the first is the high-order byte and
the second is the low-order byte of the character code.
\item[Little-endian.]
Each character is represented by two bytes where the first is the low-order byte and
the second is the high-order byte of the character code.
\item[UTF-8.]
\index{UTF-8 transformation@\T{UTF-8} transformation|(}
UTF-8 stands for UCS Transformation Format, 8-bit form, and is now part of the ISO/IEC
10646 standard \cite{utf8}.  Each Unicode character is represented by a single byte or
a sequence of two to four bytes.  All ASCII characters are represented by a single byte
and for character sequences that mostly or wholly consist of ASCII characters, the
transformation results in very short byte sequences.
\index{UTF-8 transformation@\T{UTF-8} transformation|)}
\item[UTF-7.]
\index{UTF-7 transformation@\T{UTF-7} transformation|(}
UTF-7 stands for UCS Transformation Format, 7-bit form, and is described in Internet Network
Working Group RFC-1642 \cite{rfc1642}.  Each Unicode character is represented by a single byte
or a sequence of bytes.  The ASCII letters and digits and ten other common ASCII characters
are each representable by a single byte.  The resulting sequence of bytes is safe for
Internet mail transmission (except where line length and line break restrictions are violated).
\index{UTF-7 transformation@\T{UTF-7} transformation|)}
\end{Lentry}

In order for an \Erlang\ process to send an arbitrary string to a port, it should be transformed
to a sequence of
bytes (represented, for example, as a binary) and then transmitted.  Of course, the process
receiving the sequence of bytes must be aware of the transformation if the original Unicode
characters are to be restored.

A similar process would be used to send Unicode character sequences from an external process to
an \Erlang\ process.
\index{Unicode!and ports|)}
\fi

\section{Static and dynamic properties of a port}

\label{section:port-state}
\index{port!state|(}

When a port is created, some properties of the port are determined and will be
in effect until the port is closed.  During that time also a dynamic state
is maintained, consisting of properties of the port that change as time passes.
This state is affected by and affects the behaviour of the port.

We refer to these as the \emph{static} and \emph{dynamic properties} of a port.  We
refer collectively to the values of the latter at a certain time as the \emph{state} of the port
at that time.

\subsection{Static properties}

\label{section:port-state-static}

\begin{Lentry}
\item[\T{command[\Z{R}]}]
\index{command port property@\T{command} port property|(}
For a port that was opened through a call to \T{open_port/2} with
\T{\{spawn,\Z{Cmd}\}}, the
value of \T{command[\Z{R}]} is a string that is either \TZ{Cmd}, if \TZ{Cmd} is a string,
or the printname of \TZ{Cmd}, if \TZ{Cmd} is an atom.
It can be accessed through a BIF call
\ifOld \T{erlang:port_info(\Z{R},name)}\index{port_info/2 BIF@\T{port_info}/2 BIF}
(\S\ref{section:portinfo2}).\fi
\ifStd \T{port:info(\Z{R},name)}\index{port:info/2 BIF@\T{port:info}/2 BIF}
(\S\ref{section:port:info2}).\fi
\index{command port property@\T{command} port property|)}

\item[\T{creation[\Z{R}]}]
\index{creation@\T{creation}!port property|(}
The value of \T{creation[\Z{R}]} is the value of \T{creation[\Z{N}]} for the
node \TZ{N} on which \TZ{R} was created.
\index{creation@\T{creation}!port property|)}

\item[\T{ID[\Z{R}]}]
\index{ID@\T{ID}!port property|(}
The value of \T{ID[\Z{R}]} is a nonnegative integer that is a serial number for \TZ{R}
on the node on which it was
created.  The value is used in the transformation
to the external term format (\S\ref{chapter:external-format}).
It can be accessed through a BIF call
\ifOld \T{erlang:port_info(\Z{R},id)}\index{port_info/2 BIF@\T{port_info}/2 BIF}
(\S\ref{section:portinfo2}).\fi
\ifStd \T{port:info(\Z{R},id)}\index{port:info/2 BIF@\T{port:info}/2 BIF}
(\S\ref{section:port:info2}).\fi
\index{ID@\T{ID}!port property|)}

\item[\T{in[\Z{R}]}]
\index{in port property@\T{in} port property|(}
When a port is opened it is decided whether data can be transmitted from the outside to
the process owning the port.  The value is \T{true} if one of the options \T{in} and \T{bi} was
given when the port was opened and \T{false} otherwise.  The value of \T{in[\Z{R}]} cannot
be accessed directly by \Erlang\ processes.
\index{in port property@\T{in} port property|)}

\item[\T{in_format[\Z{R}]}]
\index{in_format port property@\T{in_format} port property|(}
When a port is opened it is decided whether incoming data is sent as binaries or as lists of
bytes.  The value is \T{binary} or \T{list} depending on which of these options was given when
the port was opened.  The value of \T{in_format[\Z{R}]} cannot be accessed directly by
\Erlang\ processes.
\index{in_format port property@\T{in_format} port property|)}

\item[\T{node[\Z{R}]}]
\index{node@\T{node}!port property|(}
When a port is opened it is created on some node \T{node[\Z{R}]}.
This node never changes. Any process can access \T{node[\Z{R}]}
for a port \TZ{R} by calling the BIF
\T{node/1}\index{node/1 BIF@\T{node/1} BIF} (\S\ref{section:node1})
with \TZ{R} as argument.
\index{node@\T{node}!port property|)}

\item[\T{out[\Z{R}]}]
\index{out port property@\T{out} port property|(}
When a port is opened it is decided whether data can be transmitted from \Erlang\ processes
to the outside.  The value is \T{true} if one of the options \T{out} and \T{bi} was
given when the port was opened and \T{false} otherwise.  The value of \T{out[\Z{R}]} cannot
be accessed directly by \Erlang\ processes.
\index{out port property@\T{out} port property|)}

\item[\T{packeting[\Z{R}]}]
\index{packeting port property@\T{packeting} port property|(}
When a port is opened it is decided whether data is transmitted as a stream or as packets
beginning with a packet header of some length.  The value is \T{stream} or \T{\{packet,\Z{N}\}}
(where \TZ{N} is \T{1}, \T{2} or \T{4}) depending on which of these options was given when
the port was opened.  The value of \T{packeting[\Z{R}]} cannot be accessed directly by
\Erlang\ processes.
\index{packeting port property@\T{packeting} port property|)}
\end{Lentry}

\subsection{Dynamic properties}

\label{section:port-state-dynamic}

\begin{Lentry}
\item[\T{count_in[\Z{R}]}]
\index{count_in port property@\T{count_in} port property|(}
\T{count_in[\Z{R}]} is an integer that is the number of bytes read so far from the
port \TZ{R}.
It can be accessed through a BIF call
\ifOld\begin{alltt}
erlang:port_info(\Z{R},input)
\end{alltt}
\index{port_info/2 BIF@\T{port_info}/2 BIF}(\S\ref{section:portinfo2}).\fi
\ifStd\begin{alltt}
port:info(\Z{R},input)
\end{alltt}
\index{port:info/2 BIF@\T{port:info}/2 BIF}(\S\ref{section:port:info2}).\fi
\index{count_in port property@\T{count_in} port property|)}

\item[\T{count_out[\Z{R}]}]
\index{count_out port property@\T{count_out} port property|(}
\T{count_out[\Z{R}]} is an integer that is the number of bytes written so far to the
port \TZ{R}.
It can be accessed through a BIF call
\ifOld\begin{alltt}
erlang:port_info(\Z{R},output)
\end{alltt}
\index{port_info/2 BIF@\T{port_info}/2 BIF}(\S\ref{section:portinfo2}).\fi
\ifStd\begin{alltt}
port:info(\Z{R},output)
\end{alltt}
\index{port:info/2 BIF@\T{port:info}/2 BIF}(\S\ref{section:port:info2}).\fi
\index{count_out port property@\T{count_out} port property|)}

\item[\T{linked[\Z{R}]}]
\index{linked@\T{linked}!port property|(}
\index{port!linking|(}
The value is a representation of the set of PIDs that identify
the processes to which \TZ{R} is linked (\S\ref{section:links}).
It cannot be set directly but a PID \TZ{P} will be added to \T{linked[\Z{R}]}
if it is not already in it and \TZ{R} receives a linking request from \TZ{P}.

A PID \TZ{P} will be removed from \T{linked[\Z{R}]} if it is in the list and
either
\begin{itemize}
\item \TZ{P} receives an unlinking request from \TZ{Q}, or
\item \TZ{P} receives an exit signal \T{\{'EXIT',\Z{Q},\Z{Reason}\}} for some term
\TZ{Reason}, and the exit signal was sent due to process completion
(\S\ref{section:exit-signals}).
\end{itemize}

The value of \T{linked[\Z{R}]} can be accessed through a BIF call
\ifOld\begin{alltt}
erlang:port_info(\Z{R},links)
\end{alltt}
\index{port_info/2 BIF@\T{port_info}/2 BIF}(\S\ref{section:portinfo2}).\fi
\ifStd\begin{alltt}
port:info(\Z{R},links)
\end{alltt}
\index{port:info/2 BIF@\T{port:info}/2 BIF}(\S\ref{section:port:info2}).\fi
\index{port!linking|)}
\index{linked@\T{linked}!port property|)}

\item[\T{owner[\Z{R}]}]
\index{owner port property@\T{owner} port property|(}
The value is a PID and is the process that will receive messages when
data is written to the port \TZ{R} externally, will receive exit
signals when bad messages are sent to \TZ{R} and becomes linked with
\TZ{R} when \TZ{R} is opened. \T{owner[\Z{R}]} can be changed to a PID
\TZ{P} by sending a message \T{\{owner[\Z{R}],\{connect,\TZ{P}\}\}} to \TZ{R}.
The value of \T{owner[\Z{R}]} can be accessed through a BIF call
\ifOld\begin{alltt}
erlang:port_info(\Z{R},connected)
\end{alltt}
\index{port_info/2 BIF@\T{port_info}/2 BIF}(\S\ref{section:portinfo2}).\fi
\ifStd\begin{alltt}
port:info(\Z{R},connected)
\end{alltt}
\index{port:info/2 BIF@\T{port:info}/2 BIF}(\S\ref{section:port:info2}).\fi
\index{owner port property@\T{owner} port property|)}
\end{Lentry}

There will typically also be buffers for inbound and/or outbound data
but these are not referred to above and we do not describe them in
detail here.
\index{port!state|)}
