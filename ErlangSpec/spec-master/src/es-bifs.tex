%
% %CopyrightBegin%
%
% Copyright Ericsson AB 2017. All Rights Reserved.
%
% Licensed under the Apache License, Version 2.0 (the "License");
% you may not use this file except in compliance with the License.
% You may obtain a copy of the License at
%
%     http://www.apache.org/licenses/LICENSE-2.0
%
% Unless required by applicable law or agreed to in writing, software
% distributed under the License is distributed on an "AS IS" BASIS,
% WITHOUT WARRANTIES OR CONDITIONS OF ANY KIND, either express or implied.
% See the License for the specific language governing permissions and
% limitations under the License.
%
% %CopyrightEnd%
%

\chapter{Builtin functions}

\label{chapter:bifs}

As for any function application, when an application of a BIF is
evaluated all arguments are fully evaluated before the BIF itself is
called and begins executing.  When below we discuss the evaluation of
a BIF with $k$ parameters, we assume that the $k$ arguments have
already been evaluated and that their values are $\TZ{v}_1$, \ldots,
$\TZ{v}_k$.

For convenience we abuse our language slightly and may write
\begin{itemize}
\item ``calling a BIF'' when we mean ``evaluating an application of a BIF''
(where the values of the $k$ arguments are the terms
$\TZ{v}_1$, \ldots, $\TZ{v}_k$),
\item ''the BIF returns \ldots''\ when we mean ``the evaluation of
an application of the BIF completes normally with result \ldots'', and
\item ``the BIF exits with reason \ldots'' when we mean
``the evaluation of an application of the BIF exits with reason \ldots''.
\end{itemize}
% There is a proposal to change this!!!
\index{BIF!exit cause of|(}
In this chapter we write that a BIF ``exits with cause
\TZ{R}'' to express that the evaluation of an application of the BIF
completes abruptly with reason
\begin{alltt}
\{'EXIT',\{\Z{R},\{\Z{M},\Z{F},[\(\Z{v}\sb{1}\),\tdots,\(\Z{v}\sb{k}\)]\}\}\R{,}
\end{alltt}
where \TZ{M} is the name of the module and \TZ{F} is the symbol of the
function that called the BIF, and as usual, $\TZ{v}_1$, \ldots,
$\TZ{v}_k$ are the values of the arguments.
\index{BIF!exit cause of|)}

\ifOld
\index{badarg exit cause@\T{badarg} exit cause|(}
If the abnormal completion was because there was something wrong with
the value of an argument, e.g., is was of the wrong type or it was an
index outside the meaningful range, then \TZ{R} is always the atom
\T{badarg}.
\index{badarg exit cause@\T{badarg} exit cause|)}
\fi
\ifStd
\index{badarg exit cause@\T{badarg} exit cause|(}
If the abnormal completion was because the value of an argument was
not valid (e.g., it was not of a permitted type or it was not one of a
group of permitted alternatives such as a set of atoms), then \TZ{R}
is always the atom
\T{badarg}.
\index{badarg exit cause@\T{badarg} exit cause|)}
\index{badindex exit cause@\T{badindex} exit cause|(}
If the cause for the abrupt completion was an invalid index
in some kind of sequence, then \TZ{R} is always the atom
\T{badindex}.
\index{badindex exit cause@\T{badindex} exit cause|)}
\fi

\index{BIF!guard|(}
If it is not explicitly said about a BIF that it is a guard BIF, then
it is not a guard BIF.
\index{BIF!guard|)}

%Some BIFs are mentioned under more than one heading.

\section{Recognizer BIFs}

\label{section:recognizer-bifs}
\index{BIF!recognizer|(}
\ifOld
\index{atom/1 BIF@\T{atom/1} BIF|(}
\index{binary/1 BIF@\T{binary/1} BIF|(}
\index{constant/1 BIF@\T{constant/1} BIF|(}
\index{float/1 BIF@\T{float/1} BIF|(}
\index{function/1 BIF@\T{function/1} BIF|(}
\index{integer/1 BIF@\T{integer/1} BIF|(}
\index{list/1 BIF@\T{list/1} BIF|(}
\index{number/1 BIF@\T{number/1} BIF|(}
\index{pid/1 BIF@\T{pid/1} BIF|(}
\index{port/1 BIF@\T{port/1} BIF|(}
\index{record/2 BIF@\T{record/2} BIF|(}
\index{reference/1 BIF@\T{reference/1} BIF|(}
\index{tuple/1 BIF@\T{tuple/1} BIF|(}
\fi
\ifStd
\index{is_atom/1 BIF@\T{is_atom/1} BIF|(}
\index{is_binary/1 BIF@\T{is_binary/1} BIF|(}
\index{is_char/1 BIF@\T{is_char/1} BIF|(}
\index{is_cons/1 BIF@\T{is_cons/1} BIF|(}
\index{is_compound/1 BIF@\T{is_compound/1} BIF|(}
\index{is_float/1 BIF@\T{is_float/1} BIF|(}
\index{is_function/1 BIF@\T{is_function/1} BIF|(}
\index{is_integer/1 BIF@\T{is_integer/1} BIF|(}
\index{is_list/1 BIF@\T{is_list/1} BIF|(}
\index{is_null/1 BIF@\T{is_null/1} BIF|(}
\index{is_number/1 BIF@\T{is_number/1} BIF|(}
\index{is_pid/1 BIF@\T{is_pid/1} BIF|(}
\index{is_port/1 BIF@\T{is_port/1} BIF|(}
\index{is_ref/1 BIF@\T{is_ref/1} BIF|(}
\index{is_string/1 BIF@\T{is_string/1} BIF|(}
\index{is_tuple/1 BIF@\T{is_tuple/1} BIF|(}
\fi

\ifOld
\addcontentsline{toc}{subsection}{\protect\numberline{}{\T{atom/1}}}
\addcontentsline{toc}{subsection}{\protect\numberline{}{\T{binary/1}}}
\addcontentsline{toc}{subsection}{\protect\numberline{}{\T{constant/1}}}
\addcontentsline{toc}{subsection}{\protect\numberline{}{\T{float/1}}}
\addcontentsline{toc}{subsection}{\protect\numberline{}{\T{function/1}}}
\addcontentsline{toc}{subsection}{\protect\numberline{}{\T{integer/1}}}
\addcontentsline{toc}{subsection}{\protect\numberline{}{\T{list/1}}}
\addcontentsline{toc}{subsection}{\protect\numberline{}{\T{number/1}}}
\addcontentsline{toc}{subsection}{\protect\numberline{}{\T{pid/1}}}
\addcontentsline{toc}{subsection}{\protect\numberline{}{\T{port/1}}}
\addcontentsline{toc}{subsection}{\protect\numberline{}{\T{record/2}}}
\addcontentsline{toc}{subsection}{\protect\numberline{}{\T{reference/1}}}
\addcontentsline{toc}{subsection}{\protect\numberline{}{\T{tuple/1}}}
\fi
\ifStd
\addcontentsline{toc}{subsection}{\protect\numberline{}{\T{is_atom/1}}}
\addcontentsline{toc}{subsection}{\protect\numberline{}{\T{is_binary/1}}}
\addcontentsline{toc}{subsection}{\protect\numberline{}{\T{is_char/1}}}
\addcontentsline{toc}{subsection}{\protect\numberline{}{\T{is_compound/1}}}
\addcontentsline{toc}{subsection}{\protect\numberline{}{\T{is_cons/1}}}
\addcontentsline{toc}{subsection}{\protect\numberline{}{\T{is_float/1}}}
\addcontentsline{toc}{subsection}{\protect\numberline{}{\T{is_function/1}}}
\addcontentsline{toc}{subsection}{\protect\numberline{}{\T{is_integer/1}}}
\addcontentsline{toc}{subsection}{\protect\numberline{}{\T{is_list/1}}}
\addcontentsline{toc}{subsection}{\protect\numberline{}{\T{is_number/1}}}
\addcontentsline{toc}{subsection}{\protect\numberline{}{\T{is_null/1}}}
\addcontentsline{toc}{subsection}{\protect\numberline{}{\T{is_pid/1}}}
\addcontentsline{toc}{subsection}{\protect\numberline{}{\T{is_port/1}}}
\addcontentsline{toc}{subsection}{\protect\numberline{}{\T{is_ref/1}}}
\addcontentsline{toc}{subsection}{\protect\numberline{}{\T{is_string/1}}}
\addcontentsline{toc}{subsection}{\protect\numberline{}{\T{is_tuple/1}}}
\fi

\ifOld
In \OldErlang\ the recognizer BIFs are not true BIFs and can only be
used in guards as \NT{GuardRecognizer}.
There are twelve guard recognizers:
\T{atom/1},
\T{binary/1},
\T{constant/1},
\T{float/1},
\T{function/1},
\T{integer/1},
\T{list/1},
\T{number/1},
\T{pid/1},
\T{port/1},
\T{reference/1}, and
\T{tuple/1}.
As they all behave similarly, we describe them collectively.

\EVALUATION

If $\TZ{v}_1$ is one of the terms indicated in Table~\ref{table:typetests},
then the recognizer in the guard succeeds, otherwise it fails.

\begin{table}[hbp]
\begin{center}
\begin{tabular}{@{}ll@{}}
\hline
Recognizer & Succeeds if and only if the argument is \\
\hline
\T{atom/1} & an atom \\
\T{binary/1} & a binary \\
\T{constant/1} & of an elementary type (cf.\ \S\ref{chapter:types-terms}) \\
\T{float/1} & a float \\
\T{integer/1} & an integer \\
\T{function/1} & a function term \\
\T{list/1} & a cons or nil \\ % BAAAD NAME!!!
\T{number/1} & a number \\
\T{pid/1} & a PID \\
\T{port/1} & a port \\
\T{reference/1} & a reference \\
\T{tuple/1} & a tuple \\
\hline
\end{tabular}
\caption{\Erlang\ recognizer BIFs.}
\label{table:typetests}
\end{center}
\end{table}
\fi

\ifStd
A \emph{recognizer} BIF returns \T{true} for some terms and
\T{false} for all other terms.  Most recognizer BIFs are true
for terms of a certain type.  There are sixteen recognizer BIFs:
\T{is_atom/1},
\T{is_binary/1},
\T{is_char/1},
\T{is_cons/1},
\T{is_compound/1},
\T{is_float/1},
\T{is_function/1},
\T{is_integer/1},
\T{is_list/1},
\T{is_null/1},
\T{is_number/1},
\T{is_pid/1},
\T{is_port/1},
\T{is_ref/1},
\T{is_string/1}, and
\T{is_tuple/1}.
As they all behave similarly, we describe them collectively.

\index{record/2@\T{record/2}!recognizer|(}
Note that although a \NT{GuardRecordTest} \T{record(\Z{E},\Z{R})}
\S\ref{section:record2} is not an application
of a BIF, it is used as a recognizer for records of a certain type.
\index{record/2@\T{record/2}!recognizer|)}

All recognizer BIFs are guard BIFs (to be used in \NT{GuardRecognizer}).

\TYPE

For each recognizer BIF \TZ{F},
\begin{textdisplay}
\T{\Z{F}(term()) -> bool()}.
\end{textdisplay}

\EXITS

All recognizer BIFs always complete normally.

\EVALUATION

If $\TZ{v}_1$ is one of the terms indicated in Table~\ref{table:typetests},
then \T{true} is returned, otherwise \T{false} is returned.

All recognizers except \T{is_list/1} and \T{is_string/1} should take
$O(1)$ time.  \T{is_list/1} and \T{is_string/1} should take $O(n)$
time, where $n$ is the length of the (presumed) list.

\begin{table}[hbp]
\begin{center}
\begin{tabular}{@{}ll@{}}
\hline
BIF & Returns \T{true} if and only if the argument is \\
\hline
\T{is_atom/1} & an atom \\
\T{is_binary/1} & a binary \\
\T{is_char/1} & a character \\
\T{is_compound/1} & of a compound type (cf.\ \S\ref{chapter:types-terms}) \\
\T{is_cons/1} & a cons \\
\T{is_float/1} & a float \\
\T{is_function/1} & a function term \\
\T{is_integer/1} & an integer \\
\T{is_list/1} & a (proper) list \\
\T{is_null/1} & nil \\
\T{is_number/1} & a number \\
\T{is_pid/1} & a PID \\
\T{is_port/1} & a port \\
\T{is_ref/1} & a ref \\
\T{is_string/1} & a (proper) string \\
\T{is_tuple/1} & a tuple \\
\hline
\end{tabular}
\caption{\Erlang\ recognizer BIFs.}
\label{table:typetests}
\end{center}
\end{table}
\fi	%\ifStd

\ifOld
Note that the name of the recognizer \T{list/1} is inaccurate: it does
\emph{not} test whether its argument is a list.  (It succeeds for all
lists but also for some terms that are not lists, such as \T{[a|b]}).

\index{function/1@\T{function/1}!recognizer|(}
\index{record/2@\T{record/2}!recognizer|(}
Note that although a \NT{GuardRecordTest} \T{record(\Z{E},\Z{R})}
\S\ref{section:record2} is not an application of a BIF, it is used as
a recognizer for records of a certain type.  In \OldErlang\ function
terms are implemented with tuples.
\index{function/1@\T{function/1}!recognizer|)}
\index{record/2@\T{record/2}!recognizer|)}
\fi

\ifOld
\index{atom/1 BIF@\T{atom/1} BIF|)}
\index{binary/1 BIF@\T{binary/1} BIF|)}
\index{constant/1 BIF@\T{constant/1} BIF|)}
\index{float/1 BIF@\T{float/1} BIF|)}
\index{function/1 BIF@\T{function/1} BIF|)}
\index{integer/1 BIF@\T{integer/1} BIF|)}
\index{list/1 BIF@\T{list/1} BIF|)}
\index{number/1 BIF@\T{number/1} BIF|)}
\index{pid/1 BIF@\T{pid/1} BIF|)}
\index{port/1 BIF@\T{port/1} BIF|)}
\index{record/2 BIF@\T{record/2} BIF|)}
\index{reference/1 BIF@\T{reference/1} BIF|)}
\index{tuple/1 BIF@\T{tuple/1} BIF|)}
\fi
\ifStd
\index{is_atom/1 BIF@\T{is_atom/1} BIF|)}
\index{is_binary/1 BIF@\T{is_binary/1} BIF|)}
\index{is_char/1 BIF@\T{is_char/1} BIF|)}
\index{is_cons/1 BIF@\T{is_cons/1} BIF|)}
\index{is_compound/1 BIF@\T{is_compound/1} BIF|)}
\index{is_float/1 BIF@\T{is_float/1} BIF|)}
\index{is_function/1 BIF@\T{is_function/1} BIF|)}
\index{is_integer/1 BIF@\T{is_integer/1} BIF|)}
\index{is_list/1 BIF@\T{is_list/1} BIF|)}
\index{is_null/1 BIF@\T{is_null/1} BIF|)}
\index{is_number/1 BIF@\T{is_number/1} BIF|)}
\index{is_pid/1 BIF@\T{is_pid/1} BIF|)}
\index{is_port/1 BIF@\T{is_port/1} BIF|)}
\index{is_ref/1 BIF@\T{is_ref/1} BIF|)}
\index{is_string/1 BIF@\T{is_string/1} BIF|)}
\index{is_tuple/1 BIF@\T{is_tuple/1} BIF|)}
\fi
\index{BIF!recognizer|)}

\section{Builtin functions on atoms}

\label{section:atom-bifs}
\index{atom!BIFs|(}

\subsection{\T{atom\char'137to\char'137list/1}}

\index{atom_to_list/1 BIF@\T{atom_to_list/1} BIF|(}

An atom is converted to a list of \ifOld characters\fi\ifStd character
codes\fi.

\TYPE

\T{atom_to_list(atom()) -> [int()]}.

\EXITS

Exits with cause \T{badarg} if $\TZ{v}_1$ is not an atom.

\EVALUATION

The BIF returns the printname of the atom $\TZ{v}_1$ represented by
a list of
\ifStd integers that are Unicode codes of a sequence of \fi
characters.

\EXAMPLES

\T{atom_to_list(foo)} \RETURNS\ \T{[102,111,111]}\ifOld, i.e., \T{"foo"}\fi; \\
\T{atom_to_list('')} \RETURNS\ \T{[]}\ifOld, i.e., \T{""}\fi; \\
\T{atom_to_list('T \char`\%@\char`\\'\char`\#')} \RETURNS\ \T{[84,32,37,64,39,35]}\ifOld,
i.e., \T{"T \char`\%@'\char`\#"}\fi; \\
\T{atom_to_list('456')} \RETURNS\ \T{[52,53,54]}\ifOld, i.e., \T{"456"}\fi; \\
\T{atom_to_list(456)} \EXITSWITH\ \T{\{badarg,\tdots\}}.
\index{atom_to_list/1 BIF@\T{atom_to_list/1} BIF|)}

\subsection{\T{list\char'137to\char'137atom/1}}

\index{list_to_atom/1 BIF@\T{list_to_atom/1} BIF|(}

A list of \ifOld characters \fi \ifStd character codes \fi is
converted to an atom.

\TYPE

\T{list_to_atom([int()]) -> atom()}.

\EXITS

Exits with cause \T{badarg} if $\TZ{v}_1$ is not a list of
\ifStd integers that are codes of Unicode \fi
characters.

\EVALUATION

The BIF returns the atom that has a printname consisting of the characters
\ifStd obtained by decoding the character codes \fi
in $\TZ{v}_1$.

\EXAMPLES

\T{list_to_atom([102,111,111])}\ifOld, i.e., \T{list_to_atom("foo")}\fi\ \RETURNS\ \T{foo}; \\
\T{list_to_atom([])}\ifOld, i.e., \T{list_to_atom("")}\fi\ \RETURNS\ \T{''}; \\
\T{list_to_atom([84,32,37,64,39,35])}\ifOld, i.e., \T{list_to_atom("T \char`\%@'\char`\#")}\fi\ \RETURNS\ \T{'T \char`\%@\char`\\'\char`\#'}; \\
\T{list_to_atom([52,53,54])}\ifOld, i.e., \T{list_to_atom("456")}\fi\ \RETURNS\ \T{'456'}; \\
\T{list_to_atom([102,-5,111])} \EXITSWITH\ \T{\{badarg,\tdots\}}.
\index{list_to_atom/1 BIF@\T{list_to_atom/1} BIF|)}
\index{atom!BIFs|)}

\section{Builtin arithmetic functions}
\index{number!BIFs|(}

\label{section:number-bifs}

\subsection{\T{abs/1}}

\index{abs/1 BIF@\T{abs/1} BIF|(}

The magnitude of a number is computed.  \T{abs/1} is a guard BIF.

\TYPE

\T{abs(int()) -> int() ; \\
abs(float()) -> float()}.

\EXITS

Exits with cause \T{badarg} if $\TZ{v}_1$ is not a number.
May exit with \T{integer_overflow} when $\TZ{v}_1$ is an integer (see below).

\EVALUATION

If $\TZ{v}_1$ is an integer, compute
$\mathit{abs}_I(\Er[\TZ{v}_1])$; otherwise $\TZ{v}_1$ is a float, compute
$\mathit{abs}_F(\Er[\TZ{v}_1])$. Let $r$ be the result.
If $r$ is a number, the the BIF returns $\Re[r]$;
otherwise the BIF exits with cause $\Re[r]$.

\EXAMPLES

\T{abs(42)} \RETURNS\ \T{42}; \\
\T{abs(-88)} \RETURNS\ \T{88}; \\
\T{abs(5.0)} \RETURNS\ \T{5.0}; \\
\T{abs(-0.1)} \RETURNS\ \T{0.1}; \\
\T{abs(whoopee)} \EXITSWITH\ \T{\{badarg,\tdots\}}.
\index{abs/1 BIF@\T{abs/1} BIF|)}

\ifStd
\subsection{\T{sign/1}}

\label{section:sign1}
\index{sign/1 BIF@\T{sign/1} BIF|(}

The sign of a number is computed.  The result is always
an integer.  \T{sign/1} is a guard BIF.

\TYPE

\T{sign(num()) -> int()}.

\EXITS

Exits with cause \T{badarg} if $\TZ{v}_1$ is not a number.

\EVALUATION

If $\TZ{v}_1$ is an integer, compute
$\mathit{sign}_I(\Er[\TZ{v}_1])$; otherwise $\TZ{v}_1$ is a float, compute
$\mathit{nearest}_{F\to I}(\mathit{sign}(\Er[\TZ{v}_1]))$. Let $r$ be the result.
If $r$ is an integer, the the BIF returns $\Re[r]$;
otherwise the BIF exits with cause $\Re[r]$.

\EXAMPLES

\T{sign(42)} \RETURNS\ \T{1}; \\
\T{sign(-88.56)} \RETURNS\ \T{-1}; \\
\T{sign(0.0)} \RETURNS\ \T{0}; \\
\T{sign(0)} \RETURNS\ \T{0}; \\
\T{sign(whoopee)} \EXITSWITH\ \T{\{badarg,\tdots\}}.
\index{sign/1 BIF@\T{sign/1} BIF|)}
\fi

\subsection{\T{float/1}}

\index{float/1 BIF@\T{float/1} BIF|(}

There are two BIFs named \T{float/1} and which one of them is denoted in an
application \T{float(\Z{E})} depends on the context in which the application appears.

If the application is a \NT{GuardRecognizer} expression
(\S\ref{section:guards}), then it is the guard test \T{float/1}
described in \S\ref{section:recognizer-bifs}; otherwise it is a
function converting numbers to floating-point numbers (which may
appear in guard expressions).  The following description is for the
latter case.

Both BIFs \T{float/1} are guard BIFs.

\TYPE

\T{float(num()) -> float()}.

\EXITS

Exits with cause \T{badarg} if $\TZ{v}_1$ is not a number.
May exit with \T{float_overflow} when $\TZ{v}_1$ is an integer (see below).

\EVALUATION

The evaluation depends on the type of $\TZ{v}_1$:
\begin{itemize}
\item If $\TZ{v}_1$ is a float, it is returned.
\item If $\TZ{v}_1$ is an integer, compute
$\mathit{cvt}_{I\to F}(\Er[\TZ{v}_1])$, let the result be $r$.
If $r$ is a float, the the BIF returns $\Re[r]$;
otherwise the BIF exits with cause $\Re[r]$.
\end{itemize}

\EXAMPLES

\T{float(3)} \RETURNS\ \T{3.0}; \\
\T{float(0)} \RETURNS\ \T{0.0}; \\
\T{float(123456789123456789123456789)} \RETURNS\ \T{1.23456789123457E26}\ifStd
(on an implementation with $r=2$ and $p=53$, cf.\ \S\ref{section:float-type})\fi; \\
\T{float(whoopee)} \EXITSWITH\ \T{\{badarg,\tdots\}}.
\index{float/1 BIF@\T{float/1} BIF|)}

\subsection{\T{float\char'137to\char'137list/1}}

\label{section:floattolist1}
\index{float_to_list/1 BIF@\T{float_to_list/1} BIF|(}

The function produces a list of \ifOld characters \fi
\ifStd integers that are character codes \fi
of a printed representation of a float.

\TYPE

\T{float_to_list(float()) -> [int()]}.

\EXITS

Exits with cause \T{badarg} if $\TZ{v}_1$ is not a float (there is thus no coercion
from integers).

\EVALUATION

The BIF returns the list of
\ifStd integers that are Unicode codes of the sequence of \fi
characters which is like the canonical decimal numeral for $\Er[\TZ{v}_1]$
(\S\ref{section:float-to-numeral}) except that:
\begin{itemize}
\item There are exactly 20 digits between the decimal point and the `e'.
The right end is adjusted by dropping digits or padding with zeroes.
\item If the character after `e' is not a minus sign, then a plus sign is inserted.
\item If the exponent part (i.e., the digits after `e') has only one digit,
a zero is inserted before that digit.
\end{itemize}

\EXAMPLES

\T{float_to_list(-0.00672)} \RETURNS\ \\
\T{[45,54,46,55,50,48,48,48,48,48,48,48,48,48,48,48,48,48,50,55,56,54,55,101,45,48,51]}\ifOld, i.e.,
\T{"-6.72000000000000027867e-03"}\fi; \\
\T{float_to_list(13e4)} \RETURNS\ \\
\T{[49,46,51,48,48,48,48,48,48,48,48,48,48,48,48,48,48,48,48,48,48,48,101,43,48,53]}\ifOld, i.e.,
\T{"1.30000000000000000000e+05"}\fi; \\
\T{float_to_list(0.0)} \RETURNS\ \\
\T{[48,46,48,48,48,48,48,48,48,48,48,48,48,48,48,48,48,48,48,48,48,48,101,43,48,48]}\ifOld, i.e.,
\T{"0.00000000000000000000e+00"}\fi; \\
\T{float_to_list(636.9e121)} \RETURNS\ \\
\T{[54,46,51,54,56,57,57,57,57,57,57,57,57,57,57,57,57,53,52,53,53,51,101,43,49,50,51]}\ifOld, i.e.,
\T{"6.36899999999999954553e+123"}\fi; \\
\T{float_to_list(whoopee)} \EXITSWITH\ \T{\{badarg,\tdots\}}.
\index{float_to_list/1 BIF@\T{float_to_list/1} BIF|)}

\subsection{\T{integer\char'137to\char'137list/1}}

\label{section:integertolist1}
\index{integer_to_list/1 BIF@\T{integer_to_list/1} BIF|(}

The function produces a list of \ifOld characters \fi
\ifStd integers that are character codes \fi
of a printed representation of an integer.

\TYPE

\T{integer_to_list(int()) -> [int()]}.

\EXITS

Exits with cause \T{badarg} if $\TZ{v}_1$ is not an integer.

\EVALUATION

The BIF returns the list of
\ifStd integers that are Unicode codes of the sequence of \fi
characters making up the canonical decimal numeral for $\Er[\TZ{v}_1]$
(\S\ref{section:integer-to-numeral}).

\EXAMPLES

\T{integer_to_list(0)} \RETURNS\ \T{[48]}\ifOld, i.e., \T{"0"}\fi; \\
\T{integer_to_list(42)} \RETURNS\ \T{[52,50]}\ifOld, i.e., \T{"42"}\fi; \\
\T{integer_to_list(-39)} \RETURNS\ \T{[45,51,57]}\ifOld, i.e., \T{"-39"}\fi; \\
\T{integer_to_list(whoopee)} \EXITSWITH\ \T{\{badarg,\tdots\}}.
\index{integer_to_list/1 BIF@\T{integer_to_list/1} BIF|)}

\subsection{\T{list\char'137to\char'137float/1}}

\label{section:listtofloat1}
\index{list_to_float/1 BIF@\T{list_to_float/1} BIF|(}

A float is obtained from a float literal represented by a list of
\ifStd integers that are Unicode codes of \fi the characters of the literal.

\TYPE

\T{list_to_float([int()]) -> float()}.

\EXITS

Exits with cause \T{badarg} if $\TZ{v}_1$ is not a list of
\ifStd integers that are the codes of Unicode \fi
characters making up a float numeral, or if the number
it represents is not representable as a float (see below).
Leading or trailing spaces are not permitted.

\EVALUATION

If the sequence of characters denotes a number $f\in F$
(\S\ref{section:numeral-to-float}), then
$\Re[f]$ is returned, otherwise the BIF exits with cause \T{badarg}.

\EXAMPLES

\T{list_to_float([45,49,55,46,53,48,48])}\ifOld, i.e., \\
\T{list_to_float("-17.500")}\fi\ \RETURNS\ \T{-17.5}; \\
\T{list_to_float([49,50,51,46,52,53,101,56,55])}\ifOld, i.e., \\
\T{list_to_float("123.45e87")}\fi\ \RETURNS\ \T{1.23450e89}; \\
\T{list_to_float([54,50,46,53,69,45,51])}\ifOld, i.e., \\
\T{list_to_float("62.5E-3")}\fi\ \RETURNS\ \T{0.0625}; \\
\T{list_to_float([54,50,53,101,45,52])}\ifOld, i.e., \\
\T{list_to_float("625e-4")}\fi\ \EXITSWITH\ \T{\{badarg,\tdots\}}; \\
\T{list_to_float(whoopee)} \EXITSWITH\ \T{\{badarg,\tdots\}}.
\index{list_to_float/1 BIF@\T{list_to_float/1} BIF|)}

\subsection{\T{list\char'137to\char'137integer/1}}

\label{section:listtointeger1}
\index{list_to_integer/1 BIF@\T{list_to_integer/1} BIF|(}

An integer is obtained from an integer literal represented by a list of
\ifStd integers that are Unicode codes of \fi the characters of the literal.

\TYPE

\T{list_to_integer([int()]) -> integer()}.

\EXITS

Exits with cause \T{badarg} if $\TZ{v}_1$ is not a list of
\ifStd integers that are the codes of Unicode \fi
characters making up a decimal integer numeral, or if the
integer it represents is too large to be represented (see below).
Leading or trailing spaces are not permitted.

\EVALUATION

If the sequence of characters denotes an integer $i\in I$
(\S\ref{section:numeral-to-integer}), then
$\Re[i]$ is returned, otherwise the BIF exits with cause \T{badarg}.

\EXAMPLES

\T{list_to_integer([52,50])}\ifOld, i.e., \T{list_to_integer("42")}\fi\ \RETURNS\ \T{42}; \\
\T{list_to_integer([48,48,48,48,48,48,53,54,55])}\ifOld, i.e., \T{list_to_integer("000000567")}\fi\ \RETURNS\ \T{567}; \\
\T{list_to_integer([45,51,57])}\ifOld, i.e., \T{list_to_integer("-39")}\fi\ \RETURNS\ \T{-39}; \\
\T{list_to_integer([111,105,110,107])}\ifOld, i.e., \T{list_to_integer("oink")}\fi\ \EXITSWITH\ \T{\{badarg,\tdots\}}; \\
\T{list_to_integer(whoopee)} \EXITSWITH\ \T{\{badarg,\tdots\}}.
\index{list_to_integer/1 BIF@\T{list_to_integer/1} BIF|)}

\subsection{\T{round/1}}

\index{round/1 BIF@\T{round/1} BIF|(}

The integer closest to a given number is computed.
\T{round/1} is a guard BIF.

\TYPE

\T{round(num()) -> int()}.

\EXITS

Exits with cause \T{badarg} if $\TZ{v}_1$ is not a number.
May exit with \T{integer_overflow} when $\TZ{v}_1$ is a float (see below).

\EVALUATION

The evaluation depends on the type of $\TZ{v}_1$:
\begin{itemize}
\item If $\TZ{v}_1$ is an integer, it is returned.
\item If $\TZ{v}_1$ is a float, compute
$\mathit{nearest}_{F\to I}(\Er[\TZ{v}_1])$, let the result be $r$.
If $r$ is an integer, the the BIF returns $\Re[r]$;
otherwise the BIF exits with cause $\Re[r]$.
\end{itemize}

\EXAMPLES

\T{round(42)} \RETURNS\ \T{42}; \\
\T{round(88.56)} \RETURNS\ \T{89}; \\
\T{round(-88.56)} \RETURNS\ \T{-89}; \\
\T{round(0.0)} \RETURNS\ \T{0}; \\
\T{round(whoopee)} \EXITSWITH\ \T{\{badarg,\tdots\}}.
\index{round/1 BIF@\T{round/1} BIF|)}

\subsection{\T{trunc/1}}

\index{trunc/1 BIF@\T{trunc/1} BIF|(}

The first float between a given number and zero is computed.
\T{trunc/1} is a guard BIF.

\TYPE

\T{trunc(num()) -> int()}.

\EXITS

Exits with cause \T{badarg} if $\TZ{v}_1$ is not a number.
May exit with \T{integer_overflow} when $\TZ{v}_1$ is a float (see below).

\EVALUATION

The evaluation depends on the type of $\TZ{v}_1$:
\begin{itemize}
\item If $\TZ{v}_1$ is an integer, it is returned.
\item If $\TZ{v}_1$ is a float, compute
$\mathit{truncate}_{F\to I}(\Er[\TZ{v}_1])$, let the result be $r$.
If $r$ is an integer, the the BIF returns $\Re[r]$;
otherwise the BIF exits with cause $\Re[r]$.
\end{itemize}

\EXAMPLES

\T{trunc(42)} \RETURNS\ \T{42}; \\
\T{trunc(88.56)} \RETURNS\ \T{88}; \\
\T{trunc(-88.56)} \RETURNS\ \T{-88}; \\
\T{trunc(0.0)} \RETURNS\ \T{0}; \\
\T{trunc(whoopee)} \EXITSWITH\ \T{\{badarg,\tdots\}}.
\index{trunc/1 BIF@\T{trunc/1} BIF|)}
\index{number!BIFs|)}

\section{Builtin functions on binaries}

\label{section:binary-bifs}
\index{binary!BIFs|(}

Binaries are described in \S\ref{section:binaries}.

\subsection{\T{binary\char'137to\char'137list/1}}
\index{binary_to_list/1 BIF@\T{binary_to_list/1} BIF|(}

A list of bytes that are the elements of a binary, in the same order
as in the binary, is returned.

\TYPE

\T{binary_to_list(bin()) -> [int()]}.

\EXITS

\T{binary_to_list/1} exits with cause \T{badarg} if $\TZ{v}_1$ is not a binary.

\EVALUATION

Suppose that $\TZ{v}_1$ is a binary consisting of the bytes
$\TZ{I}_1$, \ldots, $\TZ{I}_k$.
A list \T{[$\TZ{I}_1$,\tdots,$\TZ{I}_k$]} is returned.
The time for computing the answer should be $O(k)$.
\index{binary_to_list/1 BIF@\T{binary_to_list/1} BIF|)}

\subsection{\T{binary\char'137to\char'137list/3}}
\index{binary_to_list/3 BIF@\T{binary_to_list/3} BIF|(}

A list of integers that are part of the elements of a binary, in the
same order as in the binary, is returned.

\TYPE

\T{binary_to_list(bin(),int(),int()) -> [int()]}.

\EXITS

% Note:
% the proposed binary:to_list takes different arguments.

\T{binary_to_list/3} exits with cause \T{badarg} if $\TZ{v}_1$ is not a binary,
if $\TZ{v}_2$ or $\TZ{v}_3$ is not an integer.
It exits with cause \ifOld \T{badarg} \fi \ifStd \T{\badindex} \fi
if the integers represented
by $\TZ{v}_2$ and $\TZ{v}_3$ are out of range (see below).

\EVALUATION

Suppose that $\TZ{v}_1$ is a binary consisting of the bytes
$\TZ{I}_1$, \ldots, $\TZ{I}_k$ and that $i=\Er[\TZ{v}_2]$ and $j=\Er[\TZ{v}_3]$.
\T{binary_to_list/2} exits with cause \T{badarg} if $i<1$, $j<i$ or $j>k$.
% Should change so j=i-1 or even any j for which j<i is permitted???
A list \T{[$\TZ{I}_i$,\tdots,$\TZ{I}_j$]} is returned.
% Note: This is bad bad bad bad!
The result is thus always a nonempty list.

\iffalse
The list returned depends on $i$ and $j$:
\begin{itemize}
\item If $i\leq j$, then
a list \T{[$\TZ{I}_i$,\tdots,$\TZ{I}_j$]} is returned.
\item If $i>j$, then an empty list is returned.
\end{itemize}
\fi

The time for computing the result should be $O(j)$.
\index{binary_to_list/3 BIF@\T{binary_to_list/3} BIF|)}

\subsection{\T{binary\char'137to\char'137term/1}}

\label{section:binarytoterm1}
\index{binary_to_term/1 BIF@\T{binary_to_term/1} BIF|(}
\index{term!external format|(}

Given a binary that is a representation in the external format
(\S\ref{chapter:external-format}) of a term, that term is returned.

\TYPE

\T{binary_to_term(bin()) -> term()}.

\EXITS

\T{binary_to_term/1} exits with cause \T{badarg} if $\TZ{v}_1$ is not a binary.
It also exits with cause \T{badarg} if the elements of $\TZ{v}_1$ are not
the external format representation of some term (see below).

\EVALUATION

If the bytes that are the elements of $\TZ{v}_1$ constitute a
representation in the external format
(\S\ref{chapter:external-format}) of some term \TZ{t}, then \TZ{t} is
returned; otherwise \T{binary_to_term/1} exits with cause \T{badarg}.

% Is this true???
The time for computing the result should be $O(k)$.
\index{binary_to_term/1 BIF@\T{binary_to_term/1} BIF|)}
\index{term!external format|)}

\subsection{\T{concat\char'137binary/1}}

\index{concat_binary/1 BIF@\T{concat_binary/1} BIF|(}

Given a list of binaries, one binary is returned which has
as elements the elements of the binaries in the list, in the same order as
in which the elements appear in the binaries and the binaries in the list.

\TYPE

\T{concat_binary([bin()]) -> bin()}.

\EXITS

\T{concat_binary/1} exits with cause \T{badarg} if $\TZ{v}_1$ is not a
list of binaries.

\EVALUATION

Suppose that $\TZ{v}_1$ is a list
\T{[$\T{B}_1$,\tdots,$\Z{B}_k$]}, such that each
$\TZ{B}_i$, $1\leq i\leq k$, is a binary consisting of the bytes
$\TZ{b}_{i,1}$, \ldots, $\TZ{b}_{i,n_i}$.
A binary consisting of the bytes
$\TZ{b}_{1,1}$, \ldots, $\TZ{b}_{1,n_1}$, \ldots,
$\TZ{b}_{k,1}$, \ldots, $\TZ{b}_{k,n_k}$
is returned.

The time for computing the result should be $O(\max(\sum_{i=1}^{k}n_i,k))$.
\index{concat_binary/1 BIF@\T{concat_binary/1} BIF|)}

\subsection{\T{list\char'137to\char'137binary/1}}

\index{list_to_binary/1 BIF@\T{list_to_binary/1} BIF|(}

Given a list of bytes, a binary consisting of the same bytes in the
same order is returned.

\TYPE

\T{list_to_binary([int()]) -> bin()}.

\EXITS

\T{list_to_binary/1} exits with cause \T{badarg} if $\TZ{v}_1$ is not a list of bytes.

\EVALUATION

Suppose that $\TZ{v}_1$ is a list of bytes \T{[$\Z{I}_1$,\tdots,$\Z{I}_k$]}.
A binary consisting of the bytes $\TZ{I}_1$, \ldots, $\TZ{I}_k$ is returned.

The time for computing the result should be $O(k)$.
\index{list_to_binary/1 BIF@\T{list_to_binary/1} BIF|)}

\subsection{\T{size/1}}

\index{size/1 BIF@\T{size/1} BIF|(}
See \S\ref{section:size1}.
\index{size/1 BIF@\T{size/1} BIF|)}

\subsection{\T{split\char'137binary/2}}

\label{section:splitbinary2}
\index{split_binary/2 BIF@\T{split_binary/2} BIF|(}

A binary is split into two binaries, where the number of elements in
the first binary is given.

\TYPE

\T{split_binary(bin(),int()) -> \{bin(),bin()\}}.

\EXITS

\T{split_binary/2} exits with cause \T{badarg} if $\TZ{v}_1$ is not a binary or
$\TZ{v}_2$ is not an integer.  It also exits with cause
\ifOld \T{badarg} \fi \ifStd \T{\badindex} \fi
if the number represented by $\TZ{v}_2$
is out of range (see below).

\EVALUATION

Suppose that $\TZ{v}_1$ is a binary consisting of the bytes
$\TZ{I}_1$, \ldots, $\TZ{I}_k$ and that $i=\Er[\TZ{v}_2]$.
The evaluation depends on $i$ and $k$:
\begin{itemize}
\item If $i<0$ or $i>k$, then \T{split_binary/2} exits with cause
\ifOld \T{badarg}\fi \ifStd \T{\badindex}\fi.
\item Otherwise a 2-tuple of binaries \T{\{$\Z{B}_l$,$\Z{B}_r$\}} is returned,
where $\TZ{B}_l$ consists of the bytes $\TZ{I}_1$, \ldots, $\TZ{I}_i$
and $\TZ{B}_r$ consists of the bytes $\TZ{I}_{i+1}$, \ldots, $\TZ{I}_k$.
(If $i=0$ then $\TZ{B}_l$ is an empty binary and $\TZ{B}_r$ equals
$\TZ{v}_1$, while if $i=k$ then $\TZ{B}_l$ equals
$\TZ{v}_1$ and $\TZ{B}_r$ is an empty binary.)
\end{itemize}
The time for computing the answer should be $O(1)$.
\index{split_binary/2 BIF@\T{split_binary/2} BIF|)}

\subsection{\T{term\char'137to\char'137binary/1}}

\label{section:termtobinary1}
\index{term_to_binary/1 BIF@\T{term_to_binary/1} BIF|(}
\index{term!external format|(}

Given a term, a binary having elements that represent the term in the external format
(\S\ref{chapter:external-format}) is returned.

\TYPE

\T{term_to_binary(term()) -> bin()}.

\EXITS

\ifStd
\T{term_to_binary/1} exits with cause \T{badarg} if $\TZ{v}_1$ is not a
term for which the implementation provides an external
representation.\footnote{The only case for which this is allowed to happen is
if $\TZ{v}_1$ contains a function.}
\fi
\ifOld
\T{term_to_binary/1} always completes normally.
\fi

\EVALUATION

Let the representation in the external format
(\S\ref{chapter:external-format}) of $\TZ{v}_1$ be
the sequence of bytes
$\TZ{I}_1$, \ldots, $\TZ{I}_k$.  A binary consisting
of these bytes, in that order, is returned.
% too strong???
The time for computing the answer should be $O(k)$.
\index{term_to_binary/1 BIF@\T{term_to_binary/1} BIF|)}
\index{term!external format|)}
\index{binary!BIFs|)}

\section{Builtin functions on tuples}

\label{section:tuple-bifs}
\index{tuple!BIFs|(}

Tuples are described in \S\ref{section:tuples}.

\subsection{\T{element/2}}

\index{element/2 BIF@\T{element/2} BIF|(}

One element of a tuple is returned.
\T{element/2} is a guard BIF.

\TYPE

\T{element(int(),tuple()) -> term()}.

\EXITS

\T{element/2} exits with cause \T{badarg} if $\TZ{v}_1$ is not an integer or
$\TZ{v}_2$ is not a tuple. It also exits with cause
\ifOld \T{badarg} \fi \ifStd \T{\badindex} \fi
if the number represented by $\TZ{v}_1$ is out of range (see below).

\EVALUATION

Suppose that $i=\Er[\TZ{v}_1]$ and that $\TZ{v}_2$
is a tuple with elements $\TZ{T}_1$, \ldots, $\TZ{T}_k$.
The evaluation depends on $i$ and $k$:
\begin{itemize}
\item If $i<1$ or $i>k$, then \T{element/2} exits with
cause \ifOld \T{badarg}\fi \ifStd \T{\badindex}\fi.
\item Otherwise, $\TZ{T}_i$ is returned.
\end{itemize}
The time for computing the result should be $O(1)$.
\index{element/2 BIF@\T{element/2} BIF|)}

\subsection{\T{list\char'137to\char'137tuple/1}}

\label{section:listtotuple1}
\index{list_to_tuple/1 BIF@\T{list_to_tuple/1} BIF|(}

A tuple with the same elements as a given list is returned.

\TYPE

\T{list_to_tuple([term()]) -> tuple()}.

\EXITS

\T{list_to_tuple/1} exits with cause \T{badarg} if $\TZ{v}_1$ is not a list.

\EVALUATION

Let $\TZ{v}_1$ be a list with elements
$\TZ{T}_1$, \ldots, $\TZ{T}_k$.  A tuple
\T{\{$\Z{T}_1$,\tdots,$\Z{T}_k$\}} is returned.
The time for computing the answer should be $O(k)$.
\index{list_to_tuple/1 BIF@\T{list_to_tuple/1} BIF|)}

\subsection{\T{setelement/3}}

\index{setelement/3 BIF@\T{setelement/3} BIF|(}

A tuple is computed that differs from a given tuple in exactly one element.

\TYPE

\T{setelement(int(),tuple(),term()) -> tuple()}.

\EXITS

\T{setelement/3} exits with cause \T{badarg} if $\TZ{v}_1$ is not an integer or
$\TZ{v}_2$ is not a tuple. It also exits with cause
\ifOld \T{badarg} \fi \ifStd \T{\badindex} \fi
if the number represented by $\TZ{v}_1$ is out of range (see below).

\EVALUATION

Suppose that $i=\Er[\TZ{v}_1]$ and that $\TZ{v}_2$
with elements $\TZ{T}_1$, \ldots, $\TZ{T}_k$.
The evaluation depends on $i$ and $k$:
\begin{itemize}
\item If $i<1$ or $i>k$, then \T{setelement/3} exits with
cause \ifOld \T{badarg}\fi \ifStd \T{\badindex}\fi.
\item Otherwise, a tuple
\T{\{$\Z{T}_1$,\tdots,$\Z{T}_{i-1}$,$\TZ{v}_3$,$\Z{T}_{i+1}$,$\Z{T}_k$\}}
is returned.  (That is, a tuple that is exactly like $\TZ{v}_2$ except
that the element at position $i$ is $\TZ{v}_3$.)
\end{itemize}
The time for computing the answer should be $O(k)$.  (This is not a
destructive operation so the tuple given as argument must not be observably affected.)
\index{setelement/3 BIF@\T{setelement/3} BIF|)}

\subsection{\T{size/1}}

\label{section:size1}
\index{size/1 BIF@\T{size/1} BIF|(}

The number of elements of a binary or a tuple is returned.  \T{size/1}
is a guard BIF.

\TYPE

\T{size(bin()) -> int()} ; \\
\T{size(tuple()) -> int()}.

\EXITS

\T{size/1} exits with cause \T{badarg} if $\TZ{v}_1$ is neither a binary, nor a tuple.

\EVALUATION

The integer $\Re[k]$ is returned, where $k$ is the number of elements in the binary
or tuple $\TZ{v}_1$.
The time for computing the answer should be $O(1)$.
\index{size/1 BIF@\T{size/1} BIF|)}

\subsection{\T{tuple\char'137to\char'137list/1}}

\index{tuple_to_list/1 BIF@\T{tuple_to_list/1} BIF|(}

A list with the same elements as a given tuple is returned.

\TYPE

\T{tuple_to_list(tuple()) -> [term()]}.

\EXITS

\T{tuple_to_list/1} exits with cause \T{badarg} if $\TZ{v}_1$ is not a tuple.

\EVALUATION

Let $\TZ{v}_1$ be a tuple with elements
$\TZ{T}_1$, \ldots, $\TZ{T}_k$.  A list
\T{[$\Z{T}_1$,\tdots,$\Z{T}_k$]} is returned.
The time for computing the result should be $O(k)$.
\index{tuple_to_list/1 BIF@\T{tuple_to_list/1} BIF|)}
\index{tuple!BIFs|)}

\section{Builtin functions on lists and conses}

\label{section:list-bifs}
\index{list!BIFs|(}
\index{cons!BIFs|(}

Lists and conses are described in \S\ref{section:lists}.

\subsection{\T{hd/1}}

\index{hd/1 BIF@\T{hd/1} BIF|(}

The head of a cons, e.g., the first element of a list, is returned.
\T{hd/1} is a guard BIF.

\TYPE

If cons is used as intended, the type is
\begin{textdisplay}
\T{hd([T]) -> T}.
\end{textdisplay}
If cons is used as a general pairing operator, the type is instead
\begin{textdisplay}
\T{hd([T|_]) -> T}.
\end{textdisplay}

\EXITS

\T{hd/1} exits with cause \T{badarg} if $\TZ{v}_1$ is not a cons.

\EVALUATION

The head of the cons $\TZ{v}_1$ is returned.
The time for computing the result should be $O(1)$.
\index{hd/1 BIF@\T{hd/1} BIF|)}

\subsection{\T{length/1}}

\index{length/1 BIF@\T{length/1} BIF|(}

The number of elements of a list is returned.
\T{length/1} is a guard BIF.

\TYPE

\T{length([term()]) -> int()}.

\EXITS

\T{length/1} exits with cause \T{badarg} if $\TZ{v}_1$ is not a list.

\EVALUATION

The integer $\Re[k]$ is returned, where $k$ is the number of elements
in the list.
The time for computing the answer should be $O(1)$.
\index{length/1 BIF@\T{length/1} BIF|)}

\subsection{\T{tl/1}}

\index{tl/1 BIF@\T{tl/1} BIF|(}

The tail of a cons, e.g., all but the first element of a list, is
returned.  \T{tl/1} is a guard BIF.

\TYPE

If cons is used as intended, the type is
\begin{textdisplay}
\T{tl([T]) -> [T]}.
\end{textdisplay}
If cons is used as a general pairing operator, the type is instead
\begin{textdisplay}
\T{tl([_|T]) -> T}.
\end{textdisplay}

\EXITS

\T{tl/1} exits with cause \T{badarg} if $\TZ{v}_1$ is not a cons.

\EVALUATION

The tail of the cons $\TZ{v}_1$ is returned.
The time for computing the result should be $O(1)$.
\index{tl/1 BIF@\T{tl/1} BIF|)}
\index{list!BIFs|)}
\index{cons!BIFs|)}

\iffalse
% There are such functions in \StdErlang\ but in a separate module.
\section{Builtin functions on strings}

\label{section:string-bifs}

TO BE DECIDED.
\fi

\iffalse
% It could be that ord and chr will be provided in \OldErlang.
\section{Builtin functions on characters}

\label{section:char-bifs}

ord/1
chr/1

TO BE DECIDED.
\fi

\section{Builtin functions for modules}

\label{section:module-bifs}
\index{module!BIFs|(}

The syntax of modules is described in \S\ref{section:module-declarations}
and their dynamics in \S\ref{chapter:module-dynamics}.

The BIFs in this section are not designed to be used directly in applications.
Rather,
\ifStd
a \StdErlang\ implementation is expected to use them for providing a more
\fi
\ifOld
they are provided for implementing more
\fi
high-level interface to dynamic loading and replacement
of modules.\footnote{Cf.\ the \T{code} module of
OTP \cite[pp.~158--167]{otp-dev-ref}.}

\subsection{\T{erlang:check_process_code/2}}

\label{section:checkprocesscode2}
\index{erlang:check_process_code/2 BIF@\T{erlang:check_process_code/2} BIF|(}
\index{check_process_code/2 BIF@\T{check_process_code/2} BIF|(}

A check is made whether a particular process is using a given module
(\S\ref{section:process-using-module}).

\TYPE

\T{erlang:check_process_code(pid(),atom()) -> bool()}.

\EXITS

\T{erlang:check_process_code/2} exits with cause \T{badarg} if $\TZ{v}_1$ is not a pid
or $\TZ{v}_2$ is not an atom.\ifOld\footnote{\OldErlang\ actually allows $\TZ{v}_2$
to be anything and always returns \T{false} if $\TZ{v}_2$ is not an atom.}\fi

\EVALUATION

Let \TZ{N} be \T{node[$\TZ{v}_1$]}.
\begin{itemize}
\item If module_table[\Z{N}] contains a row with $\TZ{v}_2$ as key and \TZ{R} as value,
\T{old_version[\TZ{R}]} is not \T{none} and 
a reference to \T{old_version[\TZ{R}]} is found in \T{stack_trace[\Z{P}]},
then the BIF returns \T{true} (cf.\ \S\ref{section:checking-process-module}).
\item Otherwise, it returns \T{false}.
\end{itemize}
\index{erlang:check_process_code/2 BIF@\T{erlang:check_process_code/2} BIF|)}
\index{check_process_code/2 BIF@\T{check_process_code/2} BIF|)}

\subsection{\T{erlang:delete_module/1}}

\label{section:deletemodule1}
\index{erlang:delete_module/1 BIF@\T{erlang:delete_module/1} BIF|(}
\index{delete_module/1 BIF@\T{delete_module/1} BIF|(}

The current version of a module is changed to be the old version
(\S\ref{section:replacing-module}).

\TYPE

\T{erlang:delete_module(atom()) -> atom()}.

\EXITS

\T{erlang:delete_module/1} exits with cause \T{badarg} if $\TZ{v}_1$ is not
an atom.  Moreover, it will exit with cause \T{badarg} % bad choice!!!
if there is already an old version of the module (see below)

\EVALUATION

Let \TZ{P} be the process calling \T{delete_module/1} and
let \TZ{N} be \T{node[\Z{P}]}.
\begin{itemize}
% WEIRD!!! it should look for a current version!
\item If there is no row with key $\TZ{v}_1$ in \T{module_table[\Z{N}]},
then \T{undefined} is returned.\footnote{This is contrary to the reasonable intuition
that the BIF should return \T{undefined} if, and only if, \T{module_loaded/1}
returns \T{false} for the module.}
Otherwise, let \TZ{R} be the value for $\TZ{v}_1$ in \T{module_table[\Z{N}]}.
\item If \T{old_version[\Z{R}]} is not \T{none},
the BIF exits with cause \T{badarg}. % yeah, bad choice!!!
% I think this one should be checked first
\item Otherwise, if \T{current_version[\Z{R}]} is not \T{none},
it is made the old version of module $\TZ{v}_1$ as described in
\S\ref{section:making-old-version} and \T{true} is returned.
\item Otherwise, no action is taken and \T{true} is returned.
\end{itemize}
\index{erlang:delete_module/1 BIF@\T{erlang:delete_module/1} BIF|)}
\index{delete_module/1 BIF@\T{delete_module/1} BIF|)}

\subsection{\T{erlang:load_module/2}}

\label{section:loadmodule2}
\index{erlang:load_module/2 BIF@\T{erlang:load_module/2} BIF|(}
\index{load_module/2 BIF@\T{load_module/2} BIF|(}

A compiled module is loaded as the current version of the module
(\S\ref{section:loading}).
If there is already a current version of the module, it is made the
old version of the module.

\TYPE

\T{erlang:load_module(atom(),bin()) -> atom()}.

\EXITS

\T{erlang:load_module/2} exits with cause \T{badarg} if $\TZ{v}_1$ is not
an atom or $\TZ{v}_2$ is not a binary.

\EVALUATION

Let \TZ{P} be the process calling \T{load_module/2} and
let \TZ{N} be \T{node[\Z{P}]}.
\begin{itemize}
\item If the binary $\TZ{v}_2$ does not contain compiled code for a module
named $\TZ{v}_1$, the BIF returns a tuple \T{\{error, badfile\}}.
\item Otherwise, the following is done:

First, if \T{module_table[\Z{N}]} contains no row with $\TZ{v}_1$ as
key, a row is added with $\TZ{v}_1$ as key and \TZ{R} as value, such that
\T{old_version[\Z{R}]} and \T{current_version[\Z{R}]]} are both \T{none}.

Otherwise, let \TZ{R} be the value of the row with $\TZ{v}_1$ as key;
if neither of \T{old_version[\Z{R}]} and \T{current_version[\Z{R}]]}
is \T{none}, the BIF returns a tuple \T{\{error, not_purged\}}.

Next, if \T{current_version[\Z{R}]} is not \T{none},
it is made the old version of module $\TZ{v}_1$ on node \TZ{N} as described in
\S\ref{section:making-old-version}.

Finally, the binary $\TZ{v}_2$ is made the current version of module
$\TZ{v}_1$ on node \TZ{N} as described in
\S\ref{section:making-current-version}.
\end{itemize}
\index{erlang:load_module/2 BIF@\T{erlang:load_module/2} BIF|)}
\index{load_module/2 BIF@\T{load_module/2} BIF|)}

\subsection{\T{erlang:preloaded/0}}

\label{section:preloaded0}
\index{erlang:preloaded/0 BIF@\T{erlang:preloaded/0} BIF|(}
\index{preloaded/0 BIF@\T{preloaded/0} BIF|(}

A list is returned of the names of all modules that were loaded
as part of starting the current node.

\TYPE

\T{erlang:preloaded() -> [atom()]}.

\EXITS

\T{erlang:preloaded/0} always completes normally.

\EVALUATION

The BIF returns a list representing the value of
\T{preloaded[node[\Z{P}]]}, where \TZ{P} is the
process calling the BIF.
\index{erlang:preloaded/0 BIF@\T{erlang:preloaded/0} BIF|)}
\index{preloaded/0 BIF@\T{preloaded/0} BIF|)}

\subsection{\T{erlang:purge_module/1}}

\label{section:purgemodule1}
\index{erlang:purge_module/1 BIF@\T{erlang:purge_module/1} BIF|(}
\index{purge_module/1 BIF@\T{purge_module/1} BIF|(}

The old version of a module is purged (\S\ref{section:replacing-module}).

\TYPE

\T{erlang:purge_module(atom()) -> true}.

\EXITS

\T{erlang:purge_module/1} exits with cause \T{badarg} if $\TZ{v}_1$ is not
an atom.  Moreover, it will exit with cause \T{badarg} % bad choice!!!
if there is no old version of the module (see below)

\EVALUATION

Let \TZ{P} be the process calling \T{purge_module/1} and
let \TZ{N} be \T{node[\Z{P}]}.
\begin{itemize}
\item If there is a row in \T{module_table[\Z{N}]} with $\TZ{v}_1$ as key and 
some \TZ{R} as value, and \T{old_version[\TZ{R}]} is not \T{none}, then
\T{old_version[\TZ{R}]} is purged as described in
\S\ref{section:purging-old-version} and \T{true} is returned.
\item Otherwise, the BIF exits with cause \T{badarg}. % yuck, bad choice!!!
\end{itemize}
\index{erlang:purge_module/1 BIF@\T{erlang:purge_module/1} BIF|)}
\index{purge_module/1 BIF@\T{purge_module/1} BIF|)}

\subsection{\T{erlang:module_loaded/1}}

\label{section:moduleloaded1}
\index{erlang:module_loaded/1 BIF@\T{erlang:module_loaded/1} BIF|(}
\index{module_loaded/1 BIF@\T{module_loaded/1} BIF|(}

It is found out whether there is a current version (\S\ref{section:current-version}) of some module.

\TYPE

\T{erlang:module_loaded(atom()) -> bool()}.

\EXITS

\T{erlang:module_loaded/1} exits with cause \T{badarg} if $\TZ{v}_1$ is not
an atom.

\EVALUATION

Let \TZ{P} be the process calling \T{module_loaded/1} and
let \TZ{N} be \T{node[\Z{P}]}.
\begin{itemize}
\item If there is a row in \T{module_table[\Z{N}]} with $\TZ{v}_1$ as key and 
some \TZ{R} as value, and \T{current_version[\TZ{R}]} is not \T{none}, then
the BIF returns \T{true}.
\item Otherwise, the BIF returns \T{false}.
\end{itemize}
\index{erlang:module_loaded/1 BIF@\T{erlang:module_loaded/1} BIF|)}
\index{module_loaded/1 BIF@\T{module_loaded/1} BIF|)}
\index{module!BIFs|)}

\section{Builtin functions for functions and processes}

\label{section:process-bifs}
\index{function!BIFs|(}
\index{process!BIFs|(}

\subsection{\T{apply/2}}

\label{section:apply2}
\index{apply/2 BIF@\T{apply/2} BIF|(}
\index{function!application|(}

A given function is applied to a sequence of arguments.

\TYPE

\T{apply(\{atom(),atom()\},[term()]) -> term()} ; \\
\T{apply(function(),[term()]) -> term()}

\EXITS

\T{apply/2} exits with cause \T{badarg} if $\TZ{v}_1$ is neither a 2-tuple of atoms,
nor a function,
or $\TZ{v}_2$ is not a list. \T{apply/2} may also exit with the reasons described in
\S\ref{section:function-application}.
(In addition, the function being applied may complete abnormally
with any reason.)

\EVALUATION

The evaluation depends on the type of the first argument:
\begin{itemize}
\item If $\TZ{v}_1$ is a 2-tuple of atoms \TZ{Mod} and \TZ{Fun}, then
evaluation proceeds as described by case~\ref{item:explicit-mod-fun}
in \S\ref{section:function-application} with the atoms \TZ{Mod} and
\TZ{Fun} specifying the module name and function symbol, respectively, and
the list $\TZ{v}_2$ specifying the values of the arguments (and the arity).
\item If $\TZ{v}_1$ is a function, then evaluation proceeds as
described by case~\ref{item:function-application-impl-fun}
or~\ref{item:function-application-expl-fun}
in \S\ref{section:function-application} with $\TZ{v}_1$ being
the function and
the list $\TZ{v}_2$ specifying the values of the arguments.
\end{itemize}
\index{apply/2 BIF@\T{apply/2} BIF|)}
\index{function!application|)}

\subsection{\T{apply/3}}

\label{section:apply3}
\index{apply/3 BIF@\T{apply/3} BIF|(}
\index{function!application|(}

Given a module name, a function symbol and a list of arguments, a function
is looked up and applied to the arguments.

\TYPE

\T{apply(atom(),atom(),[term()]) -> term()}.

\EXITS

\T{apply/3} exits with cause \T{badarg} if $\TZ{v}_1$ is not an atom,
$\TZ{v}_2$ is not an atom, or $\TZ{v}_3$ is not a list.
\T{apply/3} may also exit with the reasons described in
\S\ref{section:function-application}.
(In addition, the function being applied may complete abnormally
with any reason.)

\EVALUATION

The evaluation of \T{apply/3} is described by case~\ref{item:explicit-mod-fun}
in \S\ref{section:function-application} with the atoms $\TZ{v}_1$ and
$\TZ{v}_2$ specifying the module name and function symbol, respectively, and
the list $\TZ{v}_3$ specifying the values of the arguments.
\index{apply/3 BIF@\T{apply/3} BIF|)}
\index{function!application|)}

\subsection{\T{exit/1}}

\label{section:exit1}
\index{exit/1 BIF@\T{exit/1} BIF|(}
\index{exit|(}

The BIF always exits with the argument as reason.  Unless the application is
governed by a
\ifOld \T{catch} expression (\S\ref{section:catch})\fi
\ifStd \T{try} expression (\S\ref{section:try-expr})\fi,
the process calling the BIF will exit.

\TYPE

\T{exit(term()) -> _}.

\EXITS

\T{exit/1} always exits, see below.

\EVALUATION

The evaluation of \T{exit/1} exits with reason $\TZ{v}_1$.
\index{exit/1 BIF@\T{exit/1} BIF|)}
\index{exit|)}

\subsection{\T{exit/2}}

\label{section:exit2}
\index{exit/2 BIF@\T{exit/2} BIF|(}
\index{exit!signal|(}

An exit signal with the given reason is sent to a process or port.
Reception of the exit signal will cause
the receiving process to complete abruptly\index{completion!abrupt}
unless it traps exit signals\index{exit!signal!trapping} or the
reason is the atom \T{normal}\index{normal@\T{normal}!exit signal},
cf.~\S\ref{section:receiving-exit-signal}.
If the reason is the atom \T{kill}\index{kill@\T{kill}!exit signal},
the receiving process will always complete abruptly.

\TYPE

\T{exit(pid(),term()) -> true} ; \\
\T{exit(port(),term()) -> true}.

\EXITS

\T{exit/2} exits with cause \T{badarg} if $\TZ{v}_1$ is neither a PID,
nor a port.

\EVALUATION

An exit signal with $\TZ{v}_2$ as reason is sent to the process or port
identified by $\TZ{v}_1$, as described in \S\ref{section:sending-exit-signal}.
The result is always the atom \T{true}.
\index{exit/2 BIF@\T{exit/2} BIF|)}
\index{exit!signal|)}

\subsection{\T{group_leader/0}}

\label{section:groupleader0}
\index{group_leader/0 BIF@\T{group_leader/0} BIF|(}
\index{process!group|(}

The BIF returns the group leader (\S\ref{section:group-leader}) of the calling process.

\TYPE

\T{group_leader() -> pid()}.

\EXITS

\T{group_leader/0} always completes normally.

\EVALUATION

The result is \T{group_leader[\Z{P}]}, where \TZ{P} is the process calling the
BIF.
\index{group_leader/0 BIF@\T{group_leader/0} BIF|)}
\index{process!group|)}

\subsection{\T{group_leader/2}}

\label{section:groupleader2}
\index{group_leader/2 BIF@\T{group_leader/2} BIF|(}
\index{process!group|(}

The BIF changes the group leader of a process.

\TYPE

\T{group_leader(pid(),pid()) -> true}.

\EXITS

\T{group_leader/2} exits with cause \T{badarg} if $\TZ{v}_1$ or $\TZ{v}_2$
is not a PID.

\EVALUATION

A group leader signal (\S\ref{section:signals}) is sent to $\TZ{v}_2$ with
$\TZ{v}_1$ as additional data.  When the signal is received by $\TZ{v}_2$,
\T{group_leader[$\Z{v}_2$]} will be set to $\TZ{v}_1$
(\S\ref{section:signal-arrival}).

The result is always the atom \T{true}.
\index{group_leader/2 BIF@\T{group_leader/2} BIF|)}
\index{process!group|)}

\subsection{\T{link/1}}

\label{section:link1}
\index{link/1 BIF@\T{link/1} BIF|(}
\index{process!linking|(}

A request to add a link between the process calling the BIF and some process
or port is submitted.

\TYPE

\T{link(pid()) -> true} ; \\
\T{link(port()) -> true}.

\EXITS

\T{link/1} exits with cause \T{badarg} if $\TZ{v}_1$ is not a PID or a port.

\EVALUATION

Let \TZ{P} be the process evaluating the application of \T{link/1}.
\begin{itemize}
\item If $\TZ{P}\neq\TZ{v}_1$ and $\TZ{v}_1$ is not
in \T{linked[\Z{P}]}, then $\TZ{v}_1$ is added to
\T{linked[\Z{P}]} and a \I{link} signal with \TZ{P} as sender
is dispatched to process $\TZ{v}_1$ (\S\ref{section:links}, \S\ref{section:signals}).
\item Otherwise, nothing is done.
\end{itemize}
The result is always the atom \T{true}.
\index{link/1 BIF@\T{link/1} BIF|)}
\index{process!linking|)}

\ifOld

\subsection{\T{list_to_pid/1}}

\label{section:listtopid1}
\index{list_to_pid/1 BIF@\T{list_to_pid/1} BIF|(}
\index{process!PID|(}

A list of characters is converted to a PID.

\TYPE

\T{list_to_pid([int()]) -> pid()}.

\EXITS

\T{list_to_pid/1} exits with cause \T{badarg} if $\TZ{v}_1$ is not a
list of characters that represents a PID.

\EVALUATION

The result of a BIF call of \T{list_to_pid/1} is a PID \TZ{P}
such that the value of \T{pid_to_list(\Z{P})}
(cf.\ \S\ref{section:pidtolist1}) on the same node
equals $\TZ{v}_1$.
\index{list_to_pid/1 BIF@\T{list_to_pid/1} BIF|)}
\index{process!PID|)}

\subsection{\T{pid_to_list/1}}

\label{section:pidtolist1}
\index{pid_to_list/1 BIF@\T{pid_to_list/1} BIF|(}
\index{process!PID|(}

A PID is converted to a list of characters.

\TYPE

\T{list_to_pid(pid()) -> [int()]}.

\EXITS

\T{list_to_pid/1} exits with cause \T{badarg} if $\TZ{v}_1$ is not a PID.

\EVALUATION

The result of a BIF call of \T{list_to_pid/1} is
\ifStd an implementation-defined \fi
\ifOld some \fi
list of characters.  It
\ifStd must hold \fi
\ifOld is guaranteed \fi
that for any PID \TZ{P}, the value
of an expression \T{list_to_pid(pid_to_list(\Z{P}))} equals \TZ{P}.
\index{pid_to_list/1 BIF@\T{pid_to_list/1} BIF|)}
\index{process!PID|)}

\fi % ifOld

\subsection{\T{process_flag/2}}

\label{section:processflag2}
\index{process_flag/2 BIF@\T{process_flag/2} BIF|(}
\index{process!flag|(}

The value of a process flag is read and updated.

\TYPE

\T{process_flag(trap_exit,bool()) -> bool()} ; \\
\T{process_flag(error_handler,atom()) -> atom()} ; \\
\T{process_flag(priority,atom()) -> atom()}.

\EXITS

\T{process_flag/2} exits with cause \T{badarg} if $\TZ{v}_1$ and $\TZ{v}_2$
is not one of the following combinations:
\begin{itemize}
\item The atom \T{trap_exit} and a Boolean atom.
\item The atom \T{error_handler} and a module name (i.e., an atom).
\item The atom \T{priority} and a priority atom, i.e., one of
\T{normal}, \T{high} and \T{low} \S\ref{section:scheduling}).
\end{itemize}

\EVALUATION

The action depends on $\TZ{v}_1$:
\begin{itemize}
\item If $\TZ{v}_1$ is \T{trap_exit}, then
\T{trap_exit[\Z{P}]} is set to $\TZ{v}_2$ and
the previous value of \T{trap_exit[\Z{P}]} is returned.
\item If $\TZ{v}_1$ is \T{error_handler}, then
\T{error_handler[\Z{P}]} is set to $\TZ{v}_2$ and
the previous value of \T{error_handler[\Z{P}]} is returned.
\item If $\TZ{v}_1$ is \T{priority}, then
\T{priority[\Z{P}]} is set to $\TZ{v}_2$ and
the previous value of \T{priority[\Z{P}]} is returned.
\end{itemize}
\index{process_flag/2 BIF@\T{process_flag/2} BIF|)}
\index{process!flag|)}

\subsection{\T{process_info/1}}

\label{section:processinfo1}
\index{process_info/1 BIF@\T{process_info/1} BIF|(}
\index{process!information|(}

Information about various properties of a process is returned.

\TYPE

\iftrue
\T{process_info(pid()) -> term()}.
\else
% This just isn't correct: the atom undefined may be returned.
\T{process_info(pid()) -> [\{atom(),term()\}]}.
\fi

\EXITS

\T{process_info/1} exits with cause \T{badarg} if $\TZ{v}_1$ is not a PID.

\EVALUATION

The BIF returns information about the process $\TZ{v}_1$.
\begin{itemize}
\item If that process is not alive, the result is the atom \T{undefined}.
\item Otherwise, the result is
a list of 2-tuples, each of which is the same as the result of an
application of the BIF \T{process_info/2} (\S\ref{section:processinfo2})
to $\TZ{v}_1$ and a distinct atom in the left column of Table~\ref{table:processinfo}.
The list should include all properties for which \T{process_info/2} gives meaningful
information.
\end{itemize}
\index{process_info/1 BIF@\T{process_info/1} BIF|)}
\index{process!information|)}

\subsection{\T{process_info/2}}

\label{section:processinfo2}
\index{process_info/2 BIF@\T{process_info/2} BIF|(}
\index{process!information|(}

Information about some property of a process is returned,
as described in Table~\ref{table:processinfo}.

\begin{table}[htb]
\begin{center}
\begin{tabular}{@{}ll@{}}
\hline
Second argument & Information returned about the process \\
\hline
\T{current_function} & The most recently entered function. \\ 
\T{dictionary} & The process dictionary. \\
\T{error_handler} & The error handler module. \\
\T{group_leader} & The group leader of the process. \\
\T{heap_size} & The current heap size. \\
\T{initial_call} & The initial application of the process. \\
\T{links} & The processes and ports to which the process is \\
& linked. \\
\T{memory} & The total amount of memory occupied by the \\
& process. \\
\T{message_queue_len} & The number of unprocessed messages in the queue. \\
\T{messages} & The unprocessed messages in the queue. \\
\T{priority} & The scheduling priority of the process. \\
\T{reductions} & The current number of reductions. \\
\T{registered_name} & The registered name of the process, if any. \\
\T{stack_size} & The current stack size. \\
\T{status} & The scheduling status: waiting, runnable or \\
& running. \\
\T{trap_exit} & Whether exit signals are trapped. \\
\hline
\end{tabular}
\caption{Alternatives for the BIF \T{process_info/2}.}
\label{table:processinfo}
\end{center}
\end{table}

\TYPE

\iftrue
\T{process_info(pid(),atom()) -> term()}.
\else
% This just isn't correct: the atom undefined may be returned, and for
% current_function we may also get the atom undefined instead of a 3-tuple.
\T{process_info(pid(),current_function) -> \{current_function,\{atom(),atom(),[term()]\}\}} ; \\
\T{process_info(pid(),dictionary) -> \{dictionary,[\{term(),term()\}]\}} ; \\
\T{process_info(pid(),error_handler) -> \{error_handler,atom()\}} ; \\
\T{process_info(pid(),group_leader) -> \{group_leader,pid()\}} ; \\
\T{process_info(pid(),heap_size) -> \{heap_size,int()\}} ; \\
\T{process_info(pid(),initial_call) -> \{initial_call,\{atom(),atom(),[term()]\}\}} ; \\
\T{process_info(pid(),links) -> \{links,[pid()]\}} ; \\
\T{process_info(pid(),memory) -> \{memory,int()\}} ; \\
\T{process_info(pid(),message_queue_len) -> \{message_queue_len,int()\}} ; \\
\T{process_info(pid(),messages) -> \{messages,[term()]\}} ; \\
\T{process_info(pid(),priority) -> \{priority,atom()\}} ; \\
\T{process_info(pid(),reductions) -> \{reductions,int()\}} ; \\
\T{process_info(pid(),registered_name) -> \{registered_name,atom()\}} ; \\
\T{process_info(pid(),stack_size) -> \{stack_size,int()\}} ; \\
\T{process_info(pid(),status) -> \{status,atom()\}} ; \\
\T{process_info(pid(),trap_exit) -> \{trap_exit,bool()\}}.
\fi

\EXITS

\T{process_info/2} exits with cause \T{badarg} if $\TZ{v}_1$ is not a PID
or $\TZ{v}_2$ is not one of the atoms in the left column of
Table~\ref{table:processinfo}.

\EVALUATION

\begin{itemize}
\item If process $\TZ{v}_1$ is not alive, the result is the atom \T{undefined}.
\item Otherwise, the BIF returns information about the process $\TZ{v}_1$ and
the result is always a 2-tuple where the first element is $\TZ{v}_2$:
\begin{itemize}
\item If $\TZ{v}_2$ is \T{current_function}, then
return \T{\{current_function,\linebreak[0]current_function[$\TZ{v}_1$]\}}.
\item If $\TZ{v}_2$ is \T{dictionary}, then
return \T{\{dictionary,\Z{Dict}\}}, where \TZ{Dict} is an
association list representing the contents of \T{dictionary[$\TZ{v}_1$]}
(cf.\ the BIF \T{get/0} [\S\ref{section:get0}]).
\item If $\TZ{v}_2$ is \T{error_handler}, then
return \T{\{error_handler,\linebreak[0]error_handler[$\TZ{v}_1$]\}}.
\item If $\TZ{v}_2$ is \T{group_leader}, then
return \T{\{group_leader,\linebreak[0]group_leader[$\TZ{v}_1$]\}}.
\item If $\TZ{v}_2$ is \T{heap_size}, then
return \T{\{heap_size,heap_size[$\TZ{v}_1$]\}}.
\item If $\TZ{v}_2$ is \T{initial_call}, then
return \T{\{initial_call,\linebreak[0]initial_call[$\TZ{v}_1$]\}}.
\item If $\TZ{v}_2$ is \T{links}, then
return \T{\{links,\Z{Lst}\}}, where \TZ{Lst} is a
list representing the value of \T{linked[$\TZ{v}_1$]}.
\item If $\TZ{v}_2$ is \T{memory}, then
return \T{\{memory,memory_in_use[$\TZ{v}_1$]\}}.
\item If $\TZ{v}_2$ is \T{message_queue_len}, then
return \T{\{message_queue_len,\linebreak[0]$\Re[l]$\}}, where $l$ is the
length of \T{message_queue[$\TZ{v}_1$]}.
\item If $\TZ{v}_2$ is \T{messages}, then
return \T{\{messages,\Z{Lst}\}}, where \TZ{Lst} is a
list representing the value of \T{message_queue[$\TZ{v}_1$]}
(i.e., a list of the messages in the queue, in the same order).
\item If $\TZ{v}_2$ is \T{priority}, then
return \T{\{priority,priority[$\TZ{v}_1$]\}}.
\item If $\TZ{v}_2$ is \T{reductions}, then
return \T{\{reductions,reductions[$\TZ{v}_1$]\}}.
\item If $\TZ{v}_2$ is \T{registered_name}, then
return \T{\{registered_name,\linebreak[0]registered_name[$\TZ{v}_1$]\}}.
\item If $\TZ{v}_2$ is \T{stack_size}, then
return \T{\{stack_size,$\Re[s]$\}}, where $s$ is a measure of
the size of \T{stack_trace[$\TZ{v}_1$]}.
\item If $\TZ{v}_2$ is \T{status}, then
return \T{\{status,status[$\TZ{v}_1$]\}}.
\item If $\TZ{v}_2$ is \T{trap_exit}, then
return \T{\{trap_exit,trap_exit[$\TZ{v}_1$]\}}.
\end{itemize}
\end{itemize}
Let \TZ{P} be the process calling the BIF.  The behaviour with respect
to signals (\S\ref{section:signals}) should be as if the result was
obtained by \TZ{P} sending an \I{info request} signal to process $\TZ{v}_1$
with $\TZ{v}_2$ as additional information, and process $\TZ{v}_1$ responding
with a message to \TZ{P} containing the result.
\index{process_info/2 BIF@\T{process_info/2} BIF|)}
\index{process!information|)}

\subsection{\T{processes/0}}

\index{processes/0 BIF@\T{processes/0} BIF|(}
See \S\ref{section:processes0}.
\index{processes/0 BIF@\T{processes/0} BIF|)}

\subsection{\T{self/0}}

\label{section:self0}
\index{self/0 BIF@\T{self/0} BIF|(}

The PID of the process calling the BIF is returned.
\T{self/0} is a guard BIF.

\TYPE

\T{self() -> pid()}.

\EXITS

\T{self/0} always completes normally.

\EVALUATION

The PID of the process calling the BIF is returned.
\index{self/0 BIF@\T{self/0} BIF|)}

\subsection{\T{spawn/3}}

\label{section:spawn3}
\index{spawn/3 BIF@\T{spawn/3} BIF|(}
\index{process!spawning a|(}

A new process is spawned on the same node.

\TYPE

\T{spawn(atom(),atom(),[term()]) -> pid()}.

\EXITS

\T{spawn/3} exits with cause \T{badarg} if $\TZ{v}_1$ or $\TZ{v}_2$ is not an atom,
or if $\TZ{v}_3$ is not a list.

\EVALUATION

Let the elements of $\TZ{v}_3$ be $\TZ{T}_1$, \ldots, $\TZ{T}_k$.
A new process is spawned on the same node as the process calling the BIF
(\S\ref{section:spawning-processes}).
The initial call of the process is \T{$\Z{v}_1$:$\Z{v}_2$($\Z{T}_1$,\tdots,$\Z{T}_k$)}.
The PID of the newly spawned process is returned.
\index{spawn/3 BIF@\T{spawn/3} BIF|)}
\index{process!spawning a|)}

\subsection{\T{spawn/4}}

\label{section:spawn4}
\index{spawn/4 BIF@\T{spawn/4} BIF|(}
\index{process!spawning a|(}

A new process is spawned on a particular node.

\TYPE

\T{spawn(atom(),atom(),atom(),[term()]) -> pid()}.

\EXITS

\T{spawn/4} exits with cause \T{badarg} if $\TZ{v}_1$, $\TZ{v}_2$ or $\TZ{v}_3$ is not an atom,
or if $\TZ{v}_4$ is not a list.

\EVALUATION

Let the elements of $\TZ{v}_4$ be $\TZ{T}_1$, \ldots, $\TZ{T}_k$.
A new process is spawned on node $\TZ{v}_1$ (\S\ref{section:spawning-processes}).
The initial call of the process is \T{$\Z{v}_2$:$\Z{v}_3$($\Z{T}_1$,\tdots,$\Z{T}_k$)}.
The PID of the newly spawned process is returned.
\index{spawn/4 BIF@\T{spawn/4} BIF|)}
\index{process!spawning a|)}

\subsection{\T{spawn_link/3}}

\label{section:spawnlink3}
\index{spawn_link/3 BIF@\T{spawn_link/3} BIF|(}
\index{process!spawning a|(}

A new process, initially linked to its creator, is spawned on the same node.

\TYPE

\T{spawn_link(atom(),atom(),[term()]) -> pid()}.

\EXITS

\T{spawn_link/3} exits with cause \T{badarg} if $\TZ{v}_1$ or $\TZ{v}_2$ is not an atom,
or if $\TZ{v}_3$ is not a list.

\EVALUATION

The BIF does exactly the same thing as \T{spawn/3} (\S\ref{section:spawn3}),
except that when the BIF returns, the newly spawned process is linked with
the process calling \T{spawn_link/3}.
\index{spawn_link/3 BIF@\T{spawn_link/3} BIF|)}
\index{process!spawning a|)}

\subsection{\T{spawn_link/4}}

\label{section:spawnlink4}
\index{spawn_link/4 BIF@\T{spawn_link/4} BIF|(}
\index{process!spawning a|(}

A new process, initially linked to its creator, is spawned on a particular node.

\TYPE

\T{spawn_link(atom(),atom(),atom(),[term()]) -> pid()}.

\EXITS

\T{spawn/4} exits with cause \T{badarg} if $\TZ{v}_1$, $\TZ{v}_2$ or $\TZ{v}_3$ is not an atom,
or if $\TZ{v}_4$ is not a list.

\EVALUATION

The BIF does exactly the same thing as \T{spawn/4} (\S\ref{section:spawn4}),
except that when the BIF returns, the newly spawned process is linked with
the process calling \T{spawn_link/4}.
\index{spawn_link/4 BIF@\T{spawn_link/4} BIF|)}
\index{process!spawning a|)}

\subsection{\T{unlink/1}}

\label{section:unlink1}
\index{unlink/1 BIF@\T{unlink/1} BIF|(}
\index{process!linking|(}

A request to remove any link between the process calling the BIF and
some process or port is submitted.

\TYPE

\T{unlink(pid()) -> true} ; \\
\T{unlink(port()) -> true}.

\EXITS

\T{unlink/1} exits with cause \T{badarg} if $\TZ{v}_1$ is not a PID or a port.

\EVALUATION

Let \TZ{P} be the process evaluating the application of \T{unlink/1}.
\begin{itemize}
\item If $\TZ{P}\neq\TZ{v}_1$ and $\TZ{v}_1$ is
in \T{linked[\Z{P}]}, then $\TZ{v}_1$ is removed from
\T{linked[\Z{P}]} and an \I{unlink} signal with \TZ{P} as sender
is dispatched to process $\TZ{v}_1$.
\item Otherwise, nothing is done.
\end{itemize}
The result is always the atom \T{true}.
\index{unlink/1 BIF@\T{unlink/1} BIF|)}
\index{process!linking|)}
\index{function!BIFs|)}
\index{process!BIFs|)}

\section{Builtin functions for process dictionaries}

\label{section:dictionary-bifs}
\index{process!dictionary!BIFs|(}

As described in \S\ref{section:process-state-dynamic},
each process \TZ{P} has associated with it a table
\T{dictionary[\Z{P}]} (\S\ref{section:tables}).

\subsection{\T{erase/0}}
\index{erase/0 BIF@\T{erase/0} BIF|(}

The process calling the BIF has every row of its dictionary removed
but a representation of the previous contents is returned.

\TYPE

\T{erase() -> [\{term(),term()\}]}.

\EXITS

\T{erase/0} always completes normally.

\EVALUATION

Let $d$ be the value of \T{dictionary[\Z{P}]}, where \TZ{P} is the
process calling \T{erase/0} and let \TZ{lst} be an association list
representing the contents of $d$ (\S\ref{section:tables}).
The effect of the call is to remove every
row of $d$ and the result is \TZ{lst}.

The operation must be atomic.

\iffalse
The time for producing the effect and computing the answer should
be $O(\mathit{size}(d))$.
\fi
\index{erase/0 BIF@\T{erase/0} BIF|)}

\subsection{\T{erase/1}}
\index{erase/1 BIF@\T{erase/1} BIF|(}

In the dictionary of the process calling the BIF, the value recorded
for a certain key (if any) is erased, but the previously recorded
value (or \T{undefined}) is returned.

\TYPE

\T{erase(term()) -> term()}.

\EXITS

\T{erase/1} always completes normally.

\EVALUATION

Let $d$ be the value of \T{dictionary[\Z{P}]}, where \TZ{P} is the
process calling \T{erase/1}.
There are two cases depending on whether $\TZ{v}_1$ is a key in $d$ or not:
\begin{itemize}
\item If $d$ contains a row with $\TZ{v}_1$ as key and some term $\TZ{T}_1$ as
value, then that row is removed from $d$ and \TZ{t} is $\TZ{t}_1$.
\item Otherwise, $d$ is unchanged and \TZ{t} is the atom \T{undefined}.
\end{itemize}
The result is \TZ{t}.

The operation must be atomic.

\iffalse
The time for producing the effect and computing the answer should
be $O(\mathit{size}(\mathit{keys}(d))+size(\TZ{t}))$.
\fi
\index{erase/1 BIF@\T{erase/1} BIF|)}

\subsection{\T{get/0}}

\label{section:get0}
\index{get/0 BIF@\T{get/0} BIF|(}

A representation of the dictionary of
the process calling the BIF is returned.

\TYPE

\T{get() -> [\{term(),term()\}]}.

\EXITS

\T{get/0} always completes normally.

\EVALUATION

Let $d$ be the value of \T{dictionary[\Z{P}]}, where \TZ{P} is the
process calling \T{get/0}.
The result of calling \T{get/0} is a list representing $d$
(\S\ref{section:tables}).

\iffalse
The time for computing the answer should
be $O(\mathit{size}(d))$.
\fi
\index{get/0 BIF@\T{get/0} BIF|)}

\subsection{\T{get/1}}

\index{get/1 BIF@\T{get/1} BIF|(}

The value recorded for a certain
key in the dictionary of the process calling the BIF is returned
(or \T{undefined} is returned if there is no value recorded).

\TYPE

\T{get(term()) -> term()}.

\EXITS

\T{get/1} always completes normally.

\EVALUATION

Let $d$ be the value of \T{dictionary[\Z{P}]}, where \TZ{P} is the
process calling \T{get/1}.
There are two cases depending on whether $\TZ{v}_1$ is a key in $d$ or not:
\begin{itemize}
\item If $d$ contains a row with $\TZ{v}_1$ as key and some term $\TZ{T}_1$ as
value, then \TZ{t} is $\TZ{t}_1$.
\item Otherwise, \TZ{t} is  the atom \T{undefined}.
\end{itemize}
The result is \TZ{t}.

\iffalse
The time for computing the answer cannot be expected to be less than
$O(\mathit{size}(\mathit{keys}(d))+size(\TZ{t}))$.
\fi
\index{get/1 BIF@\T{get/1} BIF|)}

\subsection{\T{get\char'137keys/1}}

\index{get_keys/1 BIF@\T{get_keys/1} BIF|(}

A list of all keys in the dictionary of the process calling the BIF
that have a certain value is returned.

\TYPE

\T{get_keys(term()) -> [term()]}.

\EXITS

\T{get_keys/1} always completes normally.

\EVALUATION

Let $d$ be the value of \T{dictionary[\Z{P}]}, where \TZ{P} is the
process calling \T{get/0}.
The result of calling \T{get_keys/1} is a list without duplicates
that contains each key
of $d$ for which the value is $\TZ{v}_1$.  The elements
of the resulting list may be in any order.

\iffalse
The time for computing the answer should be
$O(\mathit{size}(\mathit{keys}(d)))$.
\fi
\index{get_keys/1 BIF@\T{get_keys/1} BIF|)}

\subsection{\T{put/2}}

\index{put/2 BIF@\T{put/2} BIF|(}

In the dictionary of the process calling the BIF, a value
is set for some key, replacing any previous value, which is returned
(or \T{undefined} is returned if there was no value recorded previously).

\TYPE

\T{put(term(),term()) -> term()}.

\EXITS

\T{put/2} always completes normally.

\EVALUATION

Let $d$ be the value of \T{dictionary[\Z{P}]}, where \TZ{P} is the
process calling \T{put/2}.
There are two cases depending on whether $\TZ{v}_1$ is a key in $d$ or not:
\begin{itemize}
\item If $d$ contains a row with $\TZ{v}_1$ as key and some term $\TZ{t}_1$ as value,
then that row is replaced with one having $\TZ{v}_1$ as key and
$\TZ{v}_2$ as value; \TZ{t} is $\TZ{t}_1$.
\item Otherwise, a row with key $\TZ{v}_1$ and value $\TZ{v}_2$ is added
to $d$ and \TZ{t} is \T{undefined}.
\end{itemize}
The result is \TZ{t}.

The operation must be atomic.

\iffalse
The time for producing the effect and computing the answer should
be $O(\mathit{size}(\mathit{keys}(d))+size(\TZ{t})+size(\TZ{v}_2))$.
\fi
\index{put/2 BIF@\T{put/2} BIF|)}
\index{process!dictionary!BIFs|)}

\section{Builtin functions for nodes}

% Have a look at all these when we are an isolated node!!!

\label{section:node-bifs}
\index{node!BIFs|(}

\iffalse
% Not part of the specification (or even \OldErlang).
\subsection{\T{alive/2}}

\label{section:alive2}
\index{erlang:alive/2 BIF@\T{erlang:alive/2} BIF|(}
\index{alive/2 BIF@\T{alive/2} BIF|(}
\index{node!communicating|(}

% ASKED VARIOUS QUESTIONS BY EMAIL 980524:
\iffalse
Hej!

BIFen alive/2 �r lite b�kig f�r mig att testa s� jag undrar:

1. Vad returnerar den?

2. Vad h�nder om den anropas p� en redan kommunicerande nod?

3. Vad h�nder om den f�rs�ker ange ett namn som redan anv�nds?

Mvh,

-- Jonas
\fi

Calling this BIF has as effect to make the node on which the calling process
resides a communicating node, provided that certain preconditions are met.

\TYPE

\T{alive(atom(),port()) -> ???}.

\EXITS

\T{alive/2} exits with cause \T{badarg} if $\TZ{v}_1$ is
not an atom or if $\TZ{v}_2$ is not a port.  \T{register/2} may also
exit with cause \T{badarg} % really lousy choice...
if certain preconditions are not met (see below).

\EVALUATION

Let \TZ{P} be the process calling \T{alive/2}.
\begin{itemize}
\item If there is no process registered under the name \T{net_kernel}
on \T{node[\Z{P}]}, then the BIF exits with cause \T{badarg}.  % yuck
\item Something about EPMD???
\item Otherwise, the node becomes a communicating node, as described in
\S\ref{section:single-multi-node} and \S\ref{section:registering-nodes}.
\end{itemize}
The result is ???.
\index{erlang:alive/2 BIF@\T{erlang:alive/2} BIF|)}
\index{alive/2 BIF@\T{alive/2} BIF|)}
\index{node!communicating|)}
\fi % iffalse

\subsection{\T{erlang:disconnect_node/1}}

\label{section:disconnectnode1}
\index{erlang:disconnect_node/1 BIF@\T{erlang:disconnect_node/1} BIF|(}
\index{disconnect_node/1 BIF@\T{disconnect_node/1} BIF|(}

Friendship is terminated with a given node.

\TYPE

\T{disconnect_node(term()) -> atom()}. % yuck, should be atom() -> bool().

\EXITS

\T{disconnect_node/1} always completes normally.

\EVALUATION

Let \TZ{P} be the process calling \T{disconnect_node/1}.
\begin{itemize}
\item If \T{node[\Z{P}]} is not communicating, then \T{ignored} is returned.
\item Otherwise, if $\TZ{v}_1$ is not an atom or $\TZ{v}_1$ is not in
\T{friends[node[\Z{P}]]}, \T{false} is returned.
\item Otherwise the same things happen as when node \T{node[\Z{P}]} has
lost contact with node $\TZ{v}_1$, as described in \S\ref{section:friendship}
and \S\ref{section:monitor-node}, and then \T{true} is returned.
\end{itemize}
\index{erlang:disconnect_node/1 BIF@\T{erlang:disconnect_node/1} BIF|)}
\index{disconnect_node/1 BIF@\T{disconnect_node/1} BIF|)}

\subsection{\T{erlang:get_cookie/0}}

\label{section:getcookie0}
\index{erlang:get_cookie/0 BIF@\T{erlang:get_cookie/0} BIF|(}
\index{get_cookie/0 BIF@\T{get_cookie/0} BIF|(}
\index{node!magic cookie|(}

The magic cookie of the current node is returned.

\TYPE

\T{get_cookie() -> atom()}.

\EXITS

\T{get_cookie/0} always completes normally.

\EVALUATION

The BIF returns \T{magic_cookie[node[\Z{P}]]}, where \TZ{P} is the process
calling \T{get_cookie/0}.
\index{erlang:get_cookie/0 BIF@\T{erlang:get_cookie/0} BIF|)}
\index{get_cookie/0 BIF@\T{get_cookie/0} BIF|)}
\index{node!magic cookie|)}

\subsection{\T{erlang:halt/0}}

\label{section:halt0}
\index{halt/0 BIF@\T{halt/0} BIF|(}
\index{node!termination|(}

The current node is terminated.

\TYPE

\T{halt() -> _}.

\EXITS

\T{halt/0} never completes abnormally.

\EVALUATION

The node \T{node[\Z{P}]}, where \TZ{P} is the process
calling \T{halt/0}, is terminated immediately.  The BIF thus never returns.
\index{node!termination|)}
\index{halt/0 BIF@\T{halt/0} BIF|)}

\subsection{\T{is_alive/0}}

\label{section:isalive0}
\index{is_alive/0 BIF@\T{is_alive/0} BIF|(}
\index{node!communicating|(}

The BIF tests whether the current node is communicating or not.

\TYPE

\T{is_alive() -> bool()}.

\EXITS

\T{is_alive/0} always completes normally.

\EVALUATION

The BIF returns the value of \T{communicating[node[\Z{P}]]},
where \TZ{P} is the process calling the BIF.
\index{is_alive/0 BIF@\T{is_alive/0} BIF|)}
\index{node!communicating|)}

\subsection{\T{monitor_node/2}}

\label{section:monitornode2}
\index{monitor_node/2 BIF@\T{monitor_node/2} BIF|(}
\index{node!monitoring|(}

Calling this BIF has as effect to increase or decrease the number of messages
that the current process will receive as notification that friendship between
the current node and a certain (other) node has ceased.

\TYPE

\T{monitor_node(atom(),bool()) -> true}.

\EXITS

\T{monitor_node/2} exits with cause \T{badarg} if $\TZ{v}_1$ is
not an atom or if $\TZ{v}_2$ is not a Boolean atom.  On an isolated node,
it will also exit with cause \T{badarg} % baaaad
if the node to be monitored is not the current node.

\EVALUATION

Let \TZ{P} be the process calling \T{monitor_node/2}.
\begin{itemize}
\item If \T{alive[node[\Z{P}]]} is \T{false} and $\TZ{v}_1$ is not
\T{node[\Z{P}]}, then the BIF exits with cause \T{badarg}. % questionable
\item If $\TZ{v}_1$ is \T{node[\Z{P}]}, do nothing.
\item If $\TZ{v}_1$ is \T{true}, do the following:
\begin{enumerate}
\item If there is not already a row in \T{monitored_nodes[node[\Z{P}]]}
with $\TZ{v}_1$ as key, then add one with an empty table as value.
\item Let $t$ be the value for $\TZ{v}_1$ in \T{monitored_nodes[node[\Z{P}]]}.
If there is no row with \TZ{P} as key in that table, add such a row with value 1.
Otherwise, add 1 to the value for \TZ{P} in $t$.
\end{enumerate}
\item Otherwise ($\TZ{v}_1$ is \T{false}):
\begin{itemize}
\item If there is no row with $\TZ{v}_1$ as key in
\T{monitored_nodes[node[\Z{P}]]}, then do nothing.
\item Otherwise, let $t$ be the value for $\TZ{v}_1$.
If there is no entry for \TZ{P} in $t$, do nothing.
\item Otherwise, if the row for \TZ{P} in $t$ has value 1, then remove the
row for \TZ{P}.
\item Otherwise, subtract 1 from the value for \TZ{P} in $t$.
\end{itemize}
\end{itemize}
The result is always \T{true}.
\index{monitor_node/2 BIF@\T{monitor_node/2} BIF|)}
\index{node!monitoring|)}

\subsection{\T{node/0}}

\label{section:node0}
\index{node/0 BIF@\T{node/0} BIF|(}

This BIF returns (the name of) the node on which the calling process
resides.  \T{node/0} is a guard BIF.

\TYPE

\T{node() -> atom()}.

\EXITS

\T{node/0} always completes normally.

\EVALUATION

The result is \T{node[\Z{P}]}, where \TZ{P} is the
process calling \T{node/0} (\S\ref{section:process-state-static}).
\index{node/0 BIF@\T{node/0} BIF|)}

\subsection{\T{node/1}}

\label{section:node1}
\index{node/1 BIF@\T{node/1} BIF|(}

This BIF returns (the name of) the node on which a given ref, PID or
port was created.  \T{node/1} is a guard BIF.

\TYPE

\T{node(ref()) -> atom()} ; \\
\T{node(pid()) -> atom()} ; \\
\T{node(port()) -> atom()}.

\EXITS

\T{node/1} exits with cause \T{badarg} if $\TZ{v}_1$ is
not a ref, nor a PID or a port.

\EVALUATION

The result is \T{node[$\Z{v}_1$]} (\S\ref{section:process-state-static},
\S\ref{section:port-state-static}).
\index{node/1 BIF@\T{node/1} BIF|)}

\subsection{\T{nodes/0}}

\label{section:nodes0}
\index{nodes/0 BIF@\T{nodes/0} BIF|(}

A list is returned of all friends
of the node on which the BIF is called.

\TYPE

\T{nodes() -> [atom()]}.

\EXITS

\T{nodes/0} always completes normally.

\EVALUATION

The BIF returns a list representing the value of
\T{friends[node[\Z{P}]]}, where \TZ{P} is the
process calling the BIF.
\index{nodes/0 BIF@\T{nodes/0} BIF|)}

\subsection{\T{processes/0}}

\ifStd \label{section:node:processes0} \fi
\ifOld \label{section:processes0} \fi
\index{processes/0 BIF@\T{processes/0} BIF|(}

A list is returned of the PIDs of all live processes
on the node on which the BIF is called.

\TYPE

\T{processes() -> [pid()]}.

\EXITS

\T{processes/0} always completes normally.

\EVALUATION

The BIF returns a list representing the value of \T{processes[node[\Z{P}]]},
where \TZ{P} is the process calling the BIF.
\index{processes/0 BIF@\T{processes/0} BIF|)}

\subsection{\T{erlang:set_cookie/2}}

\label{section:setcookie2}
\index{erlang:set_cookie/2 BIF@\T{erlang:set_cookie/2} BIF|(}
\index{set_cookie/2 BIF@\T{set_cookie/2} BIF|(}
\index{node!magic cookie|(}

The magic cookie of the current node and/or the presumed magic cookie of
another node is set.

\TYPE

\T{set_cookie(atom(),atom()) -> bool()}.

\EXITS

\T{set_cookie/2} exits with cause \T{badarg} if $\TZ{v}_1$ or $\TZ{v}_2$ is
not an atom, or if the atom $\TZ{v}_1$ could not be the name of a node (i.e.,
it does not contain exactly one `\T{@}' character,
cf.\ \S\ref{section:single-multi-node}).

\EVALUATION

There are two cases, depending on the value of $\TZ{v}_1$.  Let \TZ{P} be
the process calling \T{set_cookie/2} and let \TZ{N} be \T{node[\Z{P}]}.
\begin{itemize}
\item If $\TZ{v}_1$ equals \TZ{N}, then
\begin{enumerate}
\item Set \T{magic_cookie[\Z{N}]} to $\TZ{v}_2$.
\item For each pair in \T{magic_cookies[\Z{N}]} where the presumed magic cookie
is \T{nocookie}, change the presumed magic cookie to $\TZ{v}_2$.
\end{enumerate}
\item If $\TZ{v}_1$ does not equal \TZ{N}, then
delete any pair for $\TZ{v}_1$ in \T{magic_cookies[\Z{N}]} and add a pair
$(\TZ{v}_1,\TZ{v}_2)$ to \T{magic_cookies[\Z{N}]}.
\end{itemize}

\index{erlang:set_cookie/2 BIF@\T{erlang:set_cookie/2} BIF|)}
\index{set_cookie/2 BIF@\T{set_cookie/2} BIF|)}
\index{node!magic cookie|)}

\subsection{\T{set_node/2}}

\label{section:setnode2}
\index{erlang:set_node/2 BIF@\T{erlang:set_node/2} BIF|(}
\index{set_node/2 BIF@\T{set_node/2} BIF|(}
\index{node!communicating|(}

Calling this BIF has as effect to make the node on which the calling process
resides a communicating node, provided that certain preconditions are met.

\TYPE

\T{set_node(???,???) -> ???}.

\EXITS

???
\iffalse
\T{alive/2} exits with cause \T{badarg} if $\TZ{v}_1$ is
not an atom or if $\TZ{v}_2$ is not a port.  \T{register/2} may also
exit with cause \T{badarg} % really lousy choice...
if certain preconditions are not met (see below).
\fi

\EVALUATION

???
\iffalse
Let \TZ{P} be the process calling \T{alive/2}.
\begin{itemize}
\item If there is no process registered under the name \T{net_kernel}
on \T{node[\Z{P}]}, then the BIF exits with cause \T{badarg}.  % yuck
\item Something about EPMD???
\item Otherwise, the node becomes a communicating node, as described in
\S\ref{section:single-multi-node} and \S\ref{section:registering-nodes}.
\end{itemize}
The result is ???.
\fi
\index{erlang:set_node/2 BIF@\T{erlang:set_node/2} BIF|)}
\index{set_node/2 BIF@\T{set_node/2} BIF|)}

\subsection{\T{erlang:set_node/3}}

\label{section:setnode3}
\index{erlang:set_node/3 BIF@\T{erlang:set_node/3} BIF|(}
\index{set_node/3 BIF@\T{set_node/3} BIF|(}

Ugga mugga.

\TYPE

\T{set_node(???,???,???) -> ???}.

\EXITS

\T{set_node/3} \ldots

\EVALUATION

Yumm yumm!
\index{erlang:set_node/3 BIF@\T{erlang:set_node/3} BIF|)}
\index{set_node/3 BIF@\T{set_node/3} BIF|)}

\subsection{\T{statistics/1}}

\label{section:statistics1}
\index{statistics/1 BIF@\T{statistics/1} BIF|(}

Information about the current state of the node is returned.

\TYPE

\ifOld
\T{statistics(garbage_collection) -> \{int(),int(),int()\}} \\
\fi
\T{statistics(reductions) -> \{int(),int()\}} \\
\T{statistics(runtime) -> \{int(),int()\}} \\
\T{statistics(run_queue) -> int()} \\
\T{statistics(wall_clock) -> \{int(),int()\}}.

\EXITS

\T{statistics/1} exits with cause \T{badarg} if $\TZ{v}_1$ is not an atom
or if it is not one of the atoms \T{runtime}, \T{wall_clock} or
\T{reductions}.
\ifStd
An implementation may extend \T{statistics/1} to
accept additional atoms as argument.
\fi

\EVALUATION

Let \TZ{N} be the node on which the BIF is called.
The evaluation depends on $\TZ{v}_1$:
\begin{itemize}
\item \T{garbage_collection}: return the result of \T{current_gc[\Z{N}]}
(\S\ref{section:node-state-dynamic}), i.e., a 3-tuple
\T{\{\Z{NumberOfGCs},\Z{WordsReclaimed},0\}} of integers where \TZ{NumberOfGCs} is the
total number of garbage collection operations
that have been carried out by processes on the node and \TZ{WordsReclaimed} is
the total number of memory words reclaimed by such operations, and the third integer
is always 0.

\item \T{reductions}: return the result of \T{current_reductions[\Z{N}]}
(\S\ref{section:node-state-dynamic}), i.e.,
a 2-tuple \T{\{\Z{TotalReductions},\Z{RecentReductions}\}} where \TZ{TotalReductions}
is an \Erlang\ integer representing the number of function calls made
on the node and \Z{RecentReductions} is the same but
only since the last call \T{statistics(reductions)}.

\item \T{run_queue}: return $\Re[i]$, where $i$ is the number
of processes on the node that have \T{runnable} or \T{running} status.

\item \T{runtime}: return the result of \T{current_runtime[\Z{N}]}
(\S\ref{section:node-state-dynamic}), i.e.,
a 2-tuple \T{\{\Z{TotalRuntime},\Z{RecentRuntime}\}} where \TZ{TotalRuntime}
is an \Erlang\ integer representing the total time spent
running processes on the node and \Z{RecentRunTime} is the same but
only since the last call \T{statistics(runtime)}.

\item \T{wall_clock}: return the result of \T{current_wall_clock[\Z{N}]}
(\S\ref{section:node-state-dynamic}), i.e.,
a 2-tuple \T{\{\Z{TotalWallClock},\Z{RecentWallClock}\}} where \TZ{TotalWallClock}
is an \Erlang\ integer representing the time which has passed since the
node was started and \Z{RecentWallClock} is the same but
only since the last call \T{statistics(wall_clock)}.
\end{itemize}
% garbage_collection omitted.
\index{statistics/1 BIF@\T{statistics/1} BIF|)}
\index{node!communicating|)}
\index{node!BIFs|)}

\section{Builtin functions for process registries}

\label{section:process-registry-bifs}
\index{process!registry!BIFs|(}

The process registry of a node is described in \S\ref{section:process-registry}.

\subsection{\T{register/2}}

\label{section:register2}
\index{register/2 BIF@\T{register/2} BIF|(}

A name is registered for a process on a node.

\TYPE

\T{register(atom(),pid()) -> true}.

\EXITS

\T{register/2} exits with cause \T{badarg} if $\TZ{v}_1$ is
not an atom or if $\TZ{v}_2$ is not a PID.  \T{register/2} may also
exit with cause
\ifOld \T{badarg} \fi
\ifStd \T{registry} \fi
if some process is already registered
under the name, there is already a name registered for the process,
the process resides on a different node than the one on which the BIF
is called, or the process has completed (see below).

\EVALUATION

Let \TZ{N} be the node on which the BIF is called.

\begin{itemize}
\item
If \T{node[$\Z{v}_2$]} is not \TZ{N}
or \T{registry[\Z{N}]} already contains a process with name $\TZ{v}_1$
or \T{registry[\Z{N}]} already contains a name for process $\TZ{v}_2$,
or process $\Z{v}_2$ has completed,
then the BIF exits with cause \ifStd\T{registry}\fi\ifOld\T{badarg}\fi.
\item
Otherwise, the name $\TZ{v}_1$ is added for $\TZ{v}_2$ in \T{registry[\Z{N}]}.
The BIF always returns the atom \T{true}.
\end{itemize}
The operation should be atomic to ensure the integrity of the registry.
\index{register/2 BIF@\T{register/2} BIF|)}

\subsection{\T{registered/0}}

\label{section:registered0}
\index{registered/0 BIF@\T{registered/0} BIF|(}

A list of all names in the registry on the current node is returned.

\TYPE

\T{registered() -> [atom()]}.

\EXITS

\T{registered/0} always completes normally.

\EVALUATION

Let \TZ{N} be the node on which the BIF is called.
A list without duplicates of all names in \T{registry[\Z{N}]} is returned.
The order of the atoms in the list is not defined.
The operation should be atomic to ensure the integrity of the registry.
\index{registered/0 BIF@\T{registered/0} BIF|)}

\subsection{\T{unregister/1}}

\label{section:unregister1}
\index{unregister/1 BIF@\T{unregister/1} BIF|(}

Any registration for a name is removed.

\TYPE

\T{unregister(atom()) -> true}.

\EXITS

\T{unregister/1} exits with cause \T{badarg} if $\TZ{v}_1$ is
not an atom.

\EVALUATION

Let \TZ{N} be the node on which the BIF is called.
\begin{itemize}
\item If \T{registry[\Z{N}]} does not contains the name $\TZ{v}_1$,
then the BIF has no effect.
\item Otherwise, the association for the name $\TZ{v}_1$ is removed
from \T{registry[\Z{N}]}.
\end{itemize}
In either case, the atom \T{true} is returned.
The operation should be atomic to ensure the integrity of the registry.
\index{unregister/1 BIF@\T{unregister/1} BIF|)}

\subsection{\T{whereis/1}}

\label{section:whereis1}
\index{whereis/1 BIF@\T{whereis/1} BIF|(}

The PID of a process registered under the given name is returned, if any.

\TYPE

\T{whereis(atom()) -> term()}.

\EXITS

\T{whereis/1} exits with cause \T{badarg} if $\TZ{v}_1$ is
not an atom.

\EVALUATION

Let \TZ{N} be the node on which the BIF is called.
\begin{itemize}
\item If \T{registry[\Z{N}]} does not contains the name $\TZ{v}_1$,
then the BIF returns the atom \T{undefined}.
\item Otherwise, the PID that is the value for $\TZ{v}_1$ in
\T{registry[\Z{N}]} is returned.
\end{itemize}
The operation should be atomic to ensure the integrity of the registry.
\index{whereis/1 BIF@\T{whereis/1} BIF|)}
\index{process!registry!BIFs|)}

\section{Builtin functions for I/O and ports}

\label{section:port-bifs}
\index{port!BIFs|(}

\subsection{\T{open_port/2}}

\label{section:openport2}
\index{open_port/2 BIF@\T{open_port/2} BIF|(}
\index{port!opening|(}

A new port is opened to a recently opened driver or
recently spawned external process.

\TYPE

\T{open_port(term(),[term()]) -> port()}.

\EXITS

\T{open_port/2} exits with cause \T{badarg} if $\TZ{v}_1$ is not one
of the permitted alternatives or $\TZ{v}_2$ contains an invalid
option (\S\ref{section:opening-ports}).

\EVALUATION

The BIF \T{open_port/2} is described fully in \S\ref{section:opening-ports}.
\index{open_port/2 BIF@\T{open_port/2} BIF|)}
\index{port!opening|)}

\subsection{\T{port_close/1}}

\label{section:portclose1}
\index{port_close/1 BIF@\T{port_close/1} BIF|(}
\index{port!closing|(}

A port is closed.

\TYPE

\T{port_close(port()]) -> true}.

\EXITS

\T{port_close/1} exits with cause \T{badarg} if $\TZ{v}_1$ is not a port.
It may also exit with cause \T{badarg} if $\TZ{v}_1$ is  already closed.

\EVALUATION

If $\TZ{v}_1$ is already closed, exit with cause \T{badarg}.  % bad choice
Otherwise, close port $\TZ{v}_1$.
\index{port_close/1 BIF@\T{port_close/1} BIF|)}
\index{port!closing|)}

\subsection{\T{port_info/1}}

\label{section:portinfo1}
\index{port_info/1 BIF@\T{port_info/1} BIF|(}
\index{port!information|(}

Information about various properties of a port is returned.

\TYPE

\iftrue
\T{port_info(port()) -> term()}.
\else
% This just isn't correct: the atom undefined may be returned.
\T{port_info(port()) -> [\{atom(),term()\}]}.
\fi

\EXITS

\T{port_info/1} exits with cause \T{badarg} if $\TZ{v}_1$ is not a port.

\EVALUATION

The BIF returns information about the port $\TZ{v}_1$.
\begin{itemize}
\item If that port is not open, the result is the atom \T{undefined}.
\item Otherwise, the result is
a list of 2-tuples, each of which is the same as the result of an
application of the BIF \T{port_info/2} (\S\ref{section:portinfo2})
to $\TZ{v}_1$ and a distinct atom in the left column of Table~\ref{table:portinfo}.
The list should include all properties for which \T{port_info/2} gives meaningful
information.
\end{itemize}
\index{port_info/1 BIF@\T{port_info/1} BIF|)}
\index{port!information|)}

\subsection{\T{port_info/2}}

\label{section:portinfo2}
\index{port_info/2 BIF@\T{port_info/2} BIF|(}
\index{port!information|(}

Information about some property of a port is returned,
as described in Table~\ref{table:portinfo}.

\begin{table}[hbp]
\begin{center}
\begin{tabular}{@{}ll@{}}
\hline
Second argument & Information returned about the port \\
\hline
\T{id} & The ID of the port. \\ 
\T{connected} & The process owning the port. \\
\T{input} & The number of bytes read from the port. \\
\T{links} & The processes to which the port is linked. \\
\T{name} & The driver or external process to which the port \\
& is opened. \\
\T{output} & The number of bytes written to the port. \\
\hline
\end{tabular}
\caption{Alternatives for the BIF \T{port_info/2}.}
\label{table:portinfo}
\end{center}
\end{table}

\TYPE

\T{port_info(pid(),atom()) -> term()}.

\EXITS

\T{port_info/2} exits with cause \T{badarg} if $\TZ{v}_1$ is not a port
or $\TZ{v}_2$ is not one of the atoms in the left column of
Table~\ref{table:portinfo}.

\EVALUATION

\begin{itemize}
\item If port $\TZ{v}_1$ is closed, the result is the atom \T{undefined}.
\item Otherwise, the BIF returns information about the port $\TZ{v}_1$ and
the result is always a 2-tuple where the first element is $\TZ{v}_2$:
\begin{itemize}
\item If $\TZ{v}_2$ is \T{id}, then
return \T{\{id,ID[$\TZ{v}_1$]\}}.
\item If $\TZ{v}_2$ is \T{connected}, then
return \T{\{connected,owner[$\TZ{v}_1$]\}}.
\item If $\TZ{v}_2$ is \T{input}, then
return \T{\{input,count_in[$\TZ{v}_1$]\}}.
\item If $\TZ{v}_2$ is \T{links}, then
return \T{\{links,\Z{Lst}\}}, where \TZ{Lst} is a
list representing the value of \T{linked[$\TZ{v}_1$]}.
\item If $\TZ{v}_2$ is \T{name}, then
return \T{\{name,command[$\TZ{v}_1$]\}}.
\item If $\TZ{v}_2$ is \T{output}, then
return \T{\{output,count_out[$\TZ{v}_1$]\}}.
\end{itemize}
\end{itemize}
Let \TZ{P} be the process calling the BIF.  The behaviour with respect
to signals (\S\ref{section:signals}) should be as if the result was
obtained by \TZ{P} sending an \I{info request} signal to port $\TZ{v}_1$
with $\TZ{v}_2$ as additional information, and port $\TZ{v}_1$ responding
with a message to \TZ{P} containing the result.
\index{port_info/2 BIF@\T{port_info/2} BIF|)}
\index{port!information|)}

\subsection{\T{ports/0}}

\label{section:ports0}
\index{ports/0 BIF@\T{ports/0} BIF|(}

A list is returned of all open ports
on the node on which the BIF is called.

\TYPE

\T{ports() -> [port()]}.

\EXITS

\T{ports/0} always completes normally.

\EVALUATION

The BIF returns a list representing the value of \T{ports[node[\Z{P}]]},
where \TZ{P} is the process calling the BIF.
\index{ports/0 BIF@\T{ports/0} BIF|)}

\iffalse
WHAT ABOUT THESE BIFS???
\begin{verbatim}
port_command
port_control
port_connect
\end{verbatim}
\fi
\index{port!BIFs|)}

\section{Miscellaneous builtin functions}

\label{section:misc-bifs}

\subsection{\T{date/0}}

\label{section:date0}
\index{date/0 BIF@\T{date/0} BIF|(}

The local date when the BIF is called is returned as a triple.

\TYPE

\T{date() -> \{int(),int(),int()\}}.

\EXITS

\T{date/0} always completes normally.

\EVALUATION

Let the current local year, month and day at the time of evaluation be
$y$, $m$ and $d$.  Let \I{Month} be a function mapping January to 1,
February to 2, \ldots, December to 12.

A triple \T{\{\Z{Year},\Z{Month},\Z{Day}\}} is returned, where
\TZ{Year} is the \Erlang\ integer $\Er[y]$, \TZ{Month} is
the \Erlang\ integer $\Er[\I{Month}(m)]$, and
\TZ{Day} is the \Erlang\ integer $\Er[d]$.

\EXAMPLES

\begin{itemize}
\item \T{date()} evaluated on the 29th of June, 1996, would return \T{\{1996,6,29\}}.
\item \T{date()} evaluated on the 1st of January, 2000, would return \T{\{2000,1,1\}}.
\end{itemize}
\index{date/0 BIF@\T{date/0} BIF|)}

\subsection{\T{erlang:hash/2}}

\label{section:hash2}
\index{erlang:hash/2 BIF@\T{erlang:hash/2} BIF|(}
\index{hash/2 BIF@\T{hash/2} BIF|(}
\index{term!hashing|(}

A hash value in a specified range for an arbitrary term is returned.

\TYPE

\T{hash(term(),fixnum()) -> int()}.

\EXITS

\T{hash/2} exits with cause \T{badarg} if $\TZ{v}_2$ is not a nonnegative
integer.

\EVALUATION

The BIF returns $\Re[\I{Hash}(\TZ{v}_1,\Er[\TZ{v}_2])+1]$, where the function
\I{Hash} is as defined in \S\ref{chapter:hashing}.  That is, the BIF maps each
\Erlang\ term to an integer in the range $[1,\TZ{v}_2]$. The function \I{Hash}
is defined in such a way as to be portable across nodes and
independent of time.  (It is obviously not invertible.)
\index{erlang:hash/2 BIF@\T{erlang:hash/2} BIF|)}
\index{hash/2 BIF@\T{hash/2} BIF|)}
\index{term!hashing|)}

\subsection{\T{make_ref/0}}

\label{section:makeref0}
\index{make_ref/0 BIF@\T{make_ref/0} BIF|(}
\index{ref!creating new|(}

A ref is returned that is different from all refs created previously on the node
and that is different from all refs created on other nodes.

\TYPE

\T{make_ref() -> ref()}.

\EXITS

\T{make_ref/0} always completes normally.

\EVALUATION

Let \TZ{N} be the node on which \T{make_ref/0} is called.  The BIF
invokes the operation \T{next_ref[\Z{N}]}
(\S\ref{section:node-state-dynamic}) and the result is a new ref
that is returned.  This also modifies the state
\T{ref_state[\Z{N}]} so future invocations of
\T{next_ref[\Z{N}]} will produce different refs.
\index{make_ref/0 BIF@\T{make_ref/0} BIF|)}
\index{ref!creating new|)}

\subsection{\T{now/0}}

\label{section:now0}
\index{now/0 BIF@\T{now/0} BIF|(}

A 3-tuple of integers is returned that is guaranteed to be different
for each invocation on a node.

\TYPE

\T{now() -> \{int(),int(),int()\}}.

\EXITS

\T{now/0} always completes normally.

\EVALUATION

Two calls of the BIF \T{now()} by processes residing on the same node can never
return the same term.\footnote{The name of the BIF comes from the fact that the
three integers normally represent the universal time (with microsecond resolution).
However, the time might be inaccurate if several calls are made within a microsecond.}
\index{now/0 BIF@\T{now/0} BIF|)}

\subsection{\T{throw/1}}

\label{section:throw1}
\index{throw/1 BIF@\T{throw/1} BIF|(}

A value is thrown.

\TYPE

\T{throw(term()) -> _}.

\EXITS

\T{throw/1} always completes abruptly with reason \T{\{'THROW',$\Z{v}_1$\}}.
\index{throw/1 BIF@\T{throw/1} BIF|)}

\subsection{\T{time/0}}

\label{section:time0}
\index{time/0 BIF@\T{time/0} BIF|(}

The local time of day when the BIF is called is returned as a triple.

\TYPE

\T{time() -> \{int(),int(),int()\}}.

\EXITS

\T{time/0} always completes normally.

\EVALUATION

Let the current local hour, minute and second at the time of evaluation be
$h$, $m$ and $s$.  The hour is on 24-hour format.

A triple \T{\{\Z{Hour},\Z{Minute},\Z{Second}\}} is returned, where
\TZ{Hour} is the \Erlang\ integer $\Er[h]$, \TZ{Minute} is
the \Erlang\ integer $\Er[m]$, and
\TZ{Second} is the \Erlang\ integer $\Er[s]$.

\EXAMPLES

\begin{itemize}
\item \T{time()} evaluated at five minutes and forty-two seconds
past midnight would return \T{\{0,5,42\}}.
\item \T{time()} evaluated at five minutes and forty-two seconds
past noon would return \T{\{12,5,42\}}.
\item \T{time()} evaluated at five minutes and forty-two seconds
before midnight would return \T{\{23,44,18\}}.
\end{itemize}
\index{time/0 BIF@\T{time/0} BIF|)}

\section{Reserved function names}

\label{section:reserved-function-names}
\index{function!reserved names|(}
\index{apply_lambda/2@\T{apply_lambda/2} reserved|(}
\index{module_info/0@\T{module_info/0}!reserved|(}
\index{module_info/1@\T{module_info/1}!reserved|(}
\index{module_lambdas/4 reserved@\T{module_lambdas/4} reserved|(}
\index{record/2@\T{record/2}!reserved|(}
\index{record_index/2@\T{record_index/2} reserved|(}
\index{record_info/2@\T{record_info/2} reserved|(}

The function names in Table~\ref{table:reserved-functions}
do not name BIFs but
are recognized by the compiler and
a module must not define any function with one of these names.

The reason may be that the compiler automatically generates
a definition of a function with such a name (e.g., \T{module_info/0}
and \T{module_info/1}), or that applications
of function named as such are treated specially (e.g., \T{record_info/2}).

Function names without a reference in Table~\ref{table:reserved-functions}
are reserved because
\ifStd it is known that existing implementations use \fi
\ifOld \OldErlang\ uses \fi
them for internal purposes.  If a module defines a function with one
of the names below, the compiler gives a compile-time error.

\begin{table}[hbp]
\begin{center}
\begin{tabular}{@{}ll@{}}
\hline
Function name & Described \\ \hline
\T{apply_lambda/2} \\
\T{module_info/0} & \S\ref{section:moduleinfo0} \\
\T{module_info/1} & \S\ref{section:moduleinfo1} \\
\T{module_lambdas/4} \\
\T{record/2} & \S\ref{section:record2} \\
\T{record_index/2} \\
\T{record_info/2} \\ \hline
\end{tabular}
\caption{Reserved function names}
\label{table:reserved-functions}
\end{center}
\end{table}
\index{function!reserved names|)}
\index{apply_lambda/2@\T{apply_lambda/2} reserved|)}
\index{module_info/0@\T{module_info/0}!reserved|)}
\index{module_info/1@\T{module_info/1}!reserved|)}
\index{module_lambdas/4 reserved@\T{module_lambdas/4} reserved|)}
\index{record/2@\T{record/2}!reserved|)}
\index{record_index/2@\T{record_index/2} reserved|)}
\index{record_info/2@\T{record_info/2} reserved|)}

\iffalse
\chapter{BIFs in the standard libraries}

% this is sort of \StdErlang\ stuff, except limits.
\section{The \T{atom} library}

\label{section:atom-library}

\subsection{\T{to_string/1}}

The printname of an atom is obtained as a string.

\TYPE

\T{atom:to_string(atom()) -> string()}.

\EXITS

\T{atom:to_string/1} exits with cause \T{badarg} if $\TZ{v}_1$ is not an atom.

\EVALUATION

A string representing the printname of the atom $\TZ{v}_1$ is returned.

\subsection{\T{from_string/1}}

Given a string, an atom with that printname is obtained.

\TYPE

\T{atom:from_string(string()) -> atom()}.

\EXITS

\T{atom:from_string/1} exits with cause \T{badarg} if $\TZ{v}_1$ is not a string.

\EVALUATION

An atom having the printname represented by the string $\TZ{v}_1$ is returned.

\subsection{\T{size/1}}

The length of the printname of an atom is returned.
\T{size/1} is a guard BIF.

\TYPE

\T{atom:size(atom()) -> int()}.

\EXITS

\T{atom:size/1} exits with cause \T{badarg} if $\TZ{v}_1$ is not an atom.

\EVALUATION

The size of the printname of the atom $\TZ{v}_1$ is returned.

\section{The \T{char} library}

\label{section:char-module}

\section{The \T{float} library}

\label{section:float-module}

\section{The \T{integer} library}

\label{section:integer-module}

\subsection{\T{to_string/1}}

The shortest decimal numeral denoting an integer
is obtained as a string.

\TYPE

\T{integer:to_string(int()) -> string()}.

\EXITS

\T{integer:to_string/1} exits with cause \T{badarg} if $\TZ{v}_1$ is not an integer.

\EVALUATION

Let $n$ be a numeral satisfying satisfies
the following axioms:
\begin{itemize}
\item $n$ interpreted as a decimal numeral denotes $\Re[\TZ{v}_1]$;
\item $n$ consists of decimal digits, except that the leftmost character
is \T{-} when $\Re[\TZ{v}_1]<0$;
\item the leftmost digit of $n$ is not \T{0} when $\Re[\TZ{v}_1]<>0$;
\item $n$ is \T{0} when $\Re[\TZ{v}_1]=0$.
\end{itemize}
$n$ is unique and is the shortest numeral
denoting the integer $\Re[\TZ{v}_1]$.

\T{integer:to_string/1} returns a string representing $n$.

\subsection{\T{from_string/1}}

Given a string that is a decimal numeral, the corresponding integer is obtained.

\TYPE

\T{integer:from_string(string()) -> int()}.

\EXITS

\T{integer:from_string/1} exits with cause \T{badarg} if $\TZ{v}_1$ is
not a string or if that string does not represent an
\NT{IntegerLiteral} (\S\ref{section:integer-literals}).

\EVALUATION

An \Erlang\ integer representing the number denoted by the \NT{IntegerLiteral}
is returned.

\section{The \T{list} library}

\label{section:list:nth2}

\section{The \T{math} library}

\label{section:math-library}

This library contains a collection of simple arithmetic functions
on floats.

\subsection{Void functions}

\addcontentsline{toc}{subsection}{\protect\numberline{}{\T{pi/0}}}

\TYPE

\T{math:pi() -> float()}.

\EXITS

\T{pi/0} always completes normally.

\EVALUATION

\begin{tabular}{@{}ll@{}}
\T{pi/0} & $\Re[\pi]$
\end{tabular}

\subsection{Unary functions}

\addcontentsline{toc}{subsection}{\protect\numberline{}{\T{cos/1}}}
\addcontentsline{toc}{subsection}{\protect\numberline{}{\T{cosh/1}}}
\addcontentsline{toc}{subsection}{\protect\numberline{}{\T{sin/1}}}
\addcontentsline{toc}{subsection}{\protect\numberline{}{\T{sinh/1}}}
\addcontentsline{toc}{subsection}{\protect\numberline{}{\T{tan/1}}}
\addcontentsline{toc}{subsection}{\protect\numberline{}{\T{tanh/1}}}
\addcontentsline{toc}{subsection}{\protect\numberline{}{\T{acos/1}}}
\addcontentsline{toc}{subsection}{\protect\numberline{}{\T{acosh/1}}}
\addcontentsline{toc}{subsection}{\protect\numberline{}{\T{asin/1}}}
\addcontentsline{toc}{subsection}{\protect\numberline{}{\T{asinh/1}}}
\addcontentsline{toc}{subsection}{\protect\numberline{}{\T{atan/1}}}
\addcontentsline{toc}{subsection}{\protect\numberline{}{\T{atanh/1}}}
\addcontentsline{toc}{subsection}{\protect\numberline{}{\T{erf/1}}}
\addcontentsline{toc}{subsection}{\protect\numberline{}{\T{erfc/1}}}
\addcontentsline{toc}{subsection}{\protect\numberline{}{\T{exp/1}}}
\addcontentsline{toc}{subsection}{\protect\numberline{}{\T{log/1}}}
\addcontentsline{toc}{subsection}{\protect\numberline{}{\T{log10/1}}}
\addcontentsline{toc}{subsection}{\protect\numberline{}{\T{sqrt/1}}}

\TYPE

\T{math:cos(num()) -> float()}, \\
\T{math:cosh(num()) -> float()}, \\
\T{math:sin(num()) -> float()}, \\
\T{math:sinh(num()) -> float()}, \\
\T{math:tan(num()) -> float()}, \\
\T{math:tanh(num()) -> float()}, \\
\T{math:acos(num()) -> float()}, \\
\T{math:acosh(num()) -> float()}, \\
\T{math:asin(num()) -> float()}, \\
\T{math:asinh(num()) -> float()}, \\
\T{math:atan(num()) -> float()}, \\
\T{math:atanh(num()) -> float()}, \\
\T{math:erf(num()) -> float()}, \\
\T{math:erfc(num()) -> float()}, \\
\T{math:exp(num()) -> float()}, \\
\T{math:log(num()) -> float()}, \\
\T{math:log10(num()) -> float()}, \\
\T{math:sqrt(num()) -> float()}.

\EXITS

TBW. % !!!

\EVALUATION

\begin{tabular}{@{}ll@{}}
\T{cos/1} & $\cos \TZ{v}_1$ \\
\T{cosh/1} & $\cosh \TZ{v}_1$ \\
\T{sin/1} & $\sin \TZ{v}_1$ \\
\T{sinh/1} & $\sinh \TZ{v}_1$ \\
\T{tan/1} & $\tan \TZ{v}_1$ \\
\T{tanh/1} & $\tanh \TZ{v}_1$ \\
\T{acos/1} & $\arccos \TZ{v}_1$ \\
\T{acosh/1} & hyperbolic $\arccos \TZ{v}_1$ \\
\T{asin/1} & $\arcsin \TZ{v}_1$ \\
\T{asinh/1} & hyperbolic $\arcsin \TZ{v}_1$ \\
\T{atan/1} & $\arctan \TZ{v}_1$ \\
\T{atanh/1} & hyperbolic $\arctan \TZ{v}_1$ \\
\T{erf/1} & error function of $\TZ{v}_1$ \\
\T{erfc/1} &  error function of $\TZ{v}_1$ \\
\T{exp/1} & $e^{\TZ{v}_1}$ \\
\T{log/1} & $\ln\TZ{v}_1$ \\
\T{log10/1} & $\log_{10}\TZ{v}_1$ \\
\T{sqrt/1} & $\sqrt{\TZ{v}_1}$
\end{tabular}

\subsection{Binary functions}

\addcontentsline{toc}{subsection}{\protect\numberline{}{\T{pow/2}}}
\addcontentsline{toc}{subsection}{\protect\numberline{}{\T{atan2/2}}}

\TYPE

\T{math:pow(num(),num()) -> float()}, \\
\T{math:atan2(num(),num()) -> float()}.

\EXITS

TBW. % !!!

\EVALUATION

\begin{tabular}{@{}ll@{}}
\T{pow/2} & $\TZ{v}_1^{\TZ{v}_1}$ \\
\T{atan2/2} & $\arctan \frac{\TZ{v}_1}{\TZ{v}_2}$
\end{tabular}

\section{The \T{string} library}
\fi

\iffalse
% Don't include this stuff either until we know.
\section{The \T{limits} library}

\label{section:limits-library}

\subsection{\T{atom_parameter/1}}

A parameter of the atom type of the implementation is returned
(cf.\ \S\ref{section:atoms}).

\TYPE

\T{limits:atom_parameter(maxatomlength) -> int()}.

\EXITS

\T{limits:atom_parameter/1} exits with cause \T{badarg} if $\TZ{v}_1$ is not an atom
or if it is not the atom \T{maxatomlength}.

\EVALUATION

$\TZ{v}_1$ must be \T{maxatomlength} and the BIF returns
$\Re[\mathit{maxatomlength}]$.

\subsection{\T{integer_parameter/1}}

A parameter of the integer type of the implementation is returned
(cf.\ \S\ref{section:integer-type}).

\TYPE

\T{limits:integer_parameter(bounded) -> bool()} ; \\
\T{limits:integer_parameter(minint) -> int()} ; \\
\T{limits:integer_parameter(maxint) -> int()} ; \\
\T{limits:integer_parameter(minfixnum) -> int()} ; \\
\T{limits:integer_parameter(maxfixnum) -> int()}.

\EXITS

\T{limits:integer_parameter/1} exits with cause \T{badarg} if $\TZ{v}_1$ is not an atom
or if it is not one of the atoms \T{bounded}, \T{minint}, \T{maxint},
\T{minfixnum} or \T{maxfixnum}.

\EVALUATION

The evaluation depends on $\TZ{v}_1$:
\begin{itemize}
\item \T{bounded}: return $\Re[\mathit{bounded}]$.
\item \T{minint}: return $\Re[\mathit{minint}]$.
\item \T{maxint}: return $\Re[\mathit{maxint}]$.
\item \T{minfixnum}: return $\Re[\mathit{minfixnum}]$.
\item \T{maxfixnum}: return $\Re[\mathit{maxfixnum}]$.
\end{itemize}

\subsection{\T{float_parameter/1}}

A parameter of the float type of the implementation is returned
(cf.\ \S\ref{section:float-type}).

\TYPE

\T{limits:float_parameter(radix) -> int()} ; \\
\T{limits:float_parameter(precision) -> int()} ; \\
\T{limits:float_parameter(emin) -> int()} ; \\
\T{limits:float_parameter(emax) -> int()} ; \\
\T{limits:float_parameter(denorm) -> bool()} ; \\
\T{limits:float_parameter(fmax) -> float()} ; \\
\T{limits:float_parameter(fmin) -> float()} ; \\
\T{limits:float_parameter(fmin_norm) -> float()} ; \\
\T{limits:float_parameter(epsilon) -> float()} ; \\
\T{limits:float_parameter(rnd_error) -> float()} ; \\
\T{limits:float_parameter(rnd_style) -> atom()} ; \\
\T{limits:float_parameter(iec_559) -> bool()}.

\EXITS

\T{limits:float_parameter/1} exits with cause \T{badarg} if $\TZ{v}_1$ is not an atom
or if it is not one of the atoms \T{radix}, \T{precision}, \T{emin},
\T{emax}, \T{denorm}, \T{fmax}, \T{fmin}, \T{fmin_norm}, \T{epsilon}, \T{rnd_error},
\T{rnd_style} or \T{iec_559}.

\EVALUATION

The evaluation depends on $\TZ{v}_1$:
\begin{itemize}
\item \T{radix}: return $\Re[r]$.
\item \T{precision}: return $\Re[p]$.
\item \T{emin}: return $\Re[\mathit{emin}]$.
\item \T{emax}: return $\Re[\mathit{emax}]$.
\item \T{denorm}: return $\Re[\mathit{denorm}]$.
\item \T{fmax}: return $\Re[\mathit{fmax}]$.
\item \T{fmin}: return $\Re[\mathit{fmin}]$.
\item \T{fmin_norm}: return $\Re[\mathit{fmin}_N]$.
\item \T{epsilon}: return $\Re[\mathit{epsilon}]$.
\item \T{rnd_error}: return $\Re[\mathit{rnd\_error}]$ (\S\ref{section:float-operations}).
\item \T{rnd_style}: return $\begin{cases}
\T{nearest} & \text{if $\I{rnd\_style}=\B{nearest}$,} \\
\T{truncate} & \text{if $\I{rnd\_style}=\B{truncate}$, and} \\
\T{other} & \text{if $\I{rnd\_style}=\B{other}$.}
\end{cases}$ (\S\ref{section:float-operations}).
\item \T{iec_559}: return $\Re[\mathit{iec\_559}]$ (\S\ref{section:arith-iec559}).
\end{itemize}

\subsection{\T{tuple_parameter/1}}

A parameter of the tuple type of the implementation is returned
(cf.\ \S\ref{section:atoms}).

\TYPE

\T{limits:tuple_parameter(maxtuplesize) -> int()}.

\EXITS

\T{limits:tuple_parameter/1} exits with cause \T{badarg} if $\TZ{v}_1$ is not an atom
or if it is not the atom \T{maxtuplesize}.

\EVALUATION

$\TZ{v}_1$ must be \T{maxtuplesize} and the BIF returns
$\Re[\mathit{maxtuplesize}]$.
\fi
