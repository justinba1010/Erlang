%
% %CopyrightBegin%
%
% Copyright Ericsson AB 2017. All Rights Reserved.
%
% Licensed under the Apache License, Version 2.0 (the "License");
% you may not use this file except in compliance with the License.
% You may obtain a copy of the License at
%
%     http://www.apache.org/licenses/LICENSE-2.0
%
% Unless required by applicable law or agreed to in writing, software
% distributed under the License is distributed on an "AS IS" BASIS,
% WITHOUT WARRANTIES OR CONDITIONS OF ANY KIND, either express or implied.
% See the License for the specific language governing permissions and
% limitations under the License.
%
% %CopyrightEnd%
%

\chapter{Programs and modules}

\label{chapter:programs-modules}

\index{module!declaration|(}
An \Erlang\ \emph{module declaration} consists of a collection of
\emph{forms}.  Forms are function declarations together with some
attributes\iftypedecl, type declarations\fi\ and record declarations
for the module.  It is the smallest unit of code that can be
separately compiled and loaded.  An \Erlang\ compiler thus has access
to all the code of a module at compile-time.  Therefore it can
determine at compile-time, for example, the names of all functions
being declared in the module.
\index{module!declaration|)}

\index{program|(}
An \Erlang\ \emph{program} typically consists of a number of modules,
out of which some may be standard \Erlang\ modules or parts of library
applications \cite{otp-dev-ref}.
\index{program|)}

\section{Module declarations}

\label{section:module-declarations}
\index{module!declaration|(}

A module declaration may begin with one or more file
attributes\index{file attribute@\T{file} attribute}, as
described in \S\ref{section:file-attrib}.

\index{module attribute@\T{module} attribute|(}
\index{module!name|(}
After those must follow a module attribute
\T{-module(\Z{Name})}, in which the argument \TZ{Name} must be an atom.  The
module attribute states the name of the module.
\index{module attribute@\T{module} attribute|)}
\index{module!name|)}

The module attribute is followed by a \emph{header part}\index{module!header
part of declaration}, consisting of a
(possibly empty) sequence of
header forms, which are additional attributes and declarations of the module.

Finally there is a \emph{code part}\index{module!code
part of declaration}, consisting of a (possibly empty) sequence of
\emph{program forms}: function
declarations, possibly interspersed with such attributes and declarations that
need not be in the header part.

\begin{rules}
\grrule{ModuleDeclaration}
       {\OPT{FileAttributes} \NT{ModuleAttribute} \OPT{HeaderForms} \OPT{ProgramForms}}

\grrule{FileAttributes}
       {\NT{FileAttribute} \OR
        \NT{FileAttributes} \NT{FileAttribute}}

\grrule{ModuleAttribute}
       {\TXT{-} \TXT{module} \TXT{(} \NT{ModuleName} \TXT{)} \NT{FullStop}}

\grrule{ModuleName}
       {\NT{AtomLiteral}}
\end{rules}
\ifOld
There is a restriction in the file compiler that a file named
\T{\Z{Mod}.erl}\index{.erl file name extension@\T{.erl} file name extension}
must contain the full source code of an \Erlang\ module named \TZ{Mod}.
(This file may however
include other files containing header forms, cf.\
\S\ref{section:header-forms}.)
This restriction may be lifted
or altered in a future version of \Erlang.
\fi
\ifNew
The compiler of a \StdErlang\ implementation may require that a file named
\T{\Z{Mod}.erl}\index{.erl file name extension@\T{.erl} file name extension}
must contain the full source code of an \Erlang\ module named \TZ{Mod}.
\fi

\section{The header part}

\label{section:header-forms}
\index{module!header part of declaration}

\index{header!attribute|(}
\index{anywhere attribute|(}
The header part is a sequence of header forms, which are \emph{header
attributes} or \emph{anywhere attributes}.  The former may only appear
in the header part while the latter may also appear in the code part.
\iffalse
% [980514] Junk, I think:
, type
declarations (cf.\ \S\ref{section:type-declarations}) and record declarations (cf.\
\S\ref{section:record-declarations}), and they all begin with a dash.
There are also attributes that may appear anywhere in a module after
the header forms.
\fi

\begin{rules}
\grrule{HeaderForms}
       {\NT{HeaderForm} \OR
        \NT{HeaderForms} \NT{HeaderForm}}

\grrule{HeaderForm}
       {\NT{HeaderAttribute} \OR
        \NT{AnywhereAttribute}}

\grrule{HeaderAttribute}
       {\NT{ExportAttribute} \OR
        \NT{ImportAttribute} \OR
        \NT{CompileAttribute} \OR
        \NT{WildAttribute}}

\grrule{AnywhereAttribute}
       {\NT{FileAttribute} \OR
        \NT{MacroDefinition} \OR
        \NT{RecordDeclaration}\iftypedecl \OR
        \NT{TypeDeclaration} \OR
        \NT{RuleDeclaration}\fi}
\end{rules}
\index{header!attribute|)}
\index{anywhere attribute|)}

\iffalse Rule declarations are defined in Chapter~\ref{chapter:rules}.\fi

Macro definitions are defined in \S\ref{section:macros}.

The remaining attributes (both header attributes and others)
are described in the following sections.

\subsection{Export attributes}

\label{section:export}
\index{export attribute@\T{export} attribute|(}
\index{function!exported|(}
\index{module!exported functions of|(}

An export attribute is syntactically like an application of
a (hypothetical) function \T{export/1} to a list of
\emph{function names}, preceded by a dash.  Each function name
consists of a function symbol and an arity numeral, separated by
\T{/}.

\begin{rules}
\grrule{ExportAttribute}
       {\TXT{-} \TXT{export} \TXT{(} \NT{FunctionNameList} \TXT{)} \NT{FullStop}}

\grrule{FunctionNameList}
       {\TXT{[} \OPT{FunctionNames} \TXT{]}}

\grrule{FunctionNames}
       {\NT{FunctionName} \OR
        \NT{FunctionNames} \TXT{,} \NT{FunctionName}}

\grrule{FunctionName}
       {\NT{FunctionSymbol} \TXT{/} \NT{Arity}}
       
\grrule{FunctionSymbol}
       {\NT{AtomLiteral}}

\grrule{Arity}
       {\NT{IntegerLiteral}}
\end{rules}

There may be more than one export attribute among the header forms.
The order between functions names in an export attribute is
irrelevant.  The same function name may appear more than once in an
export attribute and in several export attributes.

If there is an export attribute in module \TZ{M} that lists the
function \T{\Z{F}/\Z{A}}, then the function with symbol \TZ{F} and
arity \TZ{A} in module \TZ{M} can be applied using a remote
application (\S\ref{section:function-application}).

Note that if there is \emph{not} an export attribute in module
\TZ{M} that lists the function \T{\Z{F}/\Z{A}}, then the function cannot
be applied through a remote application even in the code for module \TZ{M}.

It is a compile-time error if there is an export attribute listing a function
\T{\Z{F}/\Z{A}} but no declaration of a function \T{\Z{F}/\Z{A}}.
\index{export attribute@\T{export} attribute|)}
\index{function!exported|)}
\index{module!exported functions of|)}

\subsection{Import attributes}

\label{section:import-attribute}
\index{import attribute@\T{import} attribute|(}
\index{function!imported|(}
\index{module!imported functions of|(}

An import attribute is syntactically like an application of a
(hypothetical) function \T{import/1} to a module name and a list of
function names (cf.\ \S\ref{section:export}), preceded by a dash.

\begin{rules}
\grrule{ImportAttribute}
       {\TXT{-} \TXT{import} \TXT{(} \NT{ModuleName} \TXT{,} \NT{FunctionNameList} \TXT{)} \NT{FullStop}}
\end{rules}
Import attributes have the following role.  If there is an import
attribute in module \TZ{M} that lists the function \T{\Z{F}/\Z{A}} of
module $\TZ{M}'$, then a seemingly local function application
\T{\Z{F}(\(\Z{E}_{1}\),\tdots,\(\Z{E}_{\TZm{A}}\))} in module \TZ{M}
is syntactic sugar for a remote function application
\T{$\Z{M}'$:\Z{F}(\(\Z{E}_{1}\),\tdots,\(\Z{E}_{\TZm{A}}\))}
(unless there is a BIF named \T{\Z{F}/\Z{A}}, cf.\
\S\ref{section:application-exprs}).

There may be more than one import attribute among the header forms.
The order between import attributes and between functions names in an
import attribute is irrelevant.  The same function name may appear
more than once in an import attribute and in several import attributes
for the same module.

It is a compile-time error if in the same module there are import attributes
listing the same function name but different modules.

If in some module there is both an import attribute listing a function
\T{\Z{F}/\Z{A}} and a function declaration for \T{\Z{F}/\Z{A}}, then
all applications of the form
\T{\Z{F}(\(\Z{E}_{1}\),\tdots,\(\Z{E}_{\TZm{A}}\))} in the module will
refer to the imported function.  However, if there is an export
attribute listing \T{\Z{F}/\Z{A}}, it refers to the function being
declared.  For example, consider the following module:

\begin{verbatim}
-module(expimp).

-import(lists,[sort/1]).

-export([sort/1]).

sort(X) -> sort(X).
\end{verbatim}

The declaration of the function \T{sort/1} may appear to be recursive
but it is not: the application \T{sort(X)} to the right of `\T{->}'
calls the imported function.  However, the export attribute refers to
the function being declared in module \T{expimp}.  (Here the function
being declared simply returns the result of the imported function but
if the module is later replaced by a new version, the function
\T{sort/1} may be defined differently.)
\index{import attribute@\T{import} attribute|)}
\index{function!imported|)}
\index{module!imported functions of|)}

\subsection{Compile attributes}

\label{section:compile-attrib}
\index{compile attribute@\T{compile} attribute|(}
\index{module!compiler options for|(}

A compile attribute is syntactically like an application of a (hypothetical) unary
function \T{compile} to a list of terms.

\begin{rules}
\grrule{CompileAttribute}
       {\TXT{-} \TXT{compile} \TXT{(} \TXT{[} \OPT{Terms} \TXT{]} \TXT{)} \NT{FullStop}}
\end{rules}
The terms in the list will be appended at the end of the list of
options given to the \T{compile:file/2} function.  The terms must
therefore be acceptable as options for the \Erlang\ compiler being
used, or compilation will fail.

The compiler options are implementation-defined.
% or should they be???
\index{compile attribute@\T{compile} attribute|)}
\index{module!compiler options for|)}

\subsection{File attributes}

\label{section:file-attrib}
\index{file attribute@\T{file} attribute|(}

%Stinks.

A file attribute is syntactically like an application of a
(hypothetical) function \T{file/2} to a string literal (presumably a
file name although it is not an error if no such file exists) and a
line numeral, preceded by a dash.

\begin{rules}
\grrule{FileAttribute}
       {\TXT{-} \TXT{file} \TXT{(} \NT{StringLiteral} \TXT{,}
       \NT{LineNumeral} \TXT{)} \NT{FullStop}}

\grrule{LineNumeral}
       {\NT{IntegerLiteral}}
\end{rules}
The purpose of a file attribute \T{-file(\Z{F},\Z{L})} is to inform
the compiler that the code being compiled originated from line \TZ{L}
in the file \TZ{F} so that error messages can refer to the proper
place in the original file.
\index{file attribute@\T{file} attribute|)}

\iftypedecl
\subsection{Type declarations}

\label{section:type-declarations}

The \T{type} and \T{deftype} attributes are reserved for a possible future
extension.

\begin{rules}
\iftypedecl
\grrule{TypeDeclaration}
       {\TXT{-} \TXT{type} \TXT{(} \OPT{Tokens} \TXT{)} \NT{FullStop} \OR
        \TXT{-} \TXT{deftype} \TXT{(} \OPT{Tokens} \TXT{)} \NT{FullStop}}
\fi
\end{rules}
\fi

\subsection{Wild attributes}

\index{wild attribute|(}

A wild attribute is syntactically like an application of a unary
function other than \T{export/1}, \T{import/1}, \T{file/2},
\T{compile/1}, \T{type/1}, \T{deftype/1} and \T{record/1} to a term.

\begin{rules}
\grrule{WildAttribute}
       {\TXT{-} \NT{AtomLiteral} \TXT{(} \NT{Term} \TXT{)}
       \NT{FullStop}}
\end{rules}

\index{module_info/1@\T{module_info/1}!function|(}
The compiler gathers all wild attributes for a module \TZ{M} and at
runtime, a BIF call \T{\Z{M}:module_info(attributes)} returns a list
of them (cf.\ \S\ref{section:moduleinfo1}).
\index{module_info/1@\T{module_info/1}!function|)}
\index{wild attribute|)}

\section{Program forms}

\label{section:program-forms}
\index{program!form|(}
\index{function!declaration|(}

The program forms are \emph{function declarations} and such attributes
that may appear anywhere in a module declaration. (For unambiguity of
the grammar, it is required that the \NT{ProgramForms} begins with a
\NT{FunctionDeclaration} but there is no significant difference
between an \NT{AnywhereAttribute} that appears as part of the
\NT{HeaderForms} and one that appears as part of the
\NT{ProgramForms}.)

\begin{rules}
\grrule{ProgramForms}
       {\NT{FunctionDeclaration} \OR
        \NT{ProgramForms} \NT{FunctionDeclaration} \OR
        \NT{ProgramForms} \NT{AnywhereAttribute}}

\grrule{FunctionDeclaration}
       {\NT{FunctionClauses} \NT{FullStop}}

\grrule{FunctionClauses}
       {\NT{FunctionClause} \OR
        \NT{FunctionClauses} \TXT{;} \NT{FunctionClause}}

\grrule{FunctionClause}
       {\NT{FunctionSymbol} \NT{FunClause}}

\ifappendix\else
\grrule{FunClause}
       {\TXT{(} \OPT{Patterns} \TXT{)} \OPT{ClauseGuard} \NT{ClauseBody}}
\fi
\end{rules}
(The rule for \NT{FunClause} is repeated from \S\ref{section:fun-exprs} for
convenience.)

\index{function!clause|(}
\index{function!name|(}
\index{function!arity|(}
Each function declaration consists of one or more \emph{function
clauses}, separated by semicolons.  Every clause of a function
declaration must begin with the same \NT{AtomLiteral} and the sequence
of arguments following it must have the same length in every clause.
The \NT{AtomLiteral} is the function symbol and together with the
length of the argument sequences, which is the arity of the function,
it forms the \emph{function name}.
\index{function!name|)}
\index{function!arity|)}
A function declaration for a function named \T{\Z{F}/\Z{A}} is thus on
the form
\begin{alltt}
\Z{F}(\(\Z{P}\sb{1,1}\),\tdots,\(\Z{P}\sb{1,\TZm{A}}\)) \([\mbox{\T{when \(\Z{G}\sb{1}\)}}]\) -> \(\Z{B}\sb{1}\) ;
\(\vdots\) ;
\Z{F}(\(\Z{P}\sb{k,1}\),\tdots,\(\Z{P}\sb{k,\TZm{A}}\)) \([\mbox{\T{when \(\Z{G}\sb{k}\)}}]\) -> \(\Z{B}\sb{k}\).
\end{alltt}
where the guard part of each function clause is optional.
\index{function!clause|)}

It is a compile-time error if in a module there is more than one
declaration for a function name.

Application of functions is defined in \S\ref{section:application-exprs}.
\index{program!form|)}
\index{function!declaration|)}

\section{Record declarations}

\label{section:record-declarations}
\index{record!declaration|(}
\index{record@\T{record}!attribute|(}

\begin{rules}
\grrule{RecordDeclaration}
       {\TXT{-} \TXT{record} \TXT{(} \NT{RecordType} \TXT{,} \NT{RecordDeclTuple} \TXT{)} \NT{FullStop}}

\grrule{RecordDeclTuple}
       {\TXT{\char`\{} \OPT{RecordFieldDecls} \TXT{\char`\}}}

\grrule{RecordFieldDecls}
       {\NT{RecordFieldDecl} \OR
        \NT{RecordFieldDecls} \TXT{,}\ \NT{RecordFieldDecl}}

\grrule{RecordFieldDecl}
       {\NT{RecordFieldName} \OPT{RecordFieldValue}}
\end{rules}
(The rules for \NT{RecordType} and \NT{RecordFieldName} appear
in \S\ref{section:pattern-matching} while
the rules for \NT{RecordFieldValue} appear
in \S\ref{section:record-exprs}.)

In a record declaration
\begin{alltt}
-record(\Z{R},\{\(\Z{F}\sb{1}[\mbox{\T{=\(\Z{E}\sb{1}\)}}],\tdots,\Z{F}\sb{n}[\mbox{\T{=\(\Z{E}\sb{n}\)}}]\)\})
\end{alltt}
the field names $\TZ{F}_1$, \ldots, $\TZ{F}_n$ must all be distinct
and the (optional) default initializer expressions $\TZ{E}_1$, \ldots,
$\TZ{E}_n$ will be evaluated in an empty environment and must have an
empty output environment.  It should be a compile-time error if a
default initializer expression contains a free variable.

The scope of the record declaration begins immediately after its
lexical occurrence and ends at the end of the module declaration.  In
a module there must be at most one record declaration for each record
type.  It is possible for two modules that are part of the same
application to have incoherent record declarations. This could lead to
severe problems. Similar problems may appear, unless caution is taken,
when loading a new version of a module if its record declarations have
changed (because there may be existing records created according to
the old record declarations).

The record declaration establishes \TZ{R} as the name of a record type
having $n$ fields named $\TZ{F}_1$, \ldots, $\TZ{F}_n$.  The optional
default initializer expression $\TZ{E}_i$ is used when a record is
created (\S\ref{section:record-creation}) and no value is specified
for the field named $\TZ{F}_i$.  (If no default initializer expression
is given either, then the atom \T{undefined} is used.)  For example, a
record declaration
\begin{verbatim}
-record(music,{title,artist,medium=cd,number})
\end{verbatim}
declares \T{music} to be a record type and that each record term of type
\T{music} has four
fields named \T{title}, \T{artist}, \T{medium} and \T{number}.  An expression
\begin{verbatim}
#music{}
\end{verbatim}
creates a record in which these fields have the values
\T{undefined}, \T{undefined}, \T{cd} and \T{undefined}, respectively, while
an expression
\begin{verbatim}
#music{title="The Dark Side of the Moon",
       artist="Pink Floyd",
       medium=lp}
\end{verbatim}
creates a record in which the fields have the values
\T{"The Dark side of the Moon"}, \T{"Pink Floyd"}, \T{lp} and \T{undefined}, respectively
(the default initializer expression \T{cd} for \T{medium} was overridden).
\index{record!declaration|)}
\index{record@\T{record}!attribute|)}
\index{module!declaration|)}

\section{The module information functions}

When compiling a module, the compiler automatically adds declarations
for two functions \T{module_info/0} and \T{module_info/1} in it.

\subsection{The function \T{module\char'137info/0}}

\label{section:moduleinfo0}
\index{module_info/0@\T{module_info/0}!function|(}

In any module \TZ{M}, the compiler automatically adds a declaration of a function
\T{module_info/0}, such that an application \T{\Z{M}:module_info()}
returns an association list (cf.\
\S\ref{section:assocationlists}).  This association list is such
that for every key \TZ{K} that the function \T{module_info/1} accepts,
except \T{module}, the list contains a 2-tuple
\T{\{\Z{K},\Z{M}:module_info(\Z{K})\}} and no other 2-tuples.
\index{module_info/0@\T{module_info/0}!function|)}

\subsection{The function \T{module\char'137info/1}}

\label{section:moduleinfo1}
\index{module_info/1@\T{module_info/1}!function|(}

In any module \TZ{M}, the compiler adds a declaration for an exported function
\T{module_info/1} such that for certain terms, the function returns a value.
It must hold that:
\begin{itemize}
\item \index{module!name}
\T{\Z{M}:module_info(module)} returns \TZ{M} (i.e., the name of the module as an atom).
\item \index{module!exported functions of|(}
\T{\Z{M}:module_info(exports)} returns a list of 2-tuples such that
for every function name \T{\Z{F}/\Z{A}} that appears in an export
attribute, the list contains a 2-tuple \T{\{\Z{F},\Z{A}\}}.  That is, the first element
of each 2-tuple is an atom and the second element is an integer.  There are no
other 2-tuples in the list and the order between the 2-tuples is undefined.
\index{module!exported functions of|)}
\item \index{module!imported functions of|(}
\T{\Z{M}:module_info(imports)} returns a list of 2-tuples such that
for every function name \T{\Z{F}/\Z{A}} that appears in an import attribute for
a module $\TZ{M}'$, the list contains a 2-tuple \T{\{\{\Z{F},\Z{A}\},$\TZ{M}'$\}}.
That is, the first element of each 2-tuple is itself a 2-tuple of an atom and an integer,
and the second element of each pair is an atom.
There are no
other 2-tuples in the list and the order between the 2-tuples is undefined.
\index{module!imported functions of|)}
\item \index{module!attributes of|(}
\T{\Z{M}:module_info(attributes)} returns an association list such that
for every wild or \T{compile} attribute \T{-\Z{K}(\Z{T})} of module \TZ{M}, the list
contains a 2-tuple \T{\{\Z{K},\Z{T}\}}.  That is, the first element of each
2-tuple is an atom and the second element is an arbitrary term.
There are no other 2-tuples in the list and the order between the 2-tuples is undefined.
\index{module!attributes of|)}
\item \index{module!information about compilation|(}
\T{\Z{M}:module_info(compile)} returns an association list with information
about the compilation of module \TZ{M}.  The association list must at least
map the atom \T{time} to the time when compilation of module \TZ{M} began
(represented as a six-tuple of integers: year, month, day, hours, minutes and seconds,
like the elements in the result of the BIF \T{date/0} [\S\ref{section:date0}]
followed by the elements
in the result of the BIF \T{time/0} [\S\ref{section:time0}])
% UPPDATERA OM BIF-PROPOSAL G�R IGENOM!!!
% GMT?
and the atom \T{options} to the list of options that were given to the
compiler (including those provided through \T{compile} directives
[\S\ref{section:compile-attrib}]).
\ifNew
An implementation may include additional 2-tuples provided that it is documented
what they are.
\fi
\index{module!information about compilation|)}
\end{itemize}
An implementation may let \T{\Z{M}:module_info/1} accept
additional atoms as argument (in which case the association list
returned by \T{\Z{M}:module_info/0} should be extended accordingly,
cf.\ \S\ref{section:moduleinfo0}).
\index{module_info/1@\T{module_info/1}!function|)}
